\chapter*{Lista 4}
\addcontentsline{toc}{chapter}{Lista 4}
\chaptermark{}


% Inicio da Lista de Exercícios 
\begin{enumerate}[leftmargin=*]


\item 
Uma função $\varphi :\mathbb{R}\to\mathbb{C}$
é chamada {\it definida positiva} se para todo
$n=1,2,\ldots,$ e toda $n$-úpla $(c_1,\ldots,c_n)$
de números complexos temos
\[
	\sum_{k=1}^{n}\sum_{j=1}^{n}
	\varphi(t_j-t_k)c_j\overline{c}_k
	\geq 0,
	\quad
	\forall \ (t_1,\ldots,t_n)\in\mathbb{R}^n.
\]
Mostre que toda função característica é definida positiva.  


\item
Sejam $X$ e $Y$ v.a. com mesma distribuição. 
Mostre que: 
	\begin{itemize}
	\item[a)]
	se $X$ e $Y$ são independentes, então $X-Y$ tem
	distribuição simétrica em torno de zero;
	
	\item[b)]
	se $X$ e $Y$ assumem apenas dois valores, então 
	$X-Y$ tem distribuição simétrica em torno de zero.
	\end{itemize}



\item 
Sejam $\{X_n\}_{n\in\mathbb{N}}$ e $\{Y_n\}_{n\in\mathbb{N}}$
duas sequências independentes tais que  
$X_n \overset{\mathscr{D}}{\rightarrow} N(0,1)$ 
e
$Y_n \overset{\mathscr{D}}{\rightarrow} N(0,1)$.
Mostre que $X_n+Y_n \overset{\mathscr{D}}{\rightarrow} N(0,2)$.


\item Mostre que é possível uma sequência de funções 
de distribuição convergir em todo ponto sem o limite 
ser uma função distribuição.
(Dica: considere a v.a. constante $X_n\equiv n$)



\item Prove que se $F_n \to F$ fracamente e $F$ 
é contínua, então $F_n$ converge uniformemente para
$F$ em toda reta. 





\item Suponha que 
$X_n \overset{\mathscr{D}}{\rightarrow} N(0,1)$ 
e $\{a_n\}_{n\in\mathbb{N}}$ é uma sequência 
de números reais tal que $a_n\to a$. 
Mostre que 
$X_n+a_n \overset{\mathscr{D}}{\rightarrow} N(a,1)$ 




\item 
Sejam $X_1,X_2,\ldots$ v.a.'s tendo distribuição 
simétrica em torno de zero, isto é, 
$\mathbb{P}(0\leq X_n\leq t) 
= 
\mathbb{P}(-t\leq X_n\leq 0)
, \forall \ n\in\mathbb{N}$.
Mostre que se 
$X_n \overset{\mathscr{D}}{\rightarrow} X$




\item Sejam $X_1,X_2,\ldots$ v.a.'s iid tais que 
$X_n\sim U[0,\theta]$ (uniforme no intervalo $[0,\theta]$),
onde $\theta>0$. Mostre que 
\[
	Y_n
	=
	\sqrt{n}
	\left[ 
		\log\left(
		2\frac{X_1+\ldots+X_n}{n} - \log\theta 
		\right)
	\right]
\]
converge em distribuição para $N(0,1/3)$.




\item 
Sejam $X_1,X_2,\ldots$ v.a.'s iid, com 
$\mathbb{E}[X_1]=0$ e $\mathbb{E}[X_1^2]=2$.
Encontre o limite em distribuição das seguintes
sequências:
\begin{itemize}
	\item[a)] 
	$
	Y_n 
	= 
	\sqrt{n} 
	\displaystyle\ \frac{X_1+\ldots+X_n}{X_1^2+\ldots X_n^2};
	$
	
	
	\item[b)]
	$
	Z_n 
	= 
	\displaystyle 
	\frac{X_1+\ldots+X_n}{\sqrt{X_1^2+\ldots X_n^2}}.
	$
\end{itemize}




\item
Sejam $X_1,X_2,\ldots$ v.a.'s iid tais que 
$\mathbb{E}[X_1]=0$ e $\text{Var}(X_1)=\sigma^2$,
onde $0<\sigma^2<+\infty$. Sejam $Y_1,Y_2,\ldots$
v.a.'s iid tais que $\mathbb{E}[X_1]=\mu$, $\mu\in\mathbb{R}$.
Prove que 
\[
	\frac{Y_1+\ldots+Y_n}{n}
	+
	\frac{X_1+\ldots+X_n}{\sqrt{n}}
	\overset{\mathscr{D}}{\rightarrow}
	N(\mu,\sigma^2).
\] 




\item Sejam $X_1,X_2,\ldots$ uma sequência de v.a.'s independentes
e $X$ uma v.a. discreta com $\mathbb{P}(X=n)=p_n,\ n\geq 1$.
Seja $F_n$ a distribuição de $X_n$ e $\varphi_n$
sua função característica. Mostre usando a esperança condicional 
que a v.a. 
\[
	Y= \sum_{n=1}^{\infty} 1_{\{X=n\}}X_n
\]  
tem função característica dada por $\varphi = \sum_{n\geq 1}p_n\varphi_n$.






\item 
Suponha que $\{(X_n,Y_n)\}_{n\in\mathbb{N}}$ 
seja uma sequência de vetores aleatórios tomando 
valores em $\mathbb{R}^2$ que converge em distribuição 
para $(X,Y)$. Mostre que 
$X_n+Y_n \overset{\mathscr{D}}{\rightarrow} X+Y$.



\item 
Sejam $(\Omega,\F,\P)$ um espaço de probabilidade 
e $(M,d)$ um espaço métrico. 
Para cada $n\in\mathbb{N}$ seja $X_n:\Omega\to M$ uma 
aplicação aleatória, mensurável segundo as $\sigma$-álgebras
$\F$ e $\mathscr{B}(M)$.  
\begin{itemize}
	\item[a)] Mostre que $X_n$ converge em probabilidade
	para uma v.a. constante $c$ quase certamente se, e somente se,
	a sequência de medidas probabilidade $Q_n\equiv \P\circ X_n^{-1}$
	converge fracamente para a medida de probabilidade 
	$\delta_c:\mathscr{B}(M)\to [0,1]$, definida para cada 
	mensurável $E\in\mathscr{B}(M)$ por 
	\[
		\delta_c(E) 
		=
		\begin{cases}
		1,&\text{se}\ c\in E;
		\\
		0,&\text{caso contrário.}
		\end{cases}
	\] 
	[Convergência em probabilidade neste contexto significa 
	que para todo $\varepsilon>0$ dado, 
	temos $\P(d(X_n,c)>\varepsilon)\to 0$, quando $n\to\infty$. 
	]
	
	
	\item[b)]
	Mostre que se $\{X_n\}_{n\in\mathbb{N}}$ é uma
	sequência de v.a.'s em $(\Omega,\F,\P)$ que converge em probabilidade 
	para $X$, 
	então $\P\circ X_n^{-1}$ converge fracamente para $\P\circ X$.
	
	
\end{itemize}






	\item
	Sejam $v_1,\ldots,v_n\in \mathbb{R}^n$ tais que 
	$\|v_i\|=1$, 
	onde 
	$\|\cdot\|$ denota a norma euclidiana em $\mathbb{R}^n$.
	\begin{itemize}
	\item[a)]
	Mostre que existem $\varepsilon_1,\ldots,\varepsilon_n\in\{-1,1\}$
	tais que
	\[ 
		\|\varepsilon_1 v_1 +\ldots+\varepsilon_n v_n\|\leq \sqrt{n}
	\] 
	
	\item[b)] 
	Mostre que também existem $\delta_1,\ldots,\delta_n\in\{-1,1\}$
	tais que
	\[
		\sqrt{n} \leq \|\delta_1 v_1 +\ldots+\delta_n v_n\|
	\]
	(Dica: Calcule a esperança da v.a. $X=\|X_1v_1+\ldots+X_n v_n\|^2$, 
	onde as v.a.'s $X_i$ são independentes e uniformemente distribuídas 
	em $\{-1,1\}$)
	\end{itemize}
	





\item 
Seja $\mu:\mathscr{B}(\mathbb{R})\to[0,1]$ uma medida de probabilidade
e $\varphi$ a função característica associada a $\mu$.
Suponha que 
\[
	\int_{\mathbb{R}} |x|^n d\mu(x)<\infty.
\]
para algum $n\in\mathbb{N}$.
Mostre que $\varphi$ tem derivadas contínuas até ordem $n$ 
e além do mais que 
\[
\varphi^{(n)}(t)
=
\int_{\mathbb{R}} (ix)^n \ e^{itx}\ \ d\mu(x).
\]







\item Use o exercício anterior e a expansão em série de 
Taylor de $\exp(-t^2/2)$ para mostrar que se 
$X\sim N(0,1)$ então 
\[
	\mathbb{E}[X^{2n}]
	=
	\frac{(2n)!}{2^n n!}
	=
	(2n-1)(2n-3)\cdot\ldots\cdot 3\cdot 1
	\equiv 
	(2n-1)!!
\] 







\item (Cotas do tipo Chernoff)  Seja $\{X_i,\ i=1,\ldots, n\}$ uma coleção  
de v.a.'s mutuamente independentes tais que 
$\P(X_i=1)=\P(X_i=-1)=1/2$.
Denote por $S_n= X_1+\ldots +X_n$. Mostre que
para todo $a>0$ temos 
\[
	\P(\{S_n>a\})< e^{-\frac{a^2}{2n}}
\]
(Dica: Observe que para todo $\lambda>0$ temos 
$$
\P(\{S_n>a\})=\P(\{\exp(\lambda S_n)>\exp(\lambda a)\}) 
$$
aplique a desigualdade de Markov e use que 
$\cosh^n(\lambda)\leq \exp(\lambda^2 n/2).$
)


\item Usando as ideias do exercício anterior 
mostre que se $X\sim N(0,1)$ então 
\[
	\P(X>a)\leq \exp\left(-\frac{a^2}{2}\right).
\]
Note que esta cota é semelhante aquela que
podemos obter no limite quando $a\to \infty$, 
usando explicitamente a forma
da densidade da normal como mostrado abaixo 
\[
\P(X>a) 
= 
\frac{1}{\sqrt{2\pi}}
\int_{a}^{\infty} e^{-t^2/2}\, dt
\sim 
\frac{1}{\sqrt{2\pi a}}
\exp\left(-\frac{a^2}{2}\right)
\]


\begin{center}
{\red CONTINUA ...}
\end{center}





\end{enumerate}



