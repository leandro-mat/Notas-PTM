\chapter*{Lista 2}
\addcontentsline{toc}{chapter}{Lista 2}
\chaptermark{}


% Inicio da Lista de Exercícios 
\begin{enumerate}[leftmargin=*]


\item 
Sejam $B_1,\ldots, B_n$ eventos independentes. 
Mostre que 
	\[
		\P \left( \bigcup_{i=1}^n B_i \right)
		=
		1- \prod_{i=1}^n [1-\P(B_i)].
	\]


\item 
Qual é a quantidade mínima de pontos que um espaço de
probabilidade $(\Omega,\F,\P)$ deve ter para que possa
existir $n$ eventos independentes $B_1,\ldots,B_n$ 
que não tenham probabilidade zero ou um. 


\item Se $\{A_n\}$ é uma sequência de eventos independentes,
mostre que 
	\[
		\P \left( \bigcap_{i=1}^n A_n \right)
		=
		\prod_{n=1}^{\infty} \P(A_n).
	\]



\item 
Considere o espaço de probabilidade $([0,1],\mathscr{B}([0,1]),\lambda)$,
onde $\lambda$ é a medida de Lebesgue em $[0,1]$.
Seja $X:[0,1]\to\R$ a v.a. dada por $X(\omega)=\omega$.
	\begin{itemize}
		\item[a)]
		Existe alguma variável aleatória que é independente de 
		$X$ e não constante quase certamente ?
		
		\item[b)] Defina a v.a. $Y=X(1-X)$. Construa uma v.a.
		$Z$ tal que $Z$ e $Y$ sejam independentes.
	\end{itemize}



\item 
Suponha que $X$ seja uma variável aleatória.
	\begin{itemize}
		\item[a)] 
		Mostre que a v.a. $X$ é independente de si mesma 
		se, e somente se, existe uma constante $c$ tal que 
		$\P(X=c)=1$.
		
		\item[b)]
		Se existe uma função mensurável 
		$g:(\R,\mathscr{B}(\R))\to(\R,\mathscr{B}(\R))$
		tal que $X$ e $g(X)$ são independentes, então prove que
		existe uma constante $c$ tal que $\P(g(X)=c)= 1$. 				
	\end{itemize}	 
		
		
\item 
$^*$ Seja $\{X_k,k\in\N\}$ uma sequência de v.a's iid com distribuição
comum $F$. Seja $\pi$ uma permutação de $\{1,\ldots,n\}$. 
Mostre que 
	\[
		(X_1,\ldots,X_n) 
		\,{\buildrel d \over =}\, 
		(X_{\pi(1)},\ldots,X_{\pi(n)}),
	\]
onde ${\buildrel d \over =}$ significa que os dois vetores
têm a mesma distribuição conjunta. 



\item 
Se $A,B,C$ são eventos independentes, mostre que 
ambos $A\cup B$ e $A\setminus B$ são independentes de 
$C$.



\item Se $X$ e $Y$ são variáveis aleatórias independentes
e $f,g:\R\to\R$ são funções mensuráveis porque $f(X)$ e $g(Y)$
são independentes ?
\\
(Nenhuma conta é necessária).




\item
Suponha que $\{A_n\}$ é uma sequência de eventos independentes
satisfazendo $\P(A_n)<1$, para todo $n\in \N$. Mostre que 
	\[
		\P \left( \bigcup_{i=1}^n A_n \right) = 1
		\quad
		\Longleftrightarrow
		\quad
		\P(\limsup A_n) =1.
	\]
Dê um exemplo mostrando que a condição $\P(A_n)<1$ 
não pode ser removida.


\item 
Suponha que $\{X_n, n\in\N\}$ seja uma sequência de variáveis 
aleatórias independentes. Mostre que 
\[
	\P \left(  \sup_{n\in\N} X_n <\infty  \right)=1
	\quad
	\Longleftrightarrow
	\quad
	\sum_{n=1}^{\infty} \P(X_n>M) <\infty,\quad 
	\text{ para algum}\ M>0.
\]



\item
Use o Lema de Borel-Cantelli para mostrar que dada 
qualquer sequência de v.a.'s $\{X_n, n\in\N\}$, tomando valores 
reais, existe uma sequência 
$\{c_n\}$ (que não depende da sequência $\{X_n\}$) 
tal que 
	\[
		\P \left( \lim_{n\to\infty} \frac{X_n}{c_n}=0 \right)=1.
	\]
Dê uma descrição precisa das possíveis escolhas da sequência $c_n$.



\item 
O seguinte resultado é útil para ser usado juntamente 
com a Lei Zero-Um de Borel: suponha que $\{a_n\}$ 
e $\{b_n\}$ sejam duas sequências de números reais
não-negativos, satisfazendo $a_n \sim b_n$, 
isto é, $a_n/b_n\to 1$, quando $n\to\infty$.
Mostre que 
	\[	
		\sum_{n=1}^{\infty} a_n <\infty
		\quad
		\Longleftrightarrow
		\quad
		\sum_{n=1}^{\infty} b_n <\infty.
	\]	




\item 
Seja $\{X_n, n\in\N\}$ uma sequência de v.a.'s iid com
	\[
		\P(X_1=1)=p=1-\P(X_1=0).
	\]
Qual é a probabilidade que o padrão $1,0,1$ apareça
infinitas vezes ? 
\\
(Dica. Considere os eventos $A_k=\{X_k=1,X_{k+1}=0,X_{k+2}=1\}$ e 
olhe para a sequência $A_1,A_4,A_7,\ldots$).



\item 
Em uma sequência de v.a.'s de Bernoulli $\{X_n, n\in\N\}$
com 
	\[
		\P(X_n=1)=p=1-\P(X_n=0).
	\]
Seja $A_n$ o evento ocorrem $n$ uns consecutivos entre as 
observações $2^n$ e $2^{n+1}$. Mostre que se $p \geq 1/2$,
então $A_n$ ocorre infinitas vezes com probabilidade 1.
\\
Dica. Prove algo como 
\[
	\P(A_n) \geq 
	1-(1-p^n)^{\frac{2^n}{n}}
	>
	1-e^{-\frac{(2p)^n}{2n}}.		
\]


\item 
 

\end{enumerate}