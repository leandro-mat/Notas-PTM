\chapter*{Lista 3}
\addcontentsline{toc}{chapter}{Lista 3}
\chaptermark{}


% Inicio da Lista de Exercícios 
\begin{enumerate}[leftmargin=*]


\item 
Suponha que $X$ e $Y$ são duas v.a.'s independentes
tendo distribuição $\mathbb{P}_X$ e $\mathbb{P}_{Y}$,
respectivamente. Prove que a distribuição de $X+Y$ 
é dada pela convolução $\mathbb{P}_X*\mathbb{P}_Y$
definida por
\[
	\mathbb{P}_X*\mathbb{P}_Y(B)
	=
	\int_{\mathbb{R}} \mathbb{P}_X(B-x)d\mathbb{P}_Y(x),
\]
onde $B\in\mathscr{B}(\mathbb{R})$ e 
$B-x=\{b-x\in\mathbb{R}; b\in B\}$.










\item 
Seja $X_1,X_2,\ldots, X_n$ uma coleção de v.a.'s iid.
definidas em um espaço de probabilidade $(\Omega,\F,\P)$.
Suponha que estas variáveis aleatórias tenha distribuição 
comum $Q$.
%
%
%
	\begin{itemize}
	\item[i)]
	denote por $m$ a medida de Lebesgue em $\mathbb{R}$ e
	suponha que a derivada de Radon-Nykodin 
		\[
			\frac{dQ}{dm}(x) = \lambda e^{-\lambda x}1_{[0,\infty)}(x).
		\]
	para algum $\lambda>0$. Mostre que $X_1+\ldots+X_n$ 
	tem distribuição determinada pela medida
	de probabilidade $Q^{*n}$ definida em $\mathscr{B}(\mathbb{R})$
	por 
	\[
		Q^{*n}(B) = 
		\int_{B} \lambda^n \frac{x^{n-1}}{(n-1)!}e^{-\lambda x}
		1_{[0,\infty)}(x)\, dm(x).
	\]
	(OBS: A distribuição $Q^{*n}$ determinada acima é conhecida como distribuição 
	Gama de parâmetros $n,\lambda$.)
	
	\item[ii)] 
	Sejam $\mu\in \mathbb{R}$  e $\sigma^2>0$. 
	Suponha que  
		\[
			\frac{dQ}{dm}(x) = \frac{1}{\sqrt{2\pi \sigma^2}}
			\exp\left( -\frac{(x-\mu)^2}{\sigma^2} \right).
		\]	
	Mostre que $X_1+\ldots+X_n$ tem distribuição dada por 
		\[
			Q^{*n}(B) = 
			\int_{B}\frac{1}{\sqrt{2\pi n\sigma^2}}
			\exp\left( -\frac{(x-n\mu)^2}{n\sigma^2} \right)
			\, dm(x).
		\]
	(OBS: A distribuição determinada por $Q$ neste item é conhecida como distribuição 
	Gaussiana ou Normal com parâmetros $\mu$ e $\sigma^2$, notação $N(\mu,\sigma^2)$.
	Note que a distribuição determinada por $Q^{*n}$ é também uma Normal com 
	parâmetros $n\mu$ e $n\sigma^2$.)
	
	
	\item[iii)] 	
	Suponha que $X$ seja uma v.a. com distribuição $N(0,1)$.
	Encontre a distribuição de $X^2$ e calcule $Q*Q$.
	Dica: $ \int_{0}^{1} u^{-1/2}(1-u)^{-1/2}\, du= \pi $
	\end{itemize}




	





\item Seja $(\Omega,\F,\P)$ um espaço de probabilidade.
Inicialmente mostre que para todo $p>0$ e $x\geq 0$ temos 
	\[
		x^p = p \int_{0}^{x} y^{p-1} \, dy. 	
	\]
Em seguida, usando o resultado acima e o Teorema de Tonelli 
mostre que para qualquer v.a. $X$ temos 	 
	\[
		\mathbb{E}|X|^p 
		= 
		p\int_{0}^{\infty} y^{p-1} \P(\{|X|>y\})\, dy.
	\]
(OBS: Em alguns casos a igualdade acima pode 
significar simplesmente $+\infty=+\infty$.)









\item
Suponha que $X_1,X_2,\ldots$ seja uma sequência de v.a.'s  
cada uma tendo distribuição $Q=\mathbb{P}\circ X_n^{-1}$.
	\begin{itemize}
		\item[i)] 
		Mostre que se $\mathbb{E}[|X_1|]<\infty$
		então $\P(\limsup \{X_n>n\})=0$.
		Dica. Use o exercício anterior e o Lema de Borel-Cantelli.
		
		\item[ii)] 
		Assuma que a sequência de v.a.'s $\{X_n\}$ é independente e
		que $\mathbb{E}[|X_n|]=+\infty$, para todo $n\in\mathbb{N}$.
		Mostre que $\P(\limsup \{X_n>n\})=1$. 
	\end{itemize}





\item
Sejam $(\Omega,\F,\P)$ e $(\mathbb{R}^n,\mathscr{B}(\mathbb{R}^n),\mu)$
espaços de probabilidade.
Suponha que a aplicação $T:\Omega\to\mathbb{R}^n$ seja  
Borel mensurável, isto é, $T^{-1}(B)\in \F$, para todo 
$B\in \mathscr{B}(\mathbb{R}^n)$.
Seja $\mathscr{A}\subset \mathscr{B}(\mathbb{R}^n)$ uma 
sub-$\sigma$-álgebra. Defina 
$
\mathcal{E}= 
\{E\in \mathcal{F}: 
E=T^{-1}(A)\ \text{para algum}\ A\in\mathscr{A}\}.
$
Mostre que 
%
%
	\begin{itemize}
		\item[i)]
		para qualquer $A\in \mathscr{A}$ que 
		\[
		\P(T^{-1}A|\mathcal{E})(\omega)
		=
		\mu(A|\mathscr{A})(T(\omega)),
		\quad \P-\text{quase certamente}. 
		\]	
		
		\item[ii)]
		Considere $n=2$ e seja $\mu$ a distribuição de 
		um vetor aleatório $(X,Y)$ definido em $(\Omega,\F,\P)$.
		Suponha que $(X,Y)$ tem densidade $f$, isto é, 
		para todo Boreliano $A\in\ \mathscr{B}(\mathbb{R}^2)$
		temos que
		\[
		\P((X,Y)\in A) = \mu(A) = \int_{A}f(x,y) \, dxdy. 	
		\]
		Mostre para todo $B\in\mathscr{B}(\mathbb{R})$
		que é válida a seguinte igualdade $\P$-quase certamente
		\[
			\P(Y\in B|X)(\omega) 
			= 
			\frac{\int_{B}f(X(\omega),y)\, dy }
			{\int_{\mathbb{R}} f(X(\omega),y)\, dy }.
		\]
	\end{itemize}



\item 
Suponha que $\mu$ e $\nu_x$ são medidas de probabilidade 
na reta. Para qualquer qualquer Boreliano 
da reta $B$ fixado, suponha que  $x\mapsto \nu_x(B)$ é uma função 
borel mensurável. 
%
%
	\begin{itemize}
		\item[i)]
		Mostre que 
		\[
			\eta(E) =\int_{\mathbb{R}} \nu_x(\{y\in\mathbb{R}: (x,y)\in E\} )
					\, d\mu(x).
		\] 
		define uma medida de probabilidade em 
		$(\mathbb{R}^2,\mathscr{B}(\mathbb{R}^2))$
		
		\item[ii)]
		Suponha que o vetor $(X,Y)$ tenha distribuição $\eta$.
		Mostre que $\nu_{X}$ é uma versão da distribuição 
		condicional de $Y$ dado $X$. 
	\end{itemize}



\item Mostre que se $X$ e $Y$ são duas v.a. independentes
então $\mathbb{E}[X|Y]=\mathbb{E}[X]$ e conclua que 
$\mathbb{E}[XY]=\mathbb{E}[X]\mathbb{E}[Y]$.









\item Sejam $(\Omega,\F,\P)$ um espaço de probabilidade e 
$\mathscr{B}$ uma sub-$\sigma$-álgebra de $\F$.
Prove que para quaisquer variáveis aleatórias 
limitadas $X$ e $Y$ que 
	\[
	\mathbb{E}[Y\cdot \mathbb{E}[X|\mathscr{B}]]
	=
	\mathbb{E}[X\cdot \mathbb{E}[Y|\mathscr{B}]].
	\]







\item 
Suponha que $X$ e $Z$ são v.a. independentes com
distribuição normal. Seja $Y=X+bZ$, isto é, 
$X$ com adição de um termo de ``ruído'' independente $bZ$.
Calcule $\mathbb{E}[X|\sigma(Y)]$. 


\item
Seja $k$ um inteiro positivo e 
$\Omega=\{0,1,2,\ldots,k-1\}^{\mathbb{N}}$. 
Seja $\mathcal{F}$ a $\sigma$-álgebra gerada 
pelas v.a.'s $\{X_n\}$, onde para 
cada $n\in\mathbb{N}$ e $\omega=(\omega_1,\omega_2,\ldots)$ 
a v.a. $X_n$ é definida por $X_n(\omega)=\omega_n$. 
Sejam $\nu$ uma medida de probabilidade definida no conjunto das partes
de $\{0,1,2,\ldots,k-1\}$ e $\P = \prod_{i\in\mathbb{N}}\nu$ 
a medida produto dada pelo Teorema da Extensão de Kolmogorov.
Considere a aplicação $\sigma:\Omega\to\Omega$ 
dada pelo ``shift'' para esquerda, isto é, 
	\[
		\sigma(\omega_1,\omega_2,\omega_3,\ldots)
		=
		(\omega_2,\omega_3,\omega_4,\ldots).
	\] 
Mostre que $\P$ é invariante pelo ``shift'', isto é, 
para todo $E\in\mathcal{F}$ temos que 
	\[
	\P(\sigma^{-1}(E)) = \P(E).
	\] 











\item
Seja $\Omega=\{-1,1\}$, $\mathcal{F}=\mathcal{P}(\Omega)$ e 
$\kappa:\mathcal{F}\to \mathbb{R}$ 
a medida de contagem em $\Omega$.
Mostre que para toda constante $m\in\mathbb{R}$ que
\[
	\int_{\{-1,1\}} e^{m\omega}\, d\kappa(\omega)
	=
	2\cosh(m).
\]


\item 
Seja $n\in\mathbb{N}$ fixado. 
Considere produto cartesiano $\Omega_n=\{-1,1\}^n$, 
$\mathcal{F}_n=\mathcal{P}(\Omega)$.

	\begin{itemize}

		\item[i)]
		Mostre que  
		$\kappa_n = \prod_{i=1}^n \kappa$, 
		onde $\kappa$ é a medida de contagem em $\{-1,1\}$, 
		é a medida de contagem em $\Omega_n$.
		
		
		\item[ii)]
		Defina a função 
		\[
			H_n(\omega)
			\equiv
			H_n(\omega_1,\ldots,\omega_n)
			= 
			-\sum_{i=1}^{n-1} \omega_i\omega_{i+1}
		\]
		Calcule
		\[
			Z_n(\beta)
			=
			\int_{\Omega_n} 
			e^{-\beta H_n(\omega)}\, 
			d\kappa_n(\omega).
			\]
		Dica. Use o Teorema de Fubini.		
		\end{itemize}





\item 
Fixe $n\in\mathbb{N}$ e $\beta>0$. Sejam 
$\Omega_n$, $\mathcal{F}_n$ e $Z_n(\beta)$,
como definidos no exercício anterior. 
Mostre que $\mu_{n,\beta}:\mathcal{F}_n\to\mathbb{R}$
definida para cada $A\in \mathcal{F}_n$ por 
\[
	\mu_{n,\beta}(A)
	=	
	\frac{1}{Z_{n}(\beta)}
	\int_{\Omega_n} 
	1_{A}(\omega)
	e^{-\beta H_n(\omega)}\, 
	d\kappa_n(\omega)
\]
define uma medida de probabilidade.











\item Sejam $n$ um inteiro positivo e $\beta>0$ fixados.
Considere o espaço de probabilidade 
$(\Omega_n,\mathcal{F}_n,\mu_{n,\beta})$
definido nos dois exercícios anteriores.
Seja $\{X_i,i=1,\ldots,n\}$ uma coleção 
de v.a.'s definidas neste espaço de probabilidade por 
$X_i(\omega)=\omega_i$.
Mostre para qualquer $i\in \{1,\ldots,n\}$ que
	\[
		\mathbb{E}[X_i] 
		\equiv
		\int_{\Omega_n} X_i(\omega)
		\, d\mu_{n,\beta}(\omega)
		=	 
		0.
	\]









\item (Percolação em dimensão 1.)
Considere o grafo $G=(V,E)$, 
onde $V=\mathbb{Z}$ e $E=\{ i,j\in\mathbb{Z}: |i-j|=1\}$.
Seja $\Omega=\{0,1\}^{E}$.
Para cada $e\in E$ defina $X_e(\omega)=\omega_e$.
Mostre que para cada $p\in [0,1]$ fixado, 
existe um medida de probabilidade 
$\P_p$ definida em $\F=\sigma(\{X_{e},e\in E\})$
tal que para todo $e\in E$ 
temos que $\P(\{X_e=1\})=p=1-\P(\{X_e=0\})$
e além do mais as v.a's $\{X_e\}_{e\in E}$ são independentes
no espaço de probabilidade $(\Omega,\F,\P_p)$.






\item
Sejam $G=(V,E)$ e  $(\Omega,\F,\P_p)$ o grafo e o espaço de probabilidade 
definido no exercício anterior, respectivamente.
Dados dois vértices $x,y\in V$ dizemos que 
$\gamma = \{e_1,\ldots,e_n\}$ é um caminho 
conectando $x$ a $y$ se 
\begin{itemize}
\item
$e_i\in E, \forall i=1,\ldots n$;
\item 
$e_i\neq e_j$ se $i\neq j$;
\item
$e_1=\{x,x_1\}$ e $e_n=\{x_{n-1},y\}$.
\end{itemize}
Dada uma configuração $\omega\in\Omega$
dizemos que um caminho $\gamma=\{e_1,\ldots,e_n\}$
é um caminho aberto em $\omega$, se 
$X_{e_1}(\omega),X_{e_2}(\omega),\ldots,X_{e_n}(\omega)=1$.
Denote por 
$
\{x \longleftrightarrow y\}
=
\{\omega\in\Omega: \exists\ \gamma\ \text{caminho aberto em}\ \omega\ 
\text{conectando}\ x \ \text{a}\ y\}.
$
Mostre que $\{x \longleftrightarrow y\}\in \F$ e
calcule $\P_p(\{x \longleftrightarrow y\})$.










\item
Sejam $G=(V,E)$ e  $(\Omega,\F,\P_p)$ o grafo e o espaço de probabilidade 
definido no exercício anterior, respectivamente.
Para cada $x\in V$ e $\omega\in \Omega$ 
defina a componente conexa aberta de $x$ em $\omega$ por 
\[
C_{\omega}(x) \equiv 
\{y\in V: 
\exists\ \gamma \ \text{caminho aberto em}\ \omega \ \text{ligando}\
x\ \text{a}\ y\}.
\]
Mostre que 
\begin{itemize}
\item[i)] para cada $x\in V$ fixado que 
$\omega\mapsto \#C_{\omega}(x)$ é uma 
v.a. a valores em $\overline{\mathbb{R}}$;

\item[ii)]
o conjunto 
$
\{x \longleftrightarrow \infty\} 
\equiv 
\{\omega\in\Omega: \#C_{\omega}(x)=+\infty\}\in \F
$;


\item[iii)] generalize o exercício sobre invariância por translação da medida 
produto de $\mathbb{N}$ para $\mathbb{Z}$ e em seguida, mostre que 
$
\P_p(\{x \longleftrightarrow \infty\})
=
\P_p(\{0 \longleftrightarrow \infty\});
$

  
\item[iv)] para cada $p\in [0,1]$ 
calcule $\theta(p)\equiv\P_p(\{0 \longleftrightarrow \infty\})$.
Dica: use Borel-Cantelli.

\item[v)]
se $p\neq 1$, para todo $x\in V$ mostre que 
$\#C_{\omega}(x)<\infty$, $\P_p$-q.c.. 
\end{itemize}
\end{enumerate}