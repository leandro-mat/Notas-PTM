\chapter[Aula 9]{A Integral de Lebesgue}
\chaptermark{}

\section{A Integral de Funções em $M^{+}(\Omega,\F)$}

A principal diferença entre a integral de Riemann e a integral
de Lebesgue é que na integral de Riemann é feita uma 
partição do dominío da função, enquanto que na construção da
 integral de Lebesgue é feita uma partição da imagem da função.
Esse novo  "aproach" permite construir um conceito de integral muito
mais geral e com propriedades muito mais fortes  que a integral de Riemman, 
como veremos. Também discutiremos quando esses dois conceitos 
de integral coincidem.

O objeto que desempenha um papel fundamental na 
construção da integral de Lebesgue é a função simples
cuja definição é dada abaixo 
%
%
\begin{definicao}\label{def-funcao-simples}
Uma função $f:\Omega\to\R$ é dita uma função simples
se sua imagem $f(\Omega)$ possui apenas um número
finito de elementos.
\end{definicao}
%
%
%
%
\begin{observacao}
Lembramos que quando estamos trabalhando com 
espaços de probabilidade podemos chamar uma
função simples de v.a. discreta.
\end{observacao}


\begin{observacao}
Quando for dito $f\in M^{+}(\Omega,\F)$
e que tal função é uma função simples, 
estamos assumindo que $f$ toma valores apenas em $\R$.
\end{observacao}


Uma função real  mensurável simples 
$f\in M^{+}(\Omega,\mathcal{F})$ 
pode ser escrita na forma 
\begin{equation}\label{eq-funcao-simples-F-mensuravel}
f=\sum_{j=1}^{n}a_j 1_{E_j}
\end{equation}
onde $1_{E_j}$ é a função característica de um conjunto $E_j\in \mathcal{F}$. 
Não é difícil ver que existem várias maneiras de 
escrever uma mesma função simples na 
forma eqref{eq-funcao-simples-F-mensuravel}. 
Entretanto existe uma maneira de
representar tais funções que será nossa preferida 
e será chamada de
\emph{representação padrão}
\index{Representação Padrão de Funções Simples} de $f$, 
que é caracterizada do seguinte modo: 
os $a_j$'s devem ser distintos e os $E_j$'s devem ser disjuntos, 
não vazios e tais que $\bigcup_{j=1}^nE_j=\Omega$. 
%
%
\begin{exercicio}
Mostre que a representação padrão de uma função 
simples $f:\Omega\to\R$ é única.
\end{exercicio}
%
%
%
%
\begin{definicao}[Integral de Funções Simples não-negativas]
\label{def-integral-funcao-simples-positiva}
Sejam $(\Omega,\F,\mu)$ um espaço de medida e 
$\varphi:\Omega\to [0,+\infty)$ uma função simples cuja representação 
padrão é dada por $\varphi = \sum_{j=1}^n a_j1_{E_j}$.
Definimos a integral de $\varphi$ com respeito a $\mu$ 
sendo o número real estendido 
	\[
		\int_{\Omega} \varphi \, d\mu 
		=
		\sum_{j=1}^n a_j\, \mu(E_j).
	\]
\end{definicao}

Na definição acima deve ser lembrada nossa convenção 
que $0\cdot(+\infty)=0$. Portanto a integral da 
função identicamente nula é zero independentemente do 
espaço $\Omega$ ter medida finita ou não. 
Devemos observar que a integral de qualquer função 
simples não-negativa está bem definida em $\overline{\R}$.
Passamos agora à prova de alguma propriedades 
básicas desta integral.








\begin{lema}\label{lema-lin-int-funcao-simples-positiva}
Sejam $(\Omega,\F,\mu)$ um espaço de medida, 
$\varphi,\psi \in M^+(\Omega,\F)$ 
funções simples e $c\in [0,+\infty)$,
então
\begin{itemize}
	\item[a)]
	\(\displaystyle 
		\int_{\Omega} c\varphi\, d\mu
		=
		c\int_{\Omega} \varphi \, d\mu.
	\)
	

	\item[b)]
	\(\displaystyle 
		\int_{\Omega} (\varphi+\psi)\, d\mu
		=
		\int_{\Omega} \varphi \, d\mu
		+
		\int_{\Omega} \psi \, d\mu.
	\)
	
	\item[c)] A aplicação $\lambda:\F\to\overline{\R}$
	dada por 
		\(\displaystyle
			\lambda(E) = \int_{\Omega} \varphi 1_{E}\, d\mu
		\)
	é uma medida em $\F$.
\end{itemize}
\end{lema}



\begin{proof}
Prova do item a). Se $c=0$ então $c\varphi$ é a função 
identicamente nula e portanto a igualdade é válida. 
Se $c>0$ e $\varphi\in M^{+}(\Omega,\F)$ tem 
representação padrão 
	\[
		\varphi  
		=
		\sum_{j=1}^n a_j 1_{E_j} 
	\]
então $c\varphi\in M^{+}(\Omega,\F)$ e tem 
representação padrão dada por 
	\[
		c\varphi  
		=
		\sum_{j=1}^n ca_j 1_{E_j} 
	\]
Portanto temos que 
	\[
	\int_{\Omega} c\varphi\, d\mu
	=
	\sum_{j=1}^n ca_{j} \mu(E_j)
	=
	c \sum_{j=1}^n a_{j} \mu(E_j)
	=
	c \int_{\Omega} \varphi\, d\mu.
	\]



Prova do item b). Suponha que $\varphi$ e $\psi$ têm as 
seguintes representações padrões
	\[
		\varphi  
		=
		\sum_{j=1}^n a_j 1_{E_j} 
		\qquad
		\text{e}
		\qquad
		\psi 
		=
		\sum_{k=1}^m b_k 1_{F_k}.
	\]
Daí temos a seguinte representação para a soma 
destas funções 
	\[
		\varphi+\psi
		=
		\sum_{j=1}^n\sum_{k=1}^m (a_j+b_k) 1_{E_jF_k}.
	\]
Apesar da coleção $\{E_j\cap F_k\}$ ser disjunta 
quando $j=1,\ldots,n$ e $k=1,\ldots,m$ a representação
acima não é necessariamente a 
representação padrão de $\varphi+\psi$ 
pois não é garantido que $E_j\cap F_k\neq \emptyset$ 
e nem que a coleção $\{a_j+b_k\}$ seja uma coleção de
números reais distintos. Para obter a representação 
padrão procedemos da seguinte maneira. 
Seja $c_h$, com $1\leq h\leq p\leq nm$ a coleção 
de todos os número distintos da coleção 
$\{ a_j+b_k: j=1,\ldots,n;\ k=1,\ldots,m\}$. 
Para cada $h=1,\ldots, p$ defina 
	\[
		G_h
		=
		\bigcup_{ \substack {j,k : E_j\cap E_k\neq \emptyset \\a_j+b_k=c_h}  }
		E_j\cap E_k.
	\] 
Assim temos que 
	\begin{equation}\label{eq-aux1-linearidade-integral-func-simples-positiva}
		\mu(G_h)
		=
		\sum_{ \substack {j,k : E_j\cap E_k\neq \emptyset \\a_j+b_k=c_h}  }
		\mu(E_j\cap E_k).
	\end{equation} 
Já que a representação padrão de $\varphi+\psi$ é dada por 
	\[
		\varphi+\psi = \sum_{h=1}^p c_h 1_{G_h}.
	\]
Pela definição da integral e pela igualdade 
\eqref{eq-aux1-linearidade-integral-func-simples-positiva} 
temos que 
\begin{align*}
\int_{\Omega}(\varphi+\psi)\, d\mu 
&= 
\sum_{h=1}^p c_h \mu(G_h)
\\
&=
\sum_{h=1}^p 
	c_h 
	\sum_{ \substack {j,k : E_j\cap E_k\neq \emptyset \\a_j+b_k=c_h}  }
		\mu(E_j\cap E_k)
\\
&=
\sum_{h=1}^p  \ \ 
	\sum_{ \substack {j,k : E_j\cap E_k\neq \emptyset \\a_j+b_k=c_h}  }
		c_h\mu(E_j\cap E_k)
\\
&=
\sum_{h=1}^p \ \  
	\sum_{ \substack {j,k : E_j\cap E_k\neq \emptyset \\a_j+b_k=c_h}  }
		(a_j+b_k)\mu(E_j\cap E_k)
\\
&=
	\sum_{j=1}^n \   
	\sum_{k=1}^m
		(a_j+b_k)\mu(E_j\cap E_k)
\\
&=
	\sum_{j=1}^n \   
	\sum_{k=1}^m
		a_j\mu(E_j\cap E_k)
	+
	\sum_{j=1}^n \   
	\sum_{k=1}^m
		b_k\mu(E_j\cap E_k)
\\
&=
	\sum_{j=1}^n 
		a_j\mu(E_j)
	+
	\sum_{k=1}^m
		b_k\mu(E_k)
\\
&=
\int_{\Omega} \varphi\, d\mu 
+
\int_{\Omega} \psi \, d\mu
\end{align*}

Prova do item c). Usando os itens a), b) e a representação
\[
	\varphi 1_{E} 
	=
	\sum_{j=1}^n a_j 1_{E_j\cap E}
\]
temos que 
	\[
		\lambda(E)
		=
		\int_{\Omega}\varphi 1_{E}\, d\mu
		=
		\sum_{j=1}^n a_j \int_{\Omega}1_{E_j\cap E} \, d\mu
		=
		\sum_{j=1}^n a_j \mu(E_j\cap E).	
	\]
Já que aplicação $E\mapsto \mu(E_j\cap E)$ define uma medida 
em $\F$, $a_j$'s são não negativos e combinações lineares
de medidas com coeficientes não negativos são medidas então o 
lema está provado.
\end{proof}



Agora estamos preparados para introduzir a definição de
integral para qualquer função em $M^{+}(\Omega,\F)$. 
Como no caso de funções simples não-negativas 
está integral estará 
sempre bem definida como um número em $\overline{\R}$.



\begin{definicao}[Integral de Função Não-Negativa]
\label{def-integral-func-mens-nao-negativa}
A integral de Lebesgue de uma função $f\in M^{+}(\Omega,\F)$
com respeito a medida $\mu$ é definida pela seguinte número real estendido 
	\[
		\int_{\Omega} f\, d\mu
		=
		\sup_{ \substack{ 	\varphi \in M^{+}(\Omega,\F)
							\\
							\varphi\ \text{simples}
							\\ 
							0\leq\varphi(\omega)\leq f(\omega)
							,\ \forall \omega\in\Omega }  } 
		\ 
		\int_{\Omega} \varphi \, d\mu
	\]
\end{definicao}






Observe que se $f\in M^{+}(\Omega,F)$ então para
todo $E\in\F$ temos que $1_E\cdot f\in M^{+}(\Omega,\F)$
e definimos a integral de $f$ com respeito a $\mu$ 
sobre $E$ sendo o número real estendido 
	\[
		\int_{E} f\, d\mu
		=
		\int_{\Omega} 1_{E}f\, d\mu.
	\]

Vamos mostrar a seguir que a integral é monótona 
com respeito a ambos, o integrando e o conjunto
sobre o qual a integral é realizada.






\begin{lema}
\label{lema-monotonicidade-integral-funcao-nao-negativas}
Seja $(\Omega,\F,\mu)$ um espaço de medida.
\begin{itemize}
	\item[a)]
	Se $f,g\in M^{+}(\Omega,\F)$	 e $f\leq g$, então 
		\[
			\int_{\Omega} f\, d\mu
			\leq
			\int_{\Omega} g\, d\mu.
		\]
	
	\item[b)] 
	Se $f\in M^{+}(\Omega,\F)$ para 
	qualquer par $E,F\in \F$, com 
	$E\subset F$ temos que 
		\[
			\int_{E} f\, d\mu
			\leq
			\int_{F} f\, d\mu.
		\]
	 
\end{itemize}
\end{lema}





\begin{proof}
Prova do item a). 
Já que $f\leq g$ para todo função simples $\varphi$ 
tal que $0\leq \varphi\leq f$ temos que $0\leq \varphi\leq g$.
Assim segue das propriedades do supremo e da definição de 
integral que o item a) é verdadeiro.

Prova do item b). Já que $1_{E}f\leq 1_{F}f$
a prova segue do item anterior.
\end{proof}

\bigskip

Já temos tudo preparado para apresentar um dos resultados
mais importantes desta seção que é um teorema de B. Levi 
conhecido hoje em dia como Teorema da Convergência Monótona.
Ele será fundamental na determinação de várias propriedades
da integral de Lebesgue bem como base para outros teoremas 
de convergência de integrais. 








\section{O Teorema da Convergência Monótona}




\begin{teorema}
[Teorema da Convergência Monótona]
\label{Teo-Convergencia-Monótona}
Seja $\{f_n\}$ uma sequência monótona
não decrescente em 
$M^{+}(\Omega,\F)$, isto é, 
para todo $n\in\N$ temos $0\leq f_n\leq f_{n+1}$.
Se $f_n$ converge para uma função$f$,
então temos que  $f\in M^{+}(\Omega,\F)$ e
além do mais
	\[
	\int_{\Omega} f\, d\mu 
	=
	\lim_{n\to\infty}\int_{\Omega} f_n\, d\mu.
	\]
\end{teorema}



\begin{proof}
Já que $f_n\in M^{+}(\Omega,\F)$ e $f=\lim f_n\geq 0$
segue do Corolário \ref{cor-lim-mensuravel-eh-mensuravel} 
que $f\in M^{+}(\Omega,\F)$.
Da monotonicidade da sequência $\{f_n\}$ e da 
integral de Lebesgue, temos que 
$f_n\leq f_{n+1}\leq f$ e 
	\[
		\int_{\Omega} f_n\, d\mu 
		\leq
		\int_{\Omega} f_{n+1}\, d\mu
		\leq 
		\int_{\Omega} f\, d\mu
		\qquad
		\forall n\in\N.
	\] 
Portanto existe o limite em $\overline{\R}$ da
sequência de números reais estendidos $\int_{\Omega}f_n\, d\mu$ 
e ele satisfaz a seguinte desigualdade
	\begin{equation}\label{des-aux1-TCM}
		\lim_{n\to\infty} \int_{\Omega} f_n\, d\mu 
		\leq
		\int_{\Omega} f\, d\mu. 
	\end{equation}


Para estabelecer a desigualdade oposta, seja 
$\alpha\in (0,1)$ e $\varphi$ uma função mensurável
simples satisfazendo $0\leq \varphi\leq f$.
Defina a seguinte sequência de conjuntos 
\[
	A_n = \{\omega\in\Omega: 
			\alpha \varphi(\omega)\leq f_n(\omega)
		\}
\]
Claramente $A_n\in \F$ para todo $n\in\N$. Também 
podemos observar que $A_n\subset A_{n+1}$.
De fato, se $\omega\in A_n$ então 
temos que 
$\alpha \varphi(\omega)\leq f_n(\omega)\leq f_{n+1}(\omega)$
o que mostrar que $\omega\in A_{n+1}$.
Afirmamos que $\cup_{n\in\N} A_n = \Omega$.
Para verificar que esta afirmação é verdadeira 
fixado $\omega\in\Omega$ seja 
$\varepsilon>0$ tal que 
$0<\varepsilon<f(\omega)-\alpha\varphi(\omega)$. 
Para tal $\varepsilon>0$
segue da desigualdade anterior e da convergência de $f_n$
para $f$ que existe $N_0\in\N$
(que pode depender de $\omega$) tal que   
se $n\geq N_0$ temos 
$\alpha\varphi(\omega)< f(\omega)-\varepsilon < f_n(\omega)$, 
logo $\omega\in A_n$. Pelo lema anterior temos que 
	\begin{equation}\label{eq-aux1-TCM}
	\int_{A_n} \alpha\varphi\, d\mu
	\leq 
	\int_{A_n} f_n\, d\mu
	\leq
	\int_{\Omega} f_n\, d\mu.
	\end{equation}
Já que $A_n\uparrow \Omega$ segue dos item c) do 
Lema \ref{lema-lin-int-funcao-simples-positiva} e 
da continuidade da medida que 
	\[
		\int_{\Omega} \varphi\, d\mu
		=
		\lim_{n\to\infty} \int_{A_n} \varphi\, d\mu.
	\]
Da desigualdade acima e de \eqref{eq-aux1-TCM} temos 
	\[
		\alpha\int_{\Omega} \varphi\, d\mu
		\leq
		\lim_{n\to\infty} \int_{\Omega} f_n\, d\mu.
	\]
Como a desigualdade acima é válida para todo $\alpha\in(0,1)$
podemos afirmar que 
	\[
		\int_{\Omega} \varphi\, d\mu
		\leq
		\lim_{n\to\infty} \int_{\Omega} f_n\, d\mu.
	\]
Observando que $\varphi$ é uma função simples 
arbitrária em $M^{+}(\Omega,\F)$ satisfazendo 
$0\leq \varphi\leq f$, podemos concluir que
\[
		\int_{\Omega} f\, d\mu
		=
		\sup_{ \substack{ 	\varphi \in M^{+}(\Omega,\F)
							\\
							\varphi\ \text{simples}
							\\ 
							0\leq\varphi(\omega)\leq f(\omega)
							,\ \forall \omega\in\Omega }  } 
		\ 
		\int_{\Omega} \varphi \, d\mu
		\leq
		\lim_{n\to\infty} \int_{\Omega} f_n\, d\mu.
\]
Esta desigualdade junto com \eqref{des-aux1-TCM}
encerra a prova do teorema.
\end{proof}


\begin{observacao}
No Teorema da Convergência Monótona não foi 
assumido que ambos os lados da igualdade 
\[
		\int_{\Omega} f\, d\mu
		=
		\lim_{n\to\infty} \int_{\Omega} f_n\, d\mu.
\]
sejam finitos. O que usamos na verdade é que a 
sequência de número reais estendidos 
$\{\int_{\Omega} f_n\, d\mu\}$ é uma sequência 
monótona não-decrescente de números reais estendidos
e que a mesma sempre tem limite em $\overline{\R}$ 
que pode eventualmente não pertencer a $\R$.
\end{observacao}





\begin{corolario}
\label{cor-lin-int-funcao-simples-positiva}
Sejam $(\Omega,\F,\mu)$ um espaço de medida, 
$f,g \in M^+(\Omega,\F)$ e $c\in [0,+\infty)$,
então
\begin{itemize}
	\item[a)]
	\(\displaystyle 
		\int_{\Omega} c f\, d\mu
		=
		c\int_{\Omega} f \, d\mu.
	\)
	

	\item[b)]
	\(\displaystyle 
		\int_{\Omega} (f+g)\, d\mu
		=
		\int_{\Omega} f \, d\mu
		+
		\int_{\Omega} g \, d\mu.
	\)

\end{itemize}
\end{corolario}


\begin{proof}
Prova do item a). 
Se $c=0$ então o resultado é imediato.
Se $c>0$, seja $\{\varphi_n\}$ uma sequência 
monótona não-decrescente de funções simples em 
$M^{+}(\Omega,\F)$ que converge para $f$.
A existência de tal sequência é garantida 
pelo Teorema \ref{teo:aproximacao-monotona-por-func-simples}.
Já que $c\varphi_n \uparrow cf$ podemos aplicar o item a) do
Lema \ref{lema-lin-int-funcao-simples-positiva} e
o Teorema da Convergência Monótona para verificar que 
são válidas as seguintes igualdades
	\[
	\int_{\Omega} cf\, d\mu
	=
	\lim_{n\to\infty} \int_{\Omega} c\varphi_n\, d\mu
	=
	c\lim_{n\to\infty} \int_{\Omega} \varphi_n\, d\mu	
	=
	c\int_{\Omega} f d\mu		
	\] 



Prova do item b). Sejam 
$\{\varphi_n\}$ e $\{psi_n\}$ sequências
monótonas não-decrescentes de funções simples em 
$M^{+}(\Omega,\F)$ que convergem para $f$ e $g$,
respectivamente. Claramente temos que 
$(\varphi_n+\psi_n) \uparrow (f+g)$.
Segue do item b) do 
Lema \ref{lema-lin-int-funcao-simples-positiva}
e do Teorema da Convergência Monótona que 
	\[
	\int_{\Omega} f+g\, d\mu
	=
	\lim_{n\to\infty} \int_{\Omega} \varphi_n+\psi_n\, d\mu
	=
	\lim_{n\to\infty} \int_{\Omega} \varphi_n\, d\mu
	+		
	\lim_{n\to\infty} \int_{\Omega} \varphi_n\, d\mu	
	=
	\int_{\Omega} f\, d\mu+\int_{\Omega} g\, d\mu
	\] 




\end{proof}