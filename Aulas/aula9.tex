\chapter[Aula 9]{A Integral de Lebesgue}
\chaptermark{}

\section{A Integral}

A principal diferença entre a integral de Riemann e a integral
de Lebesgue é que na integral de Riemann é feita uma 
partição do dominío da função, enquanto que na construção da
 integral de Lebesgue é feita uma partição da imagem da função.
Esse novo  "aproach" permite construir um conceito de integral muito
mais geral e com propriedades muito mais fortes  que a integral de Riemman, 
como veremos. Também discutiremos quando esses dois conceitos 
de integral coincidem.

O objeto que desempenha um papel fundamental na 
construção da integral de Lebesgue é a função simples
cuja definição é dada abaixo 
%
%
\begin{definicao}\label{def-funcao-simples}
Uma função $f:\Omega\to\R$ é dita uma função simples
se sua imagem $f(\Omega)$ possui apenas um número
finito de elementos.
\end{definicao}
%
%
%
%
\begin{observacao}
Lembramos que quando estamos trabalhando com 
espaços de probabilidade podemos chamar uma
função simples de v.a. discreta.
\end{observacao}


\begin{observacao}
Quando for dito $f\in M^{+}(\Omega,\F)$
e que tal função é uma função simples, 
estamos assumindo que $f$ toma valores apenas em $\R$.
\end{observacao}


Uma função real  mensurável simples 
$f\in M^{+}(\Omega,\mathcal{F})$ 
pode ser escrita na forma 
\begin{equation}\label{eq-funcao-simples-F-mensuravel}
f=\sum_{j=1}^{n}a_j 1_{E_j}
\end{equation}
onde $1_{E_j}$ é a função característica de um conjunto $E_j\in \mathcal{F}$. 
Não é difícil ver que existem várias maneiras de 
escrever uma mesma função simples na 
forma eqref{eq-funcao-simples-F-mensuravel}. 
Entretanto existe uma maneira de
representar tais funções que será nossa preferida 
e será chamada de
\emph{representação padrão}
\index{Representação Padrão de Funções Simples} de $f$, 
que é caracterizada do seguinte modo: 
os $a_j$'s devem ser distintos e os $E_j$'s devem ser disjuntos, 
não vazios e tais que $\bigcup_{j=1}^nE_j=\Omega$. 
%
%
\begin{exercicio}
Mostre que a representação padrão de uma função 
simples $f:\Omega\to\R$ é única.
\end{exercicio}
%
%
%
%
\begin{definicao}[Integral de Funções Simples não-negativas]
\label{def-integral-funcao-simples-positiva}
Sejam $(\Omega,\F,\mu)$ um espaço de medida e 
$\varphi:\Omega\to [0,+\infty)$ uma função simples cuja representação 
padrão é dada por $\varphi = \sum_{j=1}^n a_j1_{E_j}$.
Definimos a integral de $\varphi$ com respeito a $\mu$ 
sendo o número real estendido 
	\[
		\int_{\Omega} \varphi \, d\mu 
		=
		\sum_{j=1}^n a_j\, \mu(E_j).
	\]
\end{definicao}

Na definição acima deve ser lembrada nossa convenção 
que $0\cdot(+\infty)=0$. Portanto a integral da 
função identicamente nula é zero independentemente do 
espaço $\Omega$ ter medida finita ou não. 
Devemos observar que a integral de qualquer função 
simples não-negativa está bem definida em $\overline{\R}$.
Passamos agora à prova de alguma propriedades 
básicas desta integral.








\begin{lema}
Sejam $(\Omega,\F,\mu)$ um espaço de medida, 
$\varphi,\psi \in M^+(\Omega,\F)$ 
funções simples e $c\in [0,+\infty)$,
então
\begin{itemize}
	\item[a)]
	\(\displaystyle 
		\int_{\Omega} c\varphi\, d\mu
		=
		c\int_{\Omega} \varphi \, d\mu.
	\)
	

	\item[b)]
	\(\displaystyle 
		\int_{\Omega} (\varphi+\psi)\, d\mu
		=
		\int_{\Omega} \varphi \, d\mu
		+
		\int_{\Omega} \psi \, d\mu.
	\)
	
	\item[c)] A aplicação $\lambda:\F\to\overline{\R}$
	dada por 
		\(\displaystyle
			\lambda(E) = \int_{\Omega} \varphi 1_{E}\, d\mu
		\)
	é uma medida em $\F$.
\end{itemize}
\end{lema}



\begin{proof}
Prova do item a). Se $c=0$ então $c\varphi$ é a função 
identicamente nula e portanto a igualdade é válida. 
Se $c>0$ e $\varphi\in M^{+}(\Omega,\F)$ tem 
representação padrão 
	\[
		\varphi  
		=
		\sum_{j=1}^n a_j 1_{E_j} 
	\]
então $c\varphi\in M^{+}(\Omega,\F)$ e tem 
representação padrão dada por 
	\[
		c\varphi  
		=
		\sum_{j=1}^n ca_j 1_{E_j} 
	\]
Portanto temos que 
	\[
	\int_{\Omega} c\varphi\, d\mu
	=
	\sum_{j=1}^n ca_{j} \mu(E_j)
	=
	c \sum_{j=1}^n a_{j} \mu(E_j)
	=
	c \int_{\Omega} \varphi\, d\mu.
	\]



Prova do item b). Suponha que $\varphi$ e $\psi$ têm as 
seguintes representações padrões
	\[
		\varphi  
		=
		\sum_{j=1}^n a_j 1_{E_j} 
		\qquad
		\text{e}
		\qquad
		\psi 
		=
		\sum_{k=1}^m b_k 1_{F_k}.
	\]
Daí temos a seguinte representação para a soma 
destas funções 
	\[
		\varphi+\psi
		=
		\sum_{j=1}^n\sum_{k=1}^m (a_j+b_k) 1_{E_jF_k}.
	\]
Apesar da coleção $\{E_j\cap F_k\}$ ser disjunta 
quando $j=1,\ldots,n$ e $k=1,\ldots,m$ a representação
acima não é necessariamente a 
representação padrão de $\varphi+\psi$ 
pois não é garantido que $E_j\cap F_k\neq \emptyset$ 
e nem que a coleção $\{a_j+b_k\}$ seja uma coleção de
números reais distintos. Para obter a representação 
padrão procedemos da seguinte maneira. 
Seja $c_h$, com $1\leq h\leq p\leq nm$ a coleção 
de todos os número distintos da coleção 
$\{ a_j+b_k: j=1,\ldots,n;\ k=1,\ldots,m\}$. 
Para cada $h=1,\ldots, p$ defina 
	\[
		G_h
		=
		\bigcup_{ \substack {j,k : E_j\cap E_k\neq \emptyset \\a_j+b_k=c_h}  }
		E_j\cap E_k.
	\] 
Assim temos que 
	\begin{equation}\label{eq-aux1-linearidade-integral-func-simples-positiva}
		\mu(G_h)
		=
		\sum_{ \substack {j,k : E_j\cap E_k\neq \emptyset \\a_j+b_k=c_h}  }
		\mu(E_j\cap E_k).
	\end{equation} 
Já que a representação padrão de $\varphi+\psi$ é dada por 
	\[
		\varphi+\psi = \sum_{h=1}^p c_h 1_{G_h}.
	\]
Pela definição da integral e pela igualdade 
\eqref{eq-aux1-linearidade-integral-func-simples-positiva} 
temos que 
\begin{align*}
\int_{\Omega}(\varphi+\psi)\, d\mu 
&= 
\sum_{h=1}^p c_h \mu(G_h)
\\
&=
\sum_{h=1}^p 
	c_h 
	\sum_{ \substack {j,k : E_j\cap E_k\neq \emptyset \\a_j+b_k=c_h}  }
		\mu(E_j\cap E_k)
\\
&=
\sum_{h=1}^p  \ \ 
	\sum_{ \substack {j,k : E_j\cap E_k\neq \emptyset \\a_j+b_k=c_h}  }
		c_h\mu(E_j\cap E_k)
\\
&=
\sum_{h=1}^p \ \  
	\sum_{ \substack {j,k : E_j\cap E_k\neq \emptyset \\a_j+b_k=c_h}  }
		(a_j+b_k)\mu(E_j\cap E_k)
\\
&=
	\sum_{j=1}^n \   
	\sum_{k=1}^m
		(a_j+b_k)\mu(E_j\cap E_k)
\\
&=
	\sum_{j=1}^n \   
	\sum_{k=1}^m
		a_j\mu(E_j\cap E_k)
	+
	\sum_{j=1}^n \   
	\sum_{k=1}^m
		b_k\mu(E_j\cap E_k)
\\
&=
	\sum_{j=1}^n 
		a_j\mu(E_j)
	+
	\sum_{k=1}^m
		b_k\mu(E_k)
\\
&=
\int_{\Omega} \varphi\, d\mu 
+
\int_{\Omega} \psi \, d\mu
\end{align*}

Prova do item c). Usando os itens a), b) e a representação
\[
	\varphi 1_{E} 
	=
	\sum_{j=1}^n a_j 1_{E_j\cap E}
\]
temos que 
	\[
		\lambda(E)
		=
		\int_{\Omega}\varphi 1_{E}\, d\mu
		=
		\sum_{j=1}^n a_j \int_{\Omega}1_{E_j\cap E} \, d\mu
		=
		\sum_{j=1}^n a_j \mu(E_j\cap E).	
	\]
Já que aplicação $E\mapsto \mu(E_j\cap E)$ define uma medida 
em $\F$, $a_j$'s são não negativos e combinações lineares
de medidas com coeficientes não negativos são medidas então o 
lema está provado.
\end{proof}



Agora estamos preparados para introduzir a definição de
integral para qualquer função em $M^{+}(\Omega,\F)$. 
Como no caso de funções simples não-negativas 
está integral estará 
sempre bem definida como um número em $\overline{\R}$.



\begin{definicao}[Integral de Função Não-Negativa]
\label{def-integral-func-mens-nao-negativa}
A integral de Lebesgue de uma função $f\in M^{+}(\Omega,\F)$
com respeito a medida $\mu$ é dada por 
	\[
		\int_{\Omega} f\, d\mu
		=
		\sup_{ \substack{ 	\varphi \in M^{+}(\Omega,\F)
							\\
							\varphi\ \text{simples}
							\\ 
							0\leq\varphi\leq f }  } 
		\ 
		\int_{\Omega} \varphi \, d\mu
	\]
\end{definicao}