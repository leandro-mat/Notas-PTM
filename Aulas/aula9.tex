\chapter[Aula 9]{A Integral de Lebesgue}
\chaptermark{}

\section{A Integral}

A principal diferença entre a integral de Riemann e a integral
de Lebesgue é que na integral de Riemann é feita uma 
partição do dominío da função, enquanto que na construção da
 integral de Lebesgue é feita uma partição da imagem da função.
Esse novo  "aproach" permite construir um conceito de integral muito
mais geral e com propriedades muito mais fortes  que a integral de Riemman, 
como veremos. Também discutiremos quando esses dois conceitos 
de integral coincidem.

O objeto que desempenha um papel fundamental na 
construção da integral de Lebesgue é a função simples
cuja definição é dada abaixo 
%
%
\begin{definicao}\label{def-funcao-simples}
Uma função $f:\Omega\to\R$ é dita uma função simples
se sua imagem $f(\Omega)$ possui apenas um número
finito de elementos.
\end{definicao}
%
%
%
%
\begin{observacao}
Lembramos que quando estamos trabalhando com 
espaços de probabilidade podemos chamar uma
função simples de v.a. discreta.
\end{observacao}


Uma função real  mensurável simples 
$f\in M^{+}(\Omega,\mathcal{F})$ 
pode ser escrita na forma 
\begin{equation}\label{eq-funcao-simples-F-mensuravel}
f=\sum_{j=1}^{n}a_j 1_{E_j}
\end{equation}
onde $1_{E_j}$ é a função característica de um conjunto $E_j\in \mathcal{F}$. 
Não é difícil ver que existem várias maneiras de 
escrever uma mesma função simples na 
forma eqref{eq-funcao-simples-F-mensuravel}. 
Entretanto existe uma maneira de
representar tais funções que será nossa preferida 
e será chamada de
\emph{representação padrão}
\index{Representação Padrão de Funções Simples} de $f$, 
que é caracterizada do seguinte modo: 
os $a_j$'s devem ser distintos e os $E_j$'s devem ser disjuntos, 
não vazios e tais que $\bigcup_{j=1}^nE_j=\Omega$. 
\begin{exercicio}
Mostre que a representação padrão de uma função 
simples $f:\Omega\to\R$ é única.
\end{exercicio}
