\chapter[Aula 9]{A Integral de Lebesgue}
\chaptermark{}

\section{A Integral de Funções em $M^{+}(\Omega,\F)$}

A principal diferença entre a integral de Riemann e a integral
de Lebesgue é que na integral de Riemann é feita uma 
partição do dominío da função, enquanto que na construção da
 integral de Lebesgue é feita uma partição da imagem da função.
Esse novo  "aproach" permite construir um conceito de integral muito
mais geral e com propriedades muito mais fortes  que a integral de Riemman, 
como veremos. Também discutiremos quando esses dois conceitos 
de integral coincidem.

O objeto que desempenha um papel fundamental na 
construção da integral de Lebesgue é a função simples
cuja definição é dada abaixo 2
%
%
\begin{definicao}\label{def-funcao-simples}
Uma função $f:\Omega\to\R$ é dita uma função simples
se sua imagem $f(\Omega)$ possui apenas um número
finito de elementos.
\end{definicao}
%
%
%
%
\begin{observacao}
Lembramos que quando estamos trabalhando com 
espaços de probabilidade podemos chamar uma
função simples de v.a. discreta.
\end{observacao}


\begin{observacao}
Quando for dito $f\in M^{+}(\Omega,\F)$
e que tal função é uma função simples, 
estamos assumindo que $f$ toma valores apenas em $\R$.
\end{observacao}


Uma função real  mensurável simples 
$f\in M^{+}(\Omega,\mathcal{F})$ 
pode ser escrita na forma 
\begin{equation}\label{eq-funcao-simples-F-mensuravel}
f=\sum_{j=1}^{n}a_j 1_{E_j}
\end{equation}
onde $1_{E_j}$ é a função característica de um conjunto $E_j\in \mathcal{F}$. 
Não é difícil ver que existem várias maneiras de 
escrever uma mesma função simples na 
forma eqref{eq-funcao-simples-F-mensuravel}. 
Entretanto existe uma maneira de
representar tais funções que será nossa preferida 
e será chamada de
\emph{representação padrão}
\index{Representação Padrão de Funções Simples} de $f$, 
que é caracterizada do seguinte modo: 
os $a_j$'s devem ser distintos e os $E_j$'s devem ser disjuntos, 
não vazios e tais que $\bigcup_{j=1}^nE_j=\Omega$. 
%
%
\begin{exercicio}
Mostre que a representação padrão de uma função 
simples $f:\Omega\to\R$ é única.
\end{exercicio}
%
%
%
%
\begin{definicao}[Integral de Funções Simples não-negativas]
\label{def-integral-funcao-simples-positiva}
Sejam $(\Omega,\F,\mu)$ um espaço de medida e 
$\varphi:\Omega\to [0,+\infty)$ uma função simples cuja representação 
padrão é dada por $\varphi = \sum_{j=1}^n a_j1_{E_j}$.
Definimos a integral de $\varphi$ com respeito a $\mu$ 
sendo o número real estendido 
	\[
		\int_{\Omega} \varphi \, d\mu 
		=
		\sum_{j=1}^n a_j\, \mu(E_j).
	\]
\end{definicao}

Na definição acima deve ser lembrada nossa convenção 
que $0\cdot(+\infty)=0$. Portanto a integral da 
função identicamente nula é zero independentemente do 
espaço $\Omega$ ter medida finita ou não. 
Devemos observar que a integral de qualquer função 
simples não-negativa está bem definida em $\overline{\R}$.
Passamos agora à prova de alguma propriedades 
básicas desta integral.








\begin{lema}\label{lema-lin-int-funcao-simples-positiva}
Sejam $(\Omega,\F,\mu)$ um espaço de medida, 
$\varphi,\psi \in M^+(\Omega,\F)$ 
funções simples e $c\in [0,+\infty)$,
então
\begin{itemize}
	\item[a)]
	\(\displaystyle 
		\int_{\Omega} c\varphi\, d\mu
		=
		c\int_{\Omega} \varphi \, d\mu.
	\)
	

	\item[b)]
	\(\displaystyle 
		\int_{\Omega} (\varphi+\psi)\, d\mu
		=
		\int_{\Omega} \varphi \, d\mu
		+
		\int_{\Omega} \psi \, d\mu.
	\)
	
	\item[c)] A aplicação $\lambda:\F\to\overline{\R}$
	dada por 
		\(\displaystyle
			\lambda(E) = \int_{\Omega} \varphi 1_{E}\, d\mu
		\)
	é uma medida em $\F$.
\end{itemize}
\end{lema}



\begin{proof}
Prova do item a). Se $c=0$ então $c\varphi$ é a função 
identicamente nula e portanto a igualdade é válida. 
Se $c>0$ e $\varphi\in M^{+}(\Omega,\F)$ tem 
representação padrão 
	\[
		\varphi  
		=
		\sum_{j=1}^n a_j 1_{E_j} 
	\]
então $c\varphi\in M^{+}(\Omega,\F)$ e tem 
representação padrão dada por 
	\[
		c\varphi  
		=
		\sum_{j=1}^n ca_j 1_{E_j} 
	\]
Portanto temos que 
	\[
	\int_{\Omega} c\varphi\, d\mu
	=
	\sum_{j=1}^n ca_{j} \mu(E_j)
	=
	c \sum_{j=1}^n a_{j} \mu(E_j)
	=
	c \int_{\Omega} \varphi\, d\mu.
	\]



Prova do item b). Suponha que $\varphi$ e $\psi$ têm as 
seguintes representações padrões
	\[
		\varphi  
		=
		\sum_{j=1}^n a_j 1_{E_j} 
		\qquad
		\text{e}
		\qquad
		\psi 
		=
		\sum_{k=1}^m b_k 1_{F_k}.
	\]
Daí temos a seguinte representação para a soma 
destas funções 
	\[
		\varphi+\psi
		=
		\sum_{j=1}^n\sum_{k=1}^m (a_j+b_k) 1_{E_jF_k}.
	\]
Apesar da coleção $\{E_j\cap F_k\}$ ser disjunta 
quando $j=1,\ldots,n$ e $k=1,\ldots,m$ a representação
acima não é necessariamente a 
representação padrão de $\varphi+\psi$ 
pois não é garantido que $E_j\cap F_k\neq \emptyset$ 
e nem que a coleção $\{a_j+b_k\}$ seja uma coleção de
números reais distintos. Para obter a representação 
padrão procedemos da seguinte maneira. 
Seja $c_h$, com $1\leq h\leq p\leq nm$ a coleção 
de todos os número distintos da coleção 
$\{ a_j+b_k: j=1,\ldots,n;\ k=1,\ldots,m\}$. 
Para cada $h=1,\ldots, p$ defina 
	\[
		G_h
		=
		\bigcup_{ \substack {j,k : E_j\cap E_k\neq \emptyset \\a_j+b_k=c_h}  }
		E_j\cap E_k.
	\] 
Assim temos que 
	\begin{equation}\label{eq-aux1-linearidade-integral-func-simples-positiva}
		\mu(G_h)
		=
		\sum_{ \substack {j,k : E_j\cap E_k\neq \emptyset \\a_j+b_k=c_h}  }
		\mu(E_j\cap E_k).
	\end{equation} 
Já que a representação padrão de $\varphi+\psi$ é dada por 
	\[
		\varphi+\psi = \sum_{h=1}^p c_h 1_{G_h}.
	\]
Pela definição da integral e pela igualdade 
\eqref{eq-aux1-linearidade-integral-func-simples-positiva} 
temos que 
\begin{align*}
\int_{\Omega}(\varphi+\psi)\, d\mu 
&= 
\sum_{h=1}^p c_h \mu(G_h)
\\
&=
\sum_{h=1}^p 
	c_h 
	\sum_{ \substack {j,k : E_j\cap E_k\neq \emptyset \\a_j+b_k=c_h}  }
		\mu(E_j\cap E_k)
\\
&=
\sum_{h=1}^p  \ \ 
	\sum_{ \substack {j,k : E_j\cap E_k\neq \emptyset \\a_j+b_k=c_h}  }
		c_h\mu(E_j\cap E_k)
\\
&=
\sum_{h=1}^p \ \  
	\sum_{ \substack {j,k : E_j\cap E_k\neq \emptyset \\a_j+b_k=c_h}  }
		(a_j+b_k)\mu(E_j\cap E_k)
\\
&=
	\sum_{j=1}^n \   
	\sum_{k=1}^m
		(a_j+b_k)\mu(E_j\cap E_k)
\\
&=
	\sum_{j=1}^n \   
	\sum_{k=1}^m
		a_j\mu(E_j\cap E_k)
	+
	\sum_{j=1}^n \   
	\sum_{k=1}^m
		b_k\mu(E_j\cap E_k)
\\
&=
	\sum_{j=1}^n 
		a_j\mu(E_j)
	+
	\sum_{k=1}^m
		b_k\mu(E_k)
\\
&=
\int_{\Omega} \varphi\, d\mu 
+
\int_{\Omega} \psi \, d\mu
\end{align*}

Prova do item c). Usando os itens a), b) e a representação
\[
	\varphi 1_{E} 
	=
	\sum_{j=1}^n a_j 1_{E_j\cap E}
\]
temos que 
	\[
		\lambda(E)
		=
		\int_{\Omega}\varphi 1_{E}\, d\mu
		=
		\sum_{j=1}^n a_j \int_{\Omega}1_{E_j\cap E} \, d\mu
		=
		\sum_{j=1}^n a_j \mu(E_j\cap E).	
	\]
Já que aplicação $E\mapsto \mu(E_j\cap E)$ define uma medida 
em $\F$, $a_j$'s são não negativos e combinações lineares
de medidas com coeficientes não negativos são medidas então o 
lema está provado.
\end{proof}



Agora estamos preparados para introduzir a definição de
integral para qualquer função em $M^{+}(\Omega,\F)$. 
Como no caso de funções simples não-negativas 
está integral estará 
sempre bem definida como um número em $\overline{\R}$.



\begin{definicao}[Integral de Função Não-Negativa]
\label{def-integral-func-mens-nao-negativa}
A integral de Lebesgue de uma função $f\in M^{+}(\Omega,\F)$
com respeito a medida $\mu$ é definida pela seguinte número real estendido 
	\[
		\int_{\Omega} f\, d\mu
		=
		\sup_{ \substack{ 	\varphi \in M^{+}(\Omega,\F)
							\\
							\varphi\ \text{simples}
							\\ 
							0\leq\varphi(\omega)\leq f(\omega)
							,\ \forall \omega\in\Omega }  } 
		\ 
		\int_{\Omega} \varphi \, d\mu
	\]
\end{definicao}






Observe que se $f\in M^{+}(\Omega,F)$ então para
todo $E\in\F$ temos que $1_E\cdot f\in M^{+}(\Omega,\F)$
e definimos a integral de $f$ com respeito a $\mu$ 
sobre $E$ sendo o número real estendido 
	\[
		\int_{E} f\, d\mu
		=
		\int_{\Omega} 1_{E}f\, d\mu.
	\]

Vamos mostrar a seguir que a integral é monótona 
com respeito a ambos, o integrando e o conjunto
sobre o qual a integral é realizada.






\begin{lema}
\label{lema-monotonicidade-integral-funcao-nao-negativas}
Seja $(\Omega,\F,\mu)$ um espaço de medida.
\begin{itemize}
	\item[a)]
	Se $f,g\in M^{+}(\Omega,\F)$	 e $f\leq g$, então 
		\[
			\int_{\Omega} f\, d\mu
			\leq
			\int_{\Omega} g\, d\mu.
		\]
	
	\item[b)] 
	Se $f\in M^{+}(\Omega,\F)$ para 
	qualquer par $E,F\in \F$, com 
	$E\subset F$ temos que 
		\[
			\int_{E} f\, d\mu
			\leq
			\int_{F} f\, d\mu.
		\]
	 
\end{itemize}
\end{lema}





\begin{proof}
Prova do item a). 
Já que $f\leq g$ para todo função simples $\varphi$ 
tal que $0\leq \varphi\leq f$ temos que $0\leq \varphi\leq g$.
Assim segue das propriedades do supremo e da definição de 
integral que o item a) é verdadeiro.

Prova do item b). Já que $1_{E}f\leq 1_{F}f$
a prova segue do item anterior.
\end{proof}

\bigskip

Já temos tudo preparado para apresentar um dos resultados
mais importantes desta seção que é um teorema de B. Levi 
conhecido hoje em dia como Teorema da Convergência Monótona.
Ele será fundamental na determinação de várias propriedades
da integral de Lebesgue bem como base para outros teoremas 
de convergência de integrais. 








\section{O Teorema da Convergência Monótona e Lema de Fatou}


\begin{teorema}
[Teorema da Convergência Monótona]
\label{Teo-Convergencia-Monótona}
\index{Teorema!da Convergência Monótona}
Seja $\{f_n\}$ uma sequência monótona
não decrescente em 
$M^{+}(\Omega,\F)$, isto é, 
para todo $n\in\N$ temos $0\leq f_n\leq f_{n+1}$.
Se $f_n$ converge para uma função$f$,
então temos que  $f\in M^{+}(\Omega,\F)$ e
além do mais
	\[
	\int_{\Omega} f\, d\mu 
	=
	\lim_{n\to\infty}\int_{\Omega} f_n\, d\mu.
	\]
\end{teorema}



\begin{proof}
Já que $f_n\in M^{+}(\Omega,\F)$ e $f=\lim f_n\geq 0$
segue do Corolário \ref{cor-lim-mensuravel-eh-mensuravel} 
que $f\in M^{+}(\Omega,\F)$.
Da monotonicidade da sequência $\{f_n\}$ e da 
integral de Lebesgue, temos que 
$f_n\leq f_{n+1}\leq f$ e 
	\[
		\int_{\Omega} f_n\, d\mu 
		\leq
		\int_{\Omega} f_{n+1}\, d\mu
		\leq 
		\int_{\Omega} f\, d\mu
		\qquad
		\forall n\in\N.
	\] 
Portanto existe o limite em $\overline{\R}$ da
sequência de números reais estendidos $\int_{\Omega}f_n\, d\mu$ 
e ele satisfaz a seguinte desigualdade
	\begin{equation}\label{des-aux1-TCM}
		\lim_{n\to\infty} \int_{\Omega} f_n\, d\mu 
		\leq
		\int_{\Omega} f\, d\mu. 
	\end{equation}


Para estabelecer a desigualdade oposta, seja 
$\alpha\in (0,1)$ e $\varphi$ uma função mensurável
simples satisfazendo $0\leq \varphi\leq f$.
Defina a seguinte sequência de conjuntos 
\[
	A_n = \{\omega\in\Omega: 
			\alpha \varphi(\omega)\leq f_n(\omega)
		\}
\]
Claramente $A_n\in \F$ para todo $n\in\N$. Também 
podemos observar que $A_n\subset A_{n+1}$.
De fato, se $\omega\in A_n$ então 
temos que 
$\alpha \varphi(\omega)\leq f_n(\omega)\leq f_{n+1}(\omega)$
o que mostrar que $\omega\in A_{n+1}$.
Afirmamos que $\cup_{n\in\N} A_n = \Omega$.
Para verificar que esta afirmação é verdadeira 
fixado $\omega\in\Omega$ seja 
$\varepsilon>0$ tal que 
$0<\varepsilon<f(\omega)-\alpha\varphi(\omega)$. 
Para tal $\varepsilon>0$
segue da desigualdade anterior e da convergência de $f_n$
para $f$ que existe $N_0\in\N$
(que pode depender de $\omega$) tal que   
se $n\geq N_0$ temos 
$\alpha\varphi(\omega)< f(\omega)-\varepsilon < f_n(\omega)$, 
logo $\omega\in A_n$. Pelo lema anterior temos que 
	\begin{equation}\label{eq-aux1-TCM}
	\int_{A_n} \alpha\varphi\, d\mu
	\leq 
	\int_{A_n} f_n\, d\mu
	\leq
	\int_{\Omega} f_n\, d\mu.
	\end{equation}
Já que $A_n\uparrow \Omega$ segue dos item c) do 
Lema \ref{lema-lin-int-funcao-simples-positiva} e 
da continuidade da medida que 
	\[
		\int_{\Omega} \varphi\, d\mu
		=
		\lim_{n\to\infty} \int_{A_n} \varphi\, d\mu.
	\]
Da desigualdade acima e de \eqref{eq-aux1-TCM} temos 
	\[
		\alpha\int_{\Omega} \varphi\, d\mu
		\leq
		\lim_{n\to\infty} \int_{\Omega} f_n\, d\mu.
	\]
Como a desigualdade acima é válida para todo $\alpha\in(0,1)$
podemos afirmar que 
	\[
		\int_{\Omega} \varphi\, d\mu
		\leq
		\lim_{n\to\infty} \int_{\Omega} f_n\, d\mu.
	\]
Observando que $\varphi$ é uma função simples 
arbitrária em $M^{+}(\Omega,\F)$ satisfazendo 
$0\leq \varphi\leq f$, podemos concluir que
\[
		\int_{\Omega} f\, d\mu
		=
		\sup_{ \substack{ 	\varphi \in M^{+}(\Omega,\F)
							\\
							\varphi\ \text{simples}
							\\ 
							0\leq\varphi(\omega)\leq f(\omega)
							,\ \forall \omega\in\Omega }  } 
		\ 
		\int_{\Omega} \varphi \, d\mu
		\leq
		\lim_{n\to\infty} \int_{\Omega} f_n\, d\mu.
\]
Esta desigualdade junto com \eqref{des-aux1-TCM}
encerra a prova do teorema.
\end{proof}


\begin{observacao}
No Teorema da Convergência Monótona não foi 
assumido que ambos os lados da igualdade 
\[
		\int_{\Omega} f\, d\mu
		=
		\lim_{n\to\infty} \int_{\Omega} f_n\, d\mu.
\]
sejam finitos. O que usamos na verdade é que a 
sequência de número reais estendidos 
$\{\int_{\Omega} f_n\, d\mu\}$ é uma sequência 
monótona não-decrescente de números reais estendidos
e que a mesma sempre tem limite em $\overline{\R}$ 
que pode eventualmente não pertencer a $\R$.
\end{observacao}





\begin{corolario}
\label{cor-linearidade-int-funcao-mens-positiva}
Sejam $(\Omega,\F,\mu)$ um espaço de medida, 
$f,g \in M^+(\Omega,\F)$ e $c\in [0,+\infty)$,
então
\begin{itemize}
	\item[a)]
	\(\displaystyle 
		\int_{\Omega} c f\, d\mu
		=
		c\int_{\Omega} f \, d\mu.
	\)
	

	\item[b)]
	\(\displaystyle 
		\int_{\Omega} (f+g)\, d\mu
		=
		\int_{\Omega} f \, d\mu
		+
		\int_{\Omega} g \, d\mu.
	\)

\end{itemize}
\end{corolario}


\begin{proof}
Prova do item a). 
Se $c=0$ então o resultado é imediato.
Se $c>0$, seja $\{\varphi_n\}$ uma sequência 
monótona não-decrescente de funções simples em 
$M^{+}(\Omega,\F)$ que converge para $f$.
A existência de tal sequência é garantida 
pelo Teorema \ref{teo:aproximacao-monotona-por-func-simples}.
Já que $c\varphi_n \uparrow cf$ podemos aplicar o item a) do
Lema \ref{lema-lin-int-funcao-simples-positiva} e
o Teorema da Convergência Monótona para verificar que 
são válidas as seguintes igualdades
	\[
	\int_{\Omega} cf\, d\mu
	=
	\lim_{n\to\infty} \int_{\Omega} c\varphi_n\, d\mu
	=
	c\lim_{n\to\infty} \int_{\Omega} \varphi_n\, d\mu	
	=
	c\int_{\Omega} f d\mu		
	\] 



Prova do item b). Sejam 
$\{\varphi_n\}$ e $\{psi_n\}$ sequências
monótonas não-decrescentes de funções simples em 
$M^{+}(\Omega,\F)$ que convergem para $f$ e $g$,
respectivamente. Claramente temos que 
$(\varphi_n+\psi_n) \uparrow (f+g)$.
Segue do item b) do 
Lema \ref{lema-lin-int-funcao-simples-positiva}
e do Teorema da Convergência Monótona que 
	\[
	\int_{\Omega} f+g\, d\mu
	=
	\lim_{n\to\infty} \int_{\Omega} \varphi_n+\psi_n\, d\mu
	=
	\lim_{n\to\infty} \int_{\Omega} \varphi_n\, d\mu
	+		
	\lim_{n\to\infty} \int_{\Omega} \varphi_n\, d\mu	
	=
	\int_{\Omega} f\, d\mu+\int_{\Omega} g\, d\mu.
	\] 
\end{proof}




O próximo resultado é uma consequência do Teorema 
da Convergência Monótona e é um resultado muito 
importante para lidarmos com integrais de sequências
de funções que não não monótonas.







\begin{lema}[Lema de Fatou]\index{Lema!de Fatou}
Sejam $(\Omega,\F,\mu)$ um espaço de medida e 
$\{f_n\}$ uma sequência arbitrária de 
funções em $M^+(\Omega,\F)$. Então 
	\[
		\int_{\Omega}\left( \liminf_{n\to\infty} f_n\right) \, d\mu
		\leq
		\liminf_{n\to\infty} \int_{\Omega}f_n\, d\mu
	\]
\end{lema}


\begin{proof}
Para cada $n\in\N$ seja $g_n =\inf_{k\geq n} f_k$.
Pela definição de ínfimo temos para
qualquer $m\in \N$ fixado que $g_m\leq f_n$
para todo $n\geq m$. Portanto é válida a 
seguinte desigualdade
	\[
	\int_{\Omega} g_m\, d\mu
	\leq
	\int_{\Omega} f_n\, d\mu,
	\] 
desde que $n\geq m$. Daí temos que
	\begin{equation}\label{des-aux1-lema-fatou}
	\int_{\Omega} g_m\, d\mu
	\leq
	\liminf_{n\to\infty}\int_{\Omega} f_n\, d\mu.
	\end{equation} 
Já que $g_{m}\leq g_{m+1}$ para todo $m\in\N$ 
e que 
\[	
	\lim_{m\to\infty} g_m(\omega) 
	\uparrow 
	\liminf_{n\to\infty} f_n(\omega)
	\qquad
	\forall \omega\in\Omega,
\]
segue do Teorema da Convergência Monótona e 
da Desigualdade \eqref{des-aux1-lema-fatou} que 
\[
	\int_{\Omega} \left( \liminf_{n\to\infty} f_n\right) \, d\mu
	=
	\lim_{m\to\infty}\int_{\Omega} g_m\, d\mu
	\leq
	\liminf_{n\to\infty}\int_{\Omega} f_n\, d\mu.
\]
\end{proof}






\begin{observacao}
Será visto na Lista 3 que a conclusão do Lema 
de Fatou pode não ser verdadeira se não assumimos que 
$f_n\geq 0$.
\end{observacao}





\begin{corolario}\label{cor-lambda(E)-eh-uma-medida}
Sejam $(\Omega,\F,\mu)$ um espaço de medida e 
$f\in M^{+}(\Omega,\F)$. Então a aplicação 
$\lambda:\F\to[0,+\infty]$ dada por
	\[
		\lambda(E) 
		=
		\int_{E} f\, d\mu
	\]
é uma medida.
\end{corolario}


\begin{proof}
Já que $f\geq 0$ temos para todo $E\in\F$ que
$\lambda(E)\geq 0$. Para verificar que 
$\lambda(\emptyset)=0$, 
basta observar que $1_{\emptyset}\cdot f\equiv 0$.
Para mostrar que $\lambda$ é $\sigma$-aditiva, 
seja $\{E_n\}$ uma sequência em $\F$ 
mutuamente disjunta e defina 
	\[
		f_n = \sum_{j=1}^n 1_{E_j}\cdot f.
	\]
Pelo Corolário \ref{cor-lin-int-funcao-simples-positiva}
item b) temos que 
	\[
		\int_{\Omega} f_n\, d\mu 
		 =
		 \sum_{j=1}^n \int_{\Omega} 1_{E_j} f\, d\mu
		 =
 		 \sum_{j=1}^n \lambda(E_j).
	\]
Observe que $f_n \uparrow 1_{E}f$, onde $E=\cup_{n\in\N }E_n$.
Pelo Teorema da Convergência Monótona temos que 
	\[
		\lambda(E)
		=
		\int_{\Omega} 1_{E}f\, d\mu 
		=
		\lim_{n\to\infty}\int_{\Omega} 1_{E}f_n\, d\mu 
		=
		\lim_{n\to\infty} 
			\sum_{j=1}^n \int_{\Omega} 1_{E\cap E_j} f\, d\mu
		=
 		\sum_{j=1}^{\infty} \lambda(E_j),
	\]
onde na última igualdade usamos que a sequência 
$\{E_j\}$ é mutuamente disjunta para garantir que $E\cap E_j=E_j$.
\end{proof}






\begin{corolario}
\label{cor-int-f-nneg-eh-zero-equiv-fnula-qtp}
Sejam $(\Omega,\F,\mu)$ um espaço de medida e 
$f\in M^{+}(\Omega,\F)$. Então $f=0$ $\mu$-quase 
certamente se, e somente se, 
	\begin{equation}
	\label{eq-aux1-f-nneg-eh-zero-equiv-fnula-qtp}
	\int_{\Omega}f\, d\mu = 0.
	\end{equation}
\end{corolario}


\begin{proof}
Suponha que  vale a igualdade 
\eqref{eq-aux1-f-nneg-eh-zero-equiv-fnula-qtp}
e seja 
	\[
	E_n 
	=
	\left\{
		\omega\in\Omega: \frac{1}{n}< f(\omega).
	\right\}
	\]
Da definição de $E_n$ segue que $(1/n)1_{E_{n}}\leq f$, 
para todo $n\in\N$ e assim 
	\[
	0\leq \frac{1}{n}\mu(E_n)\leq \int_{\Omega} f\, d\mu =0.
	\]
O que implica que $\mu(E_n)=0,\ \forall n\in\N$.
Já que o conjunto 
	\[
	\left\{
		\omega\in\Omega: 0< f(\omega)
	\right\}
	=
	\bigcup_{n=1}^{\infty} E_n
	\]
tem medida $\mu$ zero, 
segue que $f=0$ $\mu$-quase certamente.


Reciprocamente, suponha que 
$f(\omega)=0$ $\mu$-quase certamente, 
então o conjunto $E=\{\omega\in\Omega: f(\omega)>0\}$
é tal que  $\mu(E)=0$. 
Seja $f_n = n\cdot 1_{E}$. 
Já que $f(\omega)\leq \liminf_{n\to\infty} f_n$ 
segue do Lema de Fatou que 
	\[
	0
	\leq
	\int_{\Omega}f\, d\mu
	\leq 
	\int_{\Omega} \left( \liminf_{n\to\infty} f_n\right)\, d\mu 
	\leq 
	\liminf_{n\to\infty}	\int_{\Omega}f_n\, d\mu 
	=
	0.
	\]
\end{proof}








\begin{corolario}
Sejam $(\Omega,\F,\mu)$ um espaço de medida e 
$f\in M^{+}(\Omega,\F)$. Se  
$\lambda(E)=\int_{E}f d\mu$ então a medida 
$\lambda$ é absolutamente contínua com respeito 
a $\mu$, ou seja, para todo $E\in\F$ tal que 
$\mu(E)=0$ temos que $\lambda(E)=0$. 
\end{corolario}


\begin{proof}
Se $E\in\F$ é tal que $\mu(E)=0$, 
então $f\cdot 1_{E} =0$ $\mu$-quase certamente.
Pelo Corolário \ref{cor-int-f-nneg-eh-zero-equiv-fnula-qtp}
temos que 
	\[
		\lambda(E)
		=
		\int_{\Omega}f\cdot 1_{E}\, d\mu
		=
		0.
	\]
\end{proof}






No que segue mostramos que o Teorema 
da Convergência Monótona contínua 
verdadeiro se a convergência da sequência 
$f_n$ para $f$ se verifica em quase certamente.
O enunciado preciso deste resultado é enunciado 
abaixo.





\begin{corolario}
\label{Teo-TCM-quase-certamente}
\index{Teorema!da Convergência Monótona}
Seja $\{f_n\}$ uma sequência monótona
não decrescente em 
$M^{+}(\Omega,\F)$, isto é, 
para todo $n\in\N$ temos $0\leq f_n\leq f_{n+1}$.
Se $f_n \to f$ $\mu$-quase certamente 
e $f\in M^{+}(\Omega,\F)$ então
	\[
	\int_{\Omega} f\, d\mu 
	=
	\lim_{n\to\infty}\int_{\Omega} f_n\, d\mu.
	\]
\end{corolario}
 
 
\begin{proof}
Seja $N\in \F$ tal que $\mu(N)=0$ e 
$f_n(\omega)\to f(\omega)$ para todo 
$\omega\in M\equiv \Omega\setminus N$.
Pela definição de $M$ segue que a sequência
$\{f_n\cdot 1_{M}\}$ converge (em todo $\Omega$)
para $f\cdot 1_{M}$, além do mais 
$f_n\cdot 1_{M}\uparrow f\cdot 1_{M}$.
Aplicando o Teorema da Convergência Monótona 
temos 
	\[
		\int_{\Omega} f\cdot 1_{M}\, d\mu 
		=
		\lim_{n\to\infty} \int_{\Omega} f_n\cdot 1_{M}\, d\mu.
	\]
Por outro lado, como $\mu(N)=0$ temos que 
as funções $f\cdot 1_{N}$ e $f_n\cdot 1_{N}$
($\forall n\in\N$) 
são nulas $\mu$-quase certamente e 
pelo Corolário \ref{cor-int-f-nneg-eh-zero-equiv-fnula-qtp}
temos que 
	\[
		\int_{\Omega} f\cdot 1_{N}\, d\mu 
		=
		0
		=
		\int_{\Omega} f_n\cdot 1_{N}\, d\mu,
		\qquad\forall n\in\N.	
	\]
Notando que $f=f\cdot 1_{M}+f\cdot 1_{N}$ e 
que $f_n=f_n\cdot 1_{M}+f_n\cdot 1_{N}$, 
para todo $n\in\N$ temos que 
	\[
	\int_{\Omega}f\, d\mu 
	=
	\int_{\Omega} f\cdot 1_{M}\, d\mu 
	=
	\lim_{n\to\infty} \int_{\Omega} f_n\cdot 1_{M}\, d\mu.	
	=
	\lim_{n\to\infty} \int_{\Omega} f_n\, d\mu.	
	\]
\end{proof}



\begin{corolario}
Sejam $(\Omega,\F,\mu)$ um espaço de medida e 
$\{g_n\}$ uma sequência de funções em 
$\in M^{+}(\Omega,\F)$. Então 
	\[
		\int_{\Omega} \left(\sum_{j=1}^{\infty} g_n\right) d\mu
		=
		\sum_{j=1}^{\infty}\int_{\Omega}  g_n\, d\mu.
	\]  
\end{corolario}


\begin{proof}
Defina a sequência $f_n=g_1+\ldots+g_n$.
Claramente $f_n\in M^{+}(\Omega,\F)$,
$f_n\leq f_{n+1}$ e $f_n\to \sum_{n\in \N} g_n$. 
Aplicando o Teorema da Convergência Monótona obtemos o resultado
desejado.
\end{proof}




\section{Integrais de Funções em $M(\Omega,\F)$  }

Na seção anterior definimos a integral de uma 
função mensurável não-negativa em $M^{+}(\Omega,\F)$ 
com respeito a uma medida $\mu$ definida sobre $\F$ 
e permitimos que este valor fosse $+\infty$.
Nesta seção vamos discutir o conceito de integral
de funções que podem tomar ambos valores reais
positivos e negativos. Aqui é mais conveniente 
exigir que os valores tanto das funções quanto 
da integrais sejam números reais.


\begin{definicao}[Integral de Lebesgue de Função Real]
Seja $(\Omega,\F,\mu)$ um espaço de medida. 
Denotamos por $L(\Omega,\F,\mu)$ a coleção 
de todas as funções $f:\Omega\to\R$ que 
são $\F$-mensuráveis e tais que ambas
partes positiva e negativa de $f$ são integráveis, 
ou seja,
\[
	\int_{\Omega} f^+\, d\mu<\infty
	\quad\text{e}\quad
	\int_{\Omega} f^-\, d\mu<\infty.
\]
Neste caso definimos a integral $f$ 
com respeito a $\mu$ por
	\[
		\int_{\Omega} f\, d\mu
		=
		\int_{\Omega} f^{+}\, d\mu
		-
		\int_{\Omega} f^{-}\, d\mu
	\]
e dizemos que {\bf $f$ é integrável}.
Analogamente à seção anterior, se $E\in \F$ então definimos 
	\[
		\int_{E} f\, d\mu 
		=
		\int_{E} f^{+}\, d\mu - \int_{E} f^{-}\, d\mu.
	\]
\end{definicao}

\begin{observacao}\label{obs-int-diferenca-2func-positivas}
Embora a integral de uma função $f\in L(\Omega,\F,\mu)$ 
seja definida como sendo a diferença 
entre as integrais de $f^{+}$
e $f^{-}$ é fácil ver que se $f=f_1-f_2$,
onde $f_1$ e $f_2$ são funções não-negativas 
com integrais finitas, então 
	\[
		\int_{\Omega} f\, d\mu 
		=
		\int_{\Omega} f_1\, d\mu 
		-
		\int_{\Omega} f_2\, d\mu.
	\]
De fato, já que $f^{+}-f^{-}=f=f_1-f_2$ então 
$f^{+}+f_2=f_1+f^{-}$. Desta igualdade 
e do item b) do 
Corolário \ref{cor-linearidade-int-funcao-mens-positiva}
segue que
\[	
	\int_{\Omega}f^{+}\, d\mu 
	+
	\int_{\Omega}f_2\, d\mu 
	=
	\int_{\Omega}f_1\, d\mu 
	+
	\int_{\Omega}f^{-}\, d\mu
\]
Já que todas os termos da igualdade acima 
são finitos obtemos 
\[
	\int_{\Omega}f\, d\mu 
	=
	\int_{\Omega}f^{+}\, d\mu 
	-
	\int_{\Omega}f^{-}\, d\mu 
	=
	\int_{\Omega}f_1\, d\mu 
	-
	\int_{\Omega}f_2\, d\mu.
\]
\end{observacao}


\subsubsection*{Cargas em $(\Omega,\F)$ }

Se $(\Omega,\F)$ é um espaço mensurável uma 
função $\lambda:\F\to\R$ é chamada de uma 
{\bf carga} \index{Carga}
se $\lambda(\emptyset)=0$ e $\lambda$ 
é $\sigma$-aditiva, isto é, se $\{E_n\}$ é uma
sequência de conjuntos mutuamente disjuntos de $\F$ 
então 
	\[
		\lambda\left(\bigcup_{n=1}^{\infty}E_n\right)
		=
		\sum_{n=1}^{\infty} \lambda(E_n).
	\]
Existe uma sutileza muito grande nesta definição
que está relacionada ao seguinte fato.
Já que o lado esquerdo da igualdade acima é 
independente de qualquer reordenamento que se
faça da coleção $\{E_n\}$ a igualdade acima 
requer no fundo que a série que aparece a direita 
seja incondicionalmente somável para todas as 
sequências disjuntas de conjuntos mensuráveis. 

\begin{exercicio}\label{exercicio-cargas-esp-vet}
Mostre que o espaço de todas as cargas
de um espaço mensurável $(\Omega,\F)$ é um espaço vetorial
sobre $\R$, onde a soma de duas cargas 
$\lambda_1$ e $\lambda_2$ é definida por 
$(\lambda_1+\lambda_2)(E)=\lambda_1(E)+\lambda_2(E), 
\ \forall E\in\F$
e a multiplicação de uma carga $\lambda$ 
por um escalar real $\alpha$ é definida por
$(\alpha\cdot \lambda)(E)=\alpha\cdot\lambda(E)$. 
\end{exercicio}


\begin{lema}\label{lema-int-f-dmu-define-uma-carga}
Sejam $(\Omega,\F\,\mu)$ um espaço de medida, 
$f\in L(\Omega,\F,\mu)$. Então a aplicação 
$\lambda:\F\to\R$ dada por 
%
\begin{equation}\label{integral-indefinida}	
	\lambda(E)=\int_{E} f\, d\mu 
\end{equation}
%
é uma carga.
\end{lema}


\begin{proof}
Já que $f^+$ e $f^-$ pertencem a $M^{+}(\Omega,\F)$
segue do Corolário \ref{cor-lambda(E)-eh-uma-medida}
que as funções $\lambda^{+}$ e $\lambda^{-}$ definidas
em $\F$ por 
	\[
		\lambda^{+}(E)\equiv \int_{E} f^{+}\, d\mu 
		\qquad\text{e}\qquad
		\lambda^{-}(E)\equiv \int_{E} f^{-}\, d\mu 		
	\] 
são medidas definidas sobre $\F$. Note que ambas são medidas finitas
pois $f\in L(\Omega,\F,\mu)$. 
Já que $\lambda =\lambda^{+}-\lambda^{-}$ segue do 
Exercício \ref{exercicio-cargas-esp-vet}
que $\lambda$ é uma carga.
\end{proof}

A função $\lambda$ definida em \eqref{integral-indefinida}
é frequentemente chamada de 
\index{Integral!Indefinida}
{\bf integral indefinida de $f$}.
Uma vez que $\lambda$ é uma carga se $\{E_n\}$
é uma sequência mutuamente disjunta em $\F$ cuja 
união é $E$, então temos que
	\[
	\int_{E}f\, d\mu 
	=
	\sum_{n=1}^{\infty} \int_{E_n}f\, d\mu.	
	\]
Nos referimos a esta igualdade dizendo que 
a {\it integral de uma função em $L(\Omega,\F,\mu)$ 
é contavelmente aditiva}.




O próximo resultado é as vezes chamado de 
{\it propriedade da integrabilidade absoluta}
da integral de Lebesgue. Aqui vale mencionar que 
embora o valor absoluto de uma função 
Riemann integrável seja Riemann integrável
em intervalos compactos, isto não é necessariamente
verdadeiro para funções que possuem 
integral de Riemann imprópria, 
por exemplo considere a função $f(x)=x^{-1}\text{sen}\, x$,
definida no intervalo $1\leq x<+\infty$.

Outra diferença importante que deve ser destacada
é quanto ao uso da expressão {\it integrável}
no contexto de integrais de Riemann e Lebesgue. 
No contexto de integral de Riemann integrabilidade
de uma dada função se refere meramente a existência da integral de Riemann 
de tal função podendo ser $\pm\infty$.
Na teoria de integração segundo Lebesgue, 
esta expressão é usada para dizer que tal função 
tem integral é finita. 






\begin{teorema}
Uma função mensurável $f:\Omega\to\R$ 
pertence a $L(\Omega,\F,\mu)$ se, e somente se,
$|f|\in L(\Omega,\,F,\mu)$.  Além do mais é válida
a seguinte desigualdade 
	\[
		\left|
		\int_{\Omega}f\, d\mu 
		\right|
		\leq
		\int_{\Omega}|f|\, d\mu.
	\]
\end{teorema}


\begin{proof}
Por definição temos que $f\in L(\Omega,\,F,\mu)$
se, e somente, se $f^{+}$ e $f^{-}\in L(\Omega,\,F,\mu)$.
Já que $|f|^{+}=|f|=f^{+}+f^{-}$ e $|f|^{-}=0$ a prova de 
que $|f|\in L(\Omega,\F,\P)$ segue do item b) do 
Corolário \ref{cor-linearidade-int-funcao-mens-positiva}
e do Corolário \ref{cor-int-f-nneg-eh-zero-equiv-fnula-qtp}.
Para provar a desigualdade do enunciado basta observar que 
	\begin{align*}
		\left|
		\int_{\Omega}f\, d\mu 
		\right|
		&=
		\left|
		\int_{\Omega}f^{+}\, d\mu - \int_{\Omega}f^{-}\, d\mu 
		\right|
		\\
		&\leq
		\int_{\Omega}f^{+}\, d\mu+\int_{\Omega}f^{-}\, d\mu
		\\
		&=
		\int_{\Omega}|f|\, d\mu 		
	\end{align*}
\end{proof}









\begin{corolario}\label{cor-g-integravel-e-|f|<=|g|-f-integravel}
Seja $(\Omega,\F,\mu)$ um espaço de medida. 
Se $f,g:\Omega\to\R$ são funções mensuráveis 
com $g\in L(\Omega,\F,\mu)$ e $|f|\leq |g|$
então $f\in L(\Omega,\F,\mu)$.
\end{corolario}

\begin{proof}
A prova segue da monotonicidade da integral e do teorema acima.
\end{proof}


Nosso próximo passo é mostrar que a integral é linear 
no espaço $L(\Omega,\F,\mu)$ no seguinte sentido.



\begin{teorema}\label{teo-linearidadde-integral-func-real}
Sejam $(\Omega,\F,\mu)$ um espaço de medida,
$f,g\in L(\Omega,\F,\mu)$ e $\alpha\in\R$. 
Então
\[
	\int_{\Omega}\alpha f\, d\mu 
	= 
	\alpha\int_{\Omega} f\, d\mu
	\quad
	\text{e}
	\quad
	\int_{\Omega} (f+g)\, d\mu 
	= 
	\int_{\Omega} f\, d\mu +\int_{\Omega} g\, d\mu.
\]
\end{teorema}


\begin{proof}
Se $\alpha=0$, então $\alpha f= 0$ em todo ponto e portanto
	\[
	\int_{\Omega}\alpha f\, d\mu 
	=
	0
	= 
	\alpha\int_{\Omega} f\, d\mu.
	\]
Se $\alpha>0$ , então $(\alpha f)^{+}=\alpha\cdot f^{+}$ e 
$(\alpha f)^{-}=\alpha \cdot f^{-}$, assim 
$\alpha\cdot f\in L(\Omega,\F,\mu)$ e também temos que 
\begin{align*}
	\int_{\Omega} \alpha\cdot f\, d\mu 
	&=
	\int_{\Omega} \alpha \cdot f^{+}\, d\mu 
	-
	\int_{\Omega} \alpha \cdot f^{-}\, d\mu 
	\\[0.3cm]
	&=
	\alpha
	\left(
	\int_{\Omega}  f^{+}\, d\mu 
	-
	\int_{\Omega}  f^{-}\, d\mu 
	\right)
	\\[0.3cm]
	&=
	\alpha \int_{\Omega} f\, d\mu.
\end{align*}
O caso $\alpha<0$ pode ser tratado de maneira similar.

Vamos verificar agora que que $f+g\in L(\Omega,\F,\mu)$.
Para isto basta notar que pelo teorema anterior
temos $|f|$ e $|g| \in L(\Omega,\F,\mu)$. Da desigualdade
triangular segue que $|f+g|\leq |f|+|g|$. Aplicando  
corolário anterior temos que $f+g$ é integrável. 
Observe que 
	\[
		f+g = (f^{+}+g^{+})-(f^{-}+g^{-}).
	\]
Já que $f^{+}+g^{+}$ e $f^{-}+g^{-}$ são funções não-negativas
segue da Observação \ref{obs-int-diferenca-2func-positivas}
que 
	\[
		\int_{\Omega} f+g\, d\mu 
		=
		\int_{\Omega} (f^{+}+g^{+})\, d\mu 
		-
		\int_{\Omega} (f^{-}+g^{-})\, d\mu.
	\]
Aplicando o Corolário \ref{cor-linearidade-int-funcao-mens-positiva}
e rearranjando os termos, no lado esquerdo da igualdade acima, ficamos
com
	\begin{align*}
		\int_{\Omega} f+g\, d\mu 
		&=
		\int_{\Omega} f^{+} \, d\mu 
		-
		\int_{\Omega} f^{-} \, d\mu 
		+
		\int_{\Omega} g^{-}\, d\mu 
		-
		\int_{\Omega} g^{-} \, d\mu 
		\\[0.3cm]
		&=
		\int_{\Omega} f \, d\mu 
		+
		\int_{\Omega} g \, d\mu.
	\end{align*}
\end{proof}









Vamos encerrar esta seção estabelecendo um dos 
teoremas de convergência mais importantes 
sobre funções integráveis. 

\section{Teorema da Convergência Dominada}


\begin{teorema}
[Teorema da Convergência Dominada de Lebesgue]
\label{Teo-Convergencia-Dominada}
Sejam $(\Omega,\F\,\mu)$ um espaço de medida,
$\{f_n\}$ uma sequência de funções integráveis 
que converge $\mu$-quase certamente para uma função
mensurável $f:\Omega\to\R$. Se existe uma função 
integrável $g$ tal que $|f_n|\leq g$ para todo
$n\in\N$, então $f$ é integrável e 
	\[
		\int_{\Omega} f\, d\mu 
		=
		\lim_{n\to\infty} \int_{\Omega} f_n\, d\mu.
	\]
\end{teorema}



\begin{proof}
A menos de redefinir as funções $f_n$ e $f$ em 
um conjunto de medida nula, podemos assumir que 
$f_n(\omega)\to f(\omega)$, para todo $\omega\in\Omega$.
Já que $|f_n|\leq g,\ \forall n\in\N$, segue que 
$|f|=\lim_{n\to\infty}|f_n|\leq g$ e do Corolário 
\ref{cor-g-integravel-e-|f|<=|g|-f-integravel}
que $f$ é integrável. Já que $g+f_n\geq 0$
podemos aplicar o 
Teorema \ref{teo-linearidadde-integral-func-real}
Lema de Fatou para obter
\begin{align*}
	\int_{\Omega} g\, d\mu + \int_{\Omega} f\, d\mu 
	&=
	\int_{\Omega} (g+f)\, d\mu 
	\\
	&\leq 
	\liminf_{n\to\infty}\int_{\Omega} (g+f_n)\, d\mu 
	\\
	&=
	\liminf_{n\to\infty}
	\left( 
		\int_{\Omega} g\, d\mu 
		+
		\int_{\Omega} f_n\, d\mu 
	\right)
	\\
	&=
	\int_{\Omega} g\, d\mu 
	+
	\liminf_{n\to\infty}\int_{\Omega} f_n\, d\mu.
\end{align*}
Já que $g$ é integrável segue da desigualdade 
acima que 	
	\[
	\int_{\Omega} f\, d\mu
	\leq	
	\liminf_{n\to\infty}\int_{\Omega} f_n\, d\mu.
	\]
Observando que $g-f_n\geq 0$ outra aplicação do 
Teorema \ref{teo-linearidadde-integral-func-real} 
e do Lema de Fatou nos fornece a desigualdade
\begin{align*}
	\int_{\Omega} g\, d\mu - \int_{\Omega} f\, d\mu 
	&=
	\int_{\Omega} (g-f)\, d\mu 
	\\
	&\leq 
	\liminf_{n\to\infty}\int_{\Omega} (g-f_n)\, d\mu 
	\\
	&=
	\liminf_{n\to\infty}
	\left( 
		\int_{\Omega} g\, d\mu 
		-
		\int_{\Omega} f_n\, d\mu 
	\right)
	\\
	&=
	\int_{\Omega} g\, d\mu 
	-
	\limsup_{n\to\infty}\int_{\Omega} f_n\, d\mu.
\end{align*}
Usando novamente que $g$ é integrável podemos 
concluir da desigualdade acima que 
	\[
	\limsup_{n\to\infty}\int_{\Omega} f_n\, d\mu
	\leq	
	\int_{\Omega} f\, d\mu.
	\]
Combinado as duas estimativas obtidas para a
integral de $f$ obtemos finalmente que 
	\[
	\lim_{n\to\infty}\int_{\Omega} f_n\, d\mu
	=	
	\int_{\Omega} f\, d\mu.
	\]
\end{proof}
