\chapter[Aula 13]{Esperança Condicional}
\chaptermark{}


\section{Definição e Propriedades da Esperança Condicional} 
Nesta seção usando o Teorema de Radon-Nikodým vamos mostrar 
como são construídas as probabilidade e esperança 
condicional. 
Em várias partes desta seção vamos considerar 
um espaço de medida $(\Omega,\mathcal{F},\mu)$, 
uma sub-$\sigma$-álgebra $\mathcal{B}$ de $\mathcal{F}$ 
e usaremos a notação $\nu=\mu|_{\mathcal{B}}$, 
para denotar a restrição da medida $\mu$ 
a sub-$\sigma$-álgebra $\mathcal{B}$. 
Vamos denotar por $L^1(\Omega,\mathcal{F},\mu)$ o conjunto 
de todas as funções $\mathcal{F}$-mensuráveis 
$f:\Omega\to\overline{\mathbb{R}}$ 
tais que $\int_{\Omega} |f|\, d\mu <+\infty$. 

\begin{observacao}
Nos textos de teoria da Medida e Análise funcional
$L^1(\Omega,\mathcal{F},\mu)$ é uma notação consagrada para 
denotar um espaço de classes de equivalência obtido 
pela identificação de duas funções que diferem em um 
subconjunto de $\Omega$ de medida zero. 
Voltaremos a discutir isto mais a frente no texto. 
Neste ponto para evitar confusão e fixar a notação 
vamos pensar em $L^1(\Omega,\mathcal{F},\mu)$ 
apenas como um espaço de funções.
\end{observacao}


\begin{teorema}\label{teo:esperanca-condicional-definicao} 
Sejam $(\Omega,\mathcal{F},\mu)$ um espaço de medida finita, 
$\mathcal{B}$ uma sub-$\sigma$-álgebra de $\mathcal{F}$ 
e $\nu=\mu|_{\mathcal{B}}$. 
Para cada $f\in L^1(\Omega,\mathcal{F},\mu)$ 
existe uma função $g$ mensurável segundo a 
$\sigma$-álgebra $\mathcal{B}$ 
com $g\in L^1(\Omega,\mathcal{B},\nu)$ tal que 
\[
\int_{E} f\, d\mu 
=
\int_{E} g\, d\nu \qquad \text{para todo} \ E\in\mathcal{B}.
\]
Além do mais se $g'\in L^1(\Omega,\mathcal{B},\nu)$ 
é uma outra função satisfazendo a igualdade acima, 
então $g=g'$ $\nu$-q.t.p. 
\end{teorema}

\begin{observacao}
A função $g$ cuja a existência é garantida 
no Teorema \eqref{teo:esperanca-condicional-definicao} 
é nosso ponto de partida para apresentar a definição 
da esperança condicional.
A esperança condicional será um dos conceitos 
fundamentais deste texto.
\end{observacao}



\begin{proof}
Vamos considerar inicialmente que $f\geq 0$. 
Seja $\eta:\mathcal{B}\to[0,+\infty)$ a medida definida por 
\begin{equation}\label{teo:esperanca-cond-aux1}
\eta(E)=\int_{E} f\, d\mu.
\end{equation}
Note que se $\nu(E)=0$ então obviamente 
$\mu(E)=0$, mas se $\mu(E)=0$ então a integral 
acima é igual a zero logo
$\eta(E)=0$, e assim temos que $\eta\ll\nu$. 
Daí segue do Teorema de Radon-Nikodym que existe uma função 
$\mathcal{B}$-mensurável $g:\Omega\to\mathbb{R}$ tal que 
\begin{equation}
\label{teo:esperanca-cond-aux2}
\eta(E)=\int_{E} g\, d\nu
\end{equation}
Usando a hipótese $f\in L^1(\Omega,\mathcal{F},\mu)$ 
e tomando $E=\Omega$ nas igualdades 
\eqref{teo:esperanca-cond-aux1} e \eqref{teo:esperanca-cond-aux2} 
concluímos que $g\in L^1(\Omega,\mathcal{B},\nu)$. 
O Teorema de Radon-Nikodým garante que $g$ é 
unicamente determinada $\nu$-q.t.p. 
e portanto o teorema fica provado no caso em que $f\geq 0$.
No caso em que $f$ toma valores reais basta 
repetir argumento apresentado acima para $f^{+}$ e $f^{-}$.
\end{proof}




\begin{definicao} 
Sejam $(\Omega,\mathcal{F},\mu)$ um espaço de {\bf probabilidade}, 
$f\in L^1(\Omega,\mathcal{F},\mu)$, $\mathcal{B}$ 
uma sub-$\sigma$-álgebra de $\mathcal{F}$ e $\nu=\mu|_{\mathcal{B}}$. 
Dizemos que uma função $\mathcal{B}$-mensurável 
$g:\Omega\to\mathbb{R}$ é uma esperança condicional
de $f$ dada a $\sigma$-álgebra $\mathcal{B}$, se 
\[
\int_{E} f\, d\mu 
=
\int_{E} g\, d\nu \qquad \text{para todo} 
\ E\in\mathcal{B}.
\]
\end{definicao}

Como vimos acima o 
Teorema \eqref{teo:esperanca-condicional-definicao} 
garante a existência de uma esperança condicional para toda 
$f\in L^1(\Omega,\mathcal{F},\mu)$ e para toda 
sub-$\sigma$-álgebra $\mathcal{B}$ de $\mathcal{F}$. 
Cada uma das funções $g$ satisfazendo 
a condição acima é chamada de uma versão da esperança 
condicional de $f$ com respeito a $\mathcal{B}$. 
O Teorema \eqref{teo:esperanca-condicional-definicao} 
também nos garante que quaisquer duas 
versões da esperança condicional 
são funções que diferem apenas em um conjunto de medida $\nu$ nula. 
Já que do ponto de vista de integração 
a escolha de uma versão arbitrária da esperança condicional 
não afeta nenhum cálculo é comum tomarmos uma versão qualquer
e denotá-la por $\mathbb{E}[f|\mathcal{B}]$. 
Ressaltamos que $\mathbb{E}[f|\mathcal{B}]$ 
é uma função ($\mathbb{E}[f|\mathcal{B}]:\Omega\to\mathbb{R}$) 
cujo domínio é o conjunto $\Omega$ 
e toma valores em $\mathbb{R}$. 
Na sequência apresentamos algumas de suas principais propriedades.



\begin{proposicao}
	[Linearidade da Esperança Condicional]
	\label{teo:linearidade-esperanca-condicional}
Sejam $(\Omega,\mathcal{F},\mu)$ um espaço de medida finita, 
$\mathcal{B}$ uma sub-$\sigma$-álgebra 
de $\mathcal{F}$ e $\nu=\mu|_{\mathcal{B}}$. 
Para todas $f,g\in L^1(\Omega,\mathcal{F},\mu)$ 
e $\alpha\in\mathbb{R}$ temos que 
\[
\mathbb{E}[f+\alpha g|\mathcal{B}]
=
\mathbb{E}[f|\mathcal{B}]
+
\alpha\mathbb{E}[g|\mathcal{B}] \ \ \ \nu-\text{q.t.p.}
\]
\end{proposicao}

\begin{proof}
Pela definição de esperança condicional temos 
\begin{equation}\label{eq:aux1-lin-esp-cond}
\int_{E} (f+\alpha g)\, d\mu
=
\int_{E} \mathbb{E}[f+\alpha g|\mathcal{B}]\, d\nu 
\qquad \text{para todo} \ E\in\mathcal{B}.
\end{equation}
Usando a linearidade da Integral de Lebesgue e a 
definição de esperança condicional de 
$f$ e $g$ dado a $\sigma$-álgebra
$\mathcal{B}$, temos
para todo $E\in\mathcal{B}$ que 
\begin{equation}
\label{eq:aux2-lin-esp-cond}
\int_{E} (f+\alpha g)\, d\mu
=
\int_{E} f\, d\mu+\alpha\int_{E} g\, d\mu
=
\int_{E} \mathbb{E}[f|\mathcal{B}]\, d\nu 
+
\alpha\int_{E} \mathbb{E}[g|\mathcal{B}]\, d\nu.
\end{equation}
Já que o lado esquerdo em \eqref{eq:aux1-lin-esp-cond} 
e \eqref{eq:aux2-lin-esp-cond} são os mesmos temos  
\[
\int_{E} \mathbb{E}[f+\alpha g|\mathcal{B}]\, d\nu
=
\int_{E} \mathbb{E}[f|\mathcal{B}]\, d\nu 
+
\alpha\int_{E} \mathbb{E}[g|\mathcal{B}]\, d\nu
\qquad \text{para todo} \ E\in\mathcal{B}.
\]
Usando novamente a linearidade da integral e que 
$E$ é arbitrário em $\mathcal{B}$ concluímos que
$
\mathbb{E}[f+\alpha g|\mathcal{B}]
=
\mathbb{E}[f|\mathcal{B}]
+
\alpha\mathbb{E}[g|\mathcal{B}] \ \ \ \nu-\text{q.t.p.}$.
\end{proof}

A prova da próxima proposição é completamente análoga, 
mas repetiremos todos os detalhes 
para que o leitor menos experiente vá se 
familiarizando com o conceito da esperança 
condicional e suas propriedades.


\begin{proposicao}[$\mathcal{B}$-homogenidade da Esperança Condicional]
\label{teo:propriedades-esperanca-condicional}
Sejam $(\Omega,\mathcal{F},\mu)$ um espaço 
de medida finita, $\mathcal{B}$ uma sub-$\sigma$-álgebra 
de $\mathcal{F}$ e $\nu=\mu|_{\mathcal{B}}$. 
Se $f\in L^1(\Omega,\mathcal{F},\mu)$ e $g$ 
é uma função $\mathcal{B}$-mensurável então 
\[
\mathbb{E}[f g|\mathcal{B}]
=
g\mathbb{E}[f|\mathcal{B}] \ \ \ \nu-\text{q.t.p.}
\]
\end{proposicao}



\begin{proof}

Suponha que inicialmente que $g=\chi_{F}$ 
para algum $F\in\mathcal{B}$. 
Da definição de esperança condicional temos  
\begin{equation}\label{eq:aux1-esp-cond}
\int_{E} f\chi_{F}\ d\mu
=
\int_{E}\mathbb{E}[f\chi_{F}|\mathcal{B}]\, d\nu
\end{equation}
para todo $E\in\mathcal{B}$. 
Como estamos supondo que $F\in\mathcal{B}$, 
temos que $E\cap F\in\mathcal{B}$ 
logo, aplicando novamente a definição da 
esperança condicional obtemos  
%
%
\begin{equation}
\label{eq:aux2-esp-cond}
\int_{E} f\chi_{F}\ d\mu=\int_{E\cap F} f\ d\mu
=
\int_{E\cap F}\mathbb{E}[f|\mathcal{B}]\, d\nu
=
\int_{E}\chi_{F}\mathbb{E}[f|\mathcal{B}]\, d\nu.
\end{equation}
De \eqref{eq:aux1-esp-cond} e \eqref{eq:aux2-esp-cond} 
segue que 
\[
\int_{E}\mathbb{E}[f\chi_{F}|\mathcal{B}]\, d\nu
=
\int_{E}\chi_{F}\mathbb{E}[f|\mathcal{B}]\, d\nu,
\qquad \text{para todo}\ E\in\mathcal{B}.
\]
Logo 
$\
mathbb{E}[f\chi_{F}|\mathcal{B}]
=\chi_{F}\mathbb{E}[f|\mathcal{B}]
\ \nu$-q.t.p. e isto prova o teorema 
para o caso $g=\chi_{F}$. 

Vamos mostrar agora o teorema no caso em que $g$ 
é uma função simples $\mathcal{B}$-mensurável. 
Suponha que sua representação padrão seja 
$g = \sum_{j=1}^n a_j \chi_{E_j}$, 
onde $a_j\in\mathbb{R}$ e $E_j\in \mathcal{B}$ 
para todo $j=1,\ldots,n$.
Pela linearidade da esperança condicional e pela 
propriedade que acabamos de demostrar acima, 
uma indução mostra que as seguintes igualdades 
são verdadeiras
\[
\mathbb{E}[gf|\mathcal{B}]
=
\mathbb{E}\left[\left(\sum_{j=1}^n a_j \chi_{E_j}\right)
f\ \Big|\mathcal{B}\right]
=
\sum_{j=1}^n a_j \mathbb{E}[\chi_{E_j}f|\mathcal{B}]
=
\sum_{j=1}^n a_j\chi_{E_j} \mathbb{E}[f|\mathcal{B}]
=
g\mathbb{E}[f|\mathcal{B}].
\]
Resta agora estabelecer este fato para funções 
$g\in L^1(\Omega,\mathcal{B},\nu)$. 
Primeiro vamos mostrar que 
é suficiente provar a proposição para $f$ e $g$ não negativas. 
De fato, assuma que o teorema seja válido para 
$f^+,f^-\in L^1(\Omega,\mathcal{F},\mu)$ e 
$g^+,g^-\in L^1(\Omega,\mathcal{B},\nu)$. 
Já que $f=f^+-f^-$ e $g=g^+-g^-$
temos que 
\begin{align*}
	\mathbb{E}[gf|\mathcal{B}]
	&=
	\mathbb{E}[g^+f^+-g^-f^+-g^+f^-+g^-f^-|\mathcal{B}]
	\\[0.2cm]
	&= 
	\mathbb{E}[g^+f^+|\mathcal{B}]- \mathbb{E}[g^-f^+|\mathcal{B}] 
		-\mathbb{E}[g^+f^-|\mathcal{B}]+\mathbb{E}[g^-f^-|\mathcal{B}]
	\\[0.2cm]
	&= 
	g^+\mathbb{E}[f^+|\mathcal{B}]-g^-\mathbb{E}[f^+|\mathcal{B}]
	-g^+\mathbb{E}[f^-|\mathcal{B}]+g^-\mathbb{E}[f^-|\mathcal{B}]
	\\[0.2cm]
	&=
	(g^+-g^-)\mathbb{E}[f^+|\mathcal{B}]
	-(g^+-g^-)\mathbb{E}[f^-|\mathcal{B}]
	\\[0.2cm]
	&=
	g\mathbb{E}[f^+|\mathcal{B}]-g\mathbb{E}[f^-|\mathcal{B}]
	\\[0.2cm]
	&=
	g(\mathbb{E}[f^+|\mathcal{B}]-\mathbb{E}[f^-|\mathcal{B}])
	\\[0.2cm]
	&=
	g\mathbb{E}[f|\mathcal{B}].
\end{align*}
Portanto só resta mostrar que a proposição é verdadeira 
para $f\in L^1(\Omega,\mathcal{F},\mu)$ 
e $g\in L^1(\Omega,\mathcal{B},\nu)$ ambas não negativas. 
Pelo Teorema \ref{teo:aproximacao-monotona-por-func-simples} 
existe uma sequência $g_n:\Omega\to\mathbb{R}$ 
monótona crescente de funções simples $\mathcal{B}$-mensuráveis 
tal que $g_n\uparrow g$. Já que $f\geq 0$ 
podemos afirmar que $g_nf\uparrow gf$. 
Pela definição de esperança condicional 
%
%
\begin{equation}\label{eq:aux3-esp-cond}
\int_{E}g_nf\, d\mu 
= 
\int_{E} \mathbb{E}[g_nf|\mathcal{B}]\, d\nu,
\qquad\text{para todo}\ E\in\mathcal{B}.
\end{equation}
Já sabemos que para toda função $\mathcal{B}$-mensurável 
$g_n$ simples que a seguinte igualdade é verdadeira 
$\mathbb{E}[g_nf|\mathcal{B}]=g_n\mathbb{E}[f|\mathcal{B}]$. 
Como $f\geq 0$ é imediato verificar que 
$\mathbb{E}[f|\mathcal{B}]\geq 0$, assim 
$g_n\mathbb{E}[f|\mathcal{B}]\uparrow g\mathbb{E}[f|\mathcal{B}]$ 
o que nos permite aplicar o teorema da convergência monótona em 
ambos os lados de \eqref{eq:aux3-esp-cond} e concluir que 
para todo $E\in\mathcal{B}$.
\[
\int_{E}gf\, d\mu
=
\lim_{n\to\infty}\int_{E}g_nf\, d\mu 
= 
\lim_{n\to\infty}\int_{E} g_n\mathbb{E}[f|\mathcal{B}]\, d\nu
=
\int_{E} g\mathbb{E}[f|\mathcal{B}]\, d\nu.
\]
Observando que a integral mais a esquerda da igualdade acima é, 
por definição de esperança condicional, igual a 
$\int_{E} \mathbb{E}[gf|\mathcal{B}]\, d\nu$ e
que esta igualdade é válida para todo $E\in\mathcal{B}$ 
concluímos que 
$
\mathbb{E}[gf|\mathcal{B}]
=
g\mathbb{E}[f|\mathcal{B}] \ \nu$-q.t.p. 
e assim está completa a prova da proposição.
\end{proof}




\begin{proposicao}
[Monotonicidade da Esperança Condicional] 
Seja $(\Omega,\mathcal{F},\mu)$ um espaço de probabilidade. 
Se $f,g\in L^1(\Omega,\mathcal{F},\mu)$ são tais que $f\leq g$ e 
$\mathcal{B}$ é uma sub-$\sigma$-álgebra qualquer 
de $\mathcal{F}$ então 
$\mathbb{E}[f|\mathcal{B}]\leq \mathbb{E}[g|\mathcal{B}]$ 
$\nu$-q.t.p., onde $\nu=\mu|_{\mathcal{B}}$
\end{proposicao}

\begin{proof}
Pela definição da esperança condicional temos 
$$
\int_{E} \mathbb{E}[f|\mathcal{B}]\, d\nu= \int_{E} f\, d\mu 
\leq 
\int_{E} g\, d\mu=\int_{E} \mathbb{E}[g|\mathcal{B}]\, d\nu,
\qquad \text{para todo}\ E\in\mathcal{B}.
$$
De onde segue o resultado.
\end{proof}

\begin{teorema}[Teorema da Convergência Monótona para Esperança Condicional]
\label{teo:convergencia-monotona-esp-cond}
Seja $(\Omega,\mathcal{F},\mu)$ um espaço de probabilidade, 
$\mathcal{B}$ uma sub-$\sigma$-álgebra 
de $\mathcal{F}$ e $\nu=\mu|_{\mathcal{B}}$. 
Se $f_n:\Omega\to[0,\infty]$ é uma sequência de funções
$\mathcal{B}$-mensuráveis tal que 
$f_n\uparrow f$ $\mu$-q.t.p. então 
$\mathbb{E}[f_n|\mathcal{F}]\uparrow \mathbb{E}[f|\mathcal{F}]$ 
$\nu$-q.t.p..
\end{teorema}


\begin{proof}
Por linearidade e monotonicidade da esperança condicional temos 
\[
0\leq \int_{E} \mathbb{E}[f|\mathcal{B}]\, d\nu 
- 
\int_{E} \mathbb{E}[f_n|\mathcal{B}]\, d\nu
=\int_{E} f\, d\mu - \int_{E} f_n\, d\mu.
\qquad \text{para todo}\ E\in\mathcal{B}.
\]
Como $\int_{E} \mathbb{E}[f_n|\mathcal{B}]\, d\nu$ 
é uma sequência de números reais não decrescente e 
limitada ela possui limite. 
Assim podemos tomar o limite quando $n$ vai a infinito 
em ambos os lados da desigualdade acima e concluir 
usando o Teorema da Convergência Monótona, 
no lado direito,  que
\[
0
\leq 
\int_{E} \mathbb{E}[f|\mathcal{B}]\, d\nu 
- 
\lim_{n\to\infty}\int_{E} \mathbb{E}[f_n|\mathcal{B}]\, d\nu
=0
\qquad \text{para todo}\ E\in\mathcal{B}.
\]
Já que $\mathbb{E}[f_n|\mathcal{B}]$ é uma sequência
monótona de funções, existe o limite 
$\lim_{n\to\infty}\mathbb{E}[f_n|\mathcal{B}](\omega)$ 
$\nu$-q.t.p.
Logo podemos aplicar novamente o 
Teorema da Convergência Monótona
na desigualdade acima e concluir que 
\[
\int_{E} \mathbb{E}[f|\mathcal{B}]\, d\nu 
= 
\int_{E} \lim_{n\to\infty}\mathbb{E}[f_n|\mathcal{B}]\, d\nu
\qquad \text{para todo}\ E\in\mathcal{B}.
\]
Desta forma acabamos de mostrar que 
$\mathbb{E}[f_n|\mathcal{B}]\uparrow \mathbb{E}[f|\mathcal{B}]$ 
$\nu$-q.t.p.
\end{proof}



\begin{exercicio} 
Prove a chamada propriedade de contração 
para a esperança condicional. 
Seja $(\Omega,\mathcal{F},\mu)$ um espaço 
de probabilidade, $\mathcal{B}$ uma 
sub-$\sigma$-álgebra de $\mathcal{F}$, 
$\nu=\mu|_{\mathcal{B}}$ e 
$f\in L^1(\Omega,\mathcal{F},\mu)$. 
Mostre que
\[
\int_{\Omega} \big|\mathbb{E}[f|\mathcal{B}]\big|\, d\nu 
\leq 
\int_{\Omega} |f|\, d\mu.
\]
\end{exercicio}



\begin{teorema}[Convergência Dominada para Esperança Condicional] 
Seja $(\Omega,\mathcal{F},\mu)$ um espaço de 
medida $\mathcal{B}$ sub-$\sigma$-álgebra 
de $\mathcal{F}$ e $\nu=\mu|_{\mathcal{B}}$. 
Suponha que $f_n:\Omega\to [0,+\infty)$ 
seja uma sequência de funções 
em $L^1(\Omega,\mathcal{F},\mu)$ que converge 
$\mu$-q.t.p. para $f:\Omega\to\mathbb{R}$.
Se existe uma função integrável 
$g:\Omega\to\mathbb{R}$ tal que $|f_n|\leq g$ 
para todo $n\in\mathbb{N}$, então 
\[
\lim_{n\to\infty} \mathbb{E}[f_n|\mathcal{B}]
=
\mathbb{E}[f|\mathcal{B}] \ \ 
\nu-\text{q.t.p.}
\]
\end{teorema}


\begin{proof}
Seja $h_n:\Omega\to\mathbb[0,+\infty)$ 
uma sequência de funções dada por 
\[
h_n(\omega)
=
\displaystyle\sup_{\{j\in\mathbb{N}:n\leq j\}}
|f(\omega)-f_j(\omega)|.
\]
Observe que para todo $n\in\mathbb{N}$ temos 
$h_{n+1}(\omega)\leq h_{n}(\omega)$ e 
$h_n(\omega)\leq |f(\omega)|+|g(\omega)|$. 
Desta duas propriedades podemos concluir que 
$(|f|+|g|-h_n)\uparrow (|f|+|g|)$ $\mu$-q.t.p.. 
Pelo Teorema \ref{teo:convergencia-monotona-esp-cond} 
(convergência monótona) segue que
\[
\mathbb{E}[(|f|+|g|-h_n)|\mathcal{B}]
\uparrow
 \mathbb{E}[(|f|+|g|)|\mathcal{B}]
 \ \  \nu-\text{q.t.p.}
\]
Usando a linearidade e monotonicidade da esperança 
condicional segue da observação acima que 
$\mathbb{E}[h_n|\mathcal{B}]\downarrow 0$.
Usando novamente a monotonicidade obtemos 
\[
\big|\mathbb{E}[f|\mathcal{B}]-\mathbb{E}[f_n|\mathcal{B}]  \big|
\leq
\mathbb{E}[|f-f_n||\mathcal{B}]
\leq 
\mathbb{E}[h_n|\mathcal{B}]\downarrow 0 
\ \ \nu-\text{q.t.p.}.
\]
\end{proof}



\begin{proposicao}
Sejam $(\Omega,\mathcal{F},\mu)$ um espaço 
de medida $f:\Omega\to\mathbb{R}$ uma 
função $\mathcal{F}$-mensurável, 
$\mathscr{A}\subset\mathcal{B}\subset\mathcal{F}$ 
$\sigma$-álgebras,
$\nu=\mu|_{\mathcal{B}}$ e $\eta=\mu|_{\mathscr{A}}$.
Então 
$
\mathbb{E}[\mathbb{E}[f|\mathcal{B}]|\mathscr{A}]
=
\mathbb{E}[f|\mathscr{A}]$ 
$\eta$-q.t.p..
\end{proposicao}


\begin{proof}
Por definição da esperança condicional temos 
\[
\int_{F} 
\mathbb{E}[\mathbb{E}[f|\mathcal{B}]|\mathscr{A}]
\, d\eta
=
\int_{F} \mathbb{E}[f|\mathcal{B}]\, d\nu,
\qquad\text{para todo}\ F\in\mathscr{A}.
\]
Como $F$ também é um conjunto $\mathcal{B}$-mensurável 
segue novamente da definição de esperança condicional que 
$$
\int_{F} \mathbb{E}[f|\mathcal{B}]\, d\nu
=
\int_{F} f\, d\mu,
\qquad\text{para todo}\ F\in\mathscr{A}.
$$
Das duas igualdades acima, temos que 
$$
\int_{F} f\, d\mu 
=
\int_{F} \mathbb{E}[\mathbb{E}[f|\mathcal{B}]|\mathscr{A}]
\, d\eta,
\qquad\text{para todo}\ F\in\mathscr{A}.
$$
Da unicidade $\eta$-q.t.p. garantida pelo 
Teorema \ref{teo:esperanca-condicional-definicao} segue que 
o lado direito da igualdade acima é igual a 
$\mathbb{E}[f|\mathscr{A}]$ $\eta$-q.t.p., 
o que prova a proposição.
\end{proof}