\chapter[Aula 8]{Lei Zero-Um de Komolgorov}
\chaptermark{}

\section{Uma Propriedade de Sequências de v.a.'s Exponenciais Independentes}
\label{secao-props-va-exp-indep}

Nesta seção vamos assumir que $\{X_n\}$ é uma sequência 
de v.a.'s iid com distribuição exponencial de parâmetro 1,
isto é, $X_n\geq 0$ e  
\[
	\P(X_n>x)=e^{-x}.
\] 

O objetivo desta seção é provar que 
%
	\begin{equation}\label{exemplo-seq-exp-par-1-indep-cresce-logn}
		\P
		\left(
			\limsup_{n\to\infty}
			\frac{X_n}{\log n} 
			=1
		\right)	
		=1.
	\end{equation}

Este resultado é as vezes considerado surpreendente.
Esta tendência está associada ao erro comum que alguns
matemáticos cometem pensando que sequências iid 
são ``moralmente constantes'' e portanto a divisão 
por $\log n$ deveria fazer com que o limite fosse
zero. 
Entretanto muito frequentemente a sequência $\{X_n\}$
toma valores que vão aumentando com o índice e 
este aumento é aproximadamente dado por $\log n$.

Para provar este resultado vamos precisar de um fato 
bastante simples que se $\{A_n\}$ é uma sequência 
de eventos em um espaço de probabilidade $(\Omega,\F,\P)$
tal que $\P(A_n)=1$ então $\P(\cap_{n=1}^{\infty} A_n)=1$,
veja Lista 1 de exercícios.


\medskip

\noindent
{\bf Prova de \eqref{exemplo-seq-exp-par-1-indep-cresce-logn} }.
Se $\omega\in \Omega$ é tal que 
	\[
		\limsup_{n\to\infty}
		\frac{X_n(\omega)}{\log n} 
		=1,
	\]
isto significa que 
\begin{itemize}
	\item[a)]
	para todo $\varepsilon>0$ temos que se $n\geq n_0\equiv n_0(\omega)$ então
	$\displaystyle\frac{X_n(\omega)}{\log n}\leq 1+\varepsilon$.
	
	
	\item[b)]
	para todo $\varepsilon>0$ para infinitos valores de $n$ 
	temos que 	
	$1-\varepsilon<\displaystyle\frac{X_n(\omega)}{\log n}$.
		
\end{itemize}




Observe que o item a) diz que não existe nenhuma 
subsequência com limite maior que $1+\varepsilon$ enquanto
b) diz que sempre existe uma subsequência limitada inferiormente
por $1-\varepsilon$. 




Para qualquer sequência $\varepsilon_k\downarrow 0$, 
note que temos a seguinte igualdade
de conjuntos
\begin{align*}
\left\{ \limsup_{n\to\infty} \frac{X_n}{\log n} =1  \right\}
&
\\[0.3cm]
&
\hspace*{-2cm}
=\bigcap_{k=1}^{\infty}
\left[ 
	\liminf_{n\to\infty} 
		\left\{ 
			\frac{X_n}{\log n} \leq 1+\varepsilon_k 
		\right\}  
\right]
\quad
\bigcap
\quad
\bigcap_{k=1}^{\infty}
\left[ 
	\limsup_{n\to\infty} 
		\left\{ 
			\frac{X_n}{\log n} > 1-\varepsilon_k  
		\right\}  
\right].
\end{align*}
%
%
%
%
Para provar que o lado direito acima tem probabilidade
um, basta mostrar que cada um dos eventos 
entre colchetes tem probabilidade um. 

Para todo $k\in\N$ fixado temos 
\begin{align*}
\sum_{n=1}^{\infty} 
\P\left(  
	\frac{X_n}{\log n} > 1-\varepsilon_k 
\right)
&=
\sum_{n=1}^{\infty} 
\P\left(  
	X_n > (1-\varepsilon_k)\log n  
\right)
\\
&=
\sum_{n=1}^{\infty} 
\exp\left(  
	 -(1-\varepsilon_k)\log n  
\right)
\\
&=
\sum_{n=1}^{\infty} 
\frac{1}{n^{1-\varepsilon_k}}
\\
&=
\infty,
\end{align*}
onde na última desigualdade usamos que $k$ 
está fixado. Assim segue da Lei Zero-Um de Borel
que 
	\[
		\P\left( 
			\limsup_{n\to\infty} 
			\left\{ 
				\frac{X_n}{\log n} > 1-\varepsilon_k  
			\right\}  
		\right)
		=
		1.
	\]

De maneira análoga podemos verificar que
\begin{align*}
\sum_{n=1}^{\infty} 
\P\left(  
	\frac{X_n}{\log n} > 1+\varepsilon_k 
\right)
&=
\sum_{n=1}^{\infty} 
\P\left(  
	X_n > (1+\varepsilon_k)\log n  
\right)
\\
&=
\sum_{n=1}^{\infty} 
\exp\left(  
	 -(1+\varepsilon_k)\log n  
\right)
\\
&=
\sum_{n=1}^{\infty} 
\frac{1}{n^{1+\varepsilon_k}}
\\
&<
\infty.
\end{align*}
Aplicando novamente a Lei Zero-Um de Borel
temos que 
	\[
		\P\left( 
			\limsup_{n\to\infty} 
			\left\{ 
				\frac{X_n}{\log n} > 1+\varepsilon_k  
			\right\}  
		\right)
		=
		0.
	\]
Tomando complementar obtemos 
	\[
		\P\left( 
			\liminf_{n\to\infty} 
			\left\{ 
				\frac{X_n}{\log n} \leq 1+\varepsilon_k  
			\right\}  
		\right)
		=1.
	\]
O que encerra a demonstração.















\section{Vetores Aleatórios em $\R^n$}

Se consideramos $\R^n$ como um espaço munido de uma 
topologia $\tau$ então a $\sigma$-álgebra de Borel 
é a $\sigma$-álgebra gerada pelos conjuntos abertos,
isto é, $\sigma(\tau)$.
Nos interessa apenas o caso onde $\tau$ é induzida 
pela métrica euclidiana e neste caso a 
$\sigma$-álgebra de Borel em $\R^n$ é 
$\sigma$-álgebra gerada pela coleção 
dos retângulos da forma 
	\[
		(a_1,b_1) \times\ldots\times (a_n,b_n)
		\equiv
		\{
		(x_1,\ldots,x_n)\in\R^n\ ; a_i<x_i<b_i,\ i=1,\ldots,n
		\},
	\]
onde $a_i,b_i\in \R$. Esta $\sigma$-álgebra será denotada
por $\mathscr{B}(\R^n)$.



\begin{definicao}[Vetor Aleatório]
	Seja $(\Omega,\F,\P)$ um espaço de probabilidade.
	Um vetor aleatório é uma aplicação $\mathbf{X}:\Omega\to\R^n$,
	tal que para todo boreliano $E\in \R^n$ temos que 
	$\mathbf{X}^{-1}(E)\in \F$.
\end{definicao}


\begin{exercicio}
Seja $(\Omega,\F,\P)$ um espaço de probabilidade e 
$\mathbf{X}$ um vetor aleatório em $\R^n$.  
Mostre que para qualquer coleção 
$\mathcal{C}$ de subconjuntos de $\R^n$ que 
$
\mathbf{X}^{-1}(\sigma(\mathcal{C}))
=
\sigma(\mathbf{X}^{-1}(\mathcal{C}))
$
\end{exercicio}


Segue do exercício acima que uma
aplicação $\mathbf{X}:\Omega\to\R^n$ é um vetor aleatório
em $(\Omega,\F,\P)$ se, e somente se, 
$\mathbf{X}^{-1}(R)\in \F$ para todo retângulo $R\subset \R^n$.   



\begin{proposicao}\label{prop-caracterizacao-vetor-aleatorio}
Seja $(\Omega,\F,\P)$ um espaço de probabilidade.
Uma aplicação $\mathbf{X}:\Omega\to\R^n$ 
dada por $\mathbf{X}(\omega)=(X_1(\omega),\ldots,X_n(\omega))$
é um vetor aleatório, se e somente se, $X_i:\Omega\to\R$ 
para todo $i=1,\ldots n$ é uma v.a. real.
\end{proposicao}


\begin{proof}
Vamos supor que $\mathbf{X}$ é um vetor aleatório.
Considere as projeções $\pi_i:\R^n\to \R$ dadas por 
$\pi_i(x_1,\ldots,x_n)=x_i$ e seja $I=(-\infty,a)$,
onde $a\in\R$. Já que $X_i = \pi_i\circ \mathbf{X}$,
temos que 
\[
	X_i^{-1}\big((-\infty,a)\big) 
	= 
	\mathbf{X}^{-1}\Big( \pi_i^{-1}\big( (-\infty,a)\big)\Big).
\]
Já que 
$\pi_i^{-1}\big( (-\infty,a)\big)
 = 
 \R\times\ldots\times\R\times (-\infty,a)\times\R\times\ldots\times\R
 \in
 \mathscr{B}(\R^n)
$
e $\mathbf{X}^{-1}(\mathscr{B}(\R^n))\subset \F$ temos que 
que $X_i$ é variável aleatória.

Para provar a recíproca primeiro lembramos que
a $\sigma$-álgebra dos gerada pelos 
retângulos abertos é igual a $\mathscr{B}(\R^n)$. Portanto 
segue do comentário que está logo acima do enunciado
desta proposição que é suficiente provar que 
$\mathbf{X}^{-1}(R)\in F$ para todo retângulo $R$ em 
$\R^n$. 

Supondo que $X_1,\ldots,X_n$ são variáveis aleatórias e 
fixando um retângulo aberto $R$ da forma 
$R=I_1\times\ldots\times I_n$, onde $I_j=(a_j,b_j)$ com $a_j<b_j$
temos que 
	\[
	\mathbf{X}^{-1}(R) 
	= 
	\bigcap_{j=1}^n X_j^{-1}(I_j).
	\]	
Já que estamos assumindo que 
$X_j$ para todo $j=1,\ldots,n$ é v.a.
então $X_j^{-1}(I_j)\in \F$, logo $\mathbf{X}^{-1}(R)\in \F$,
como $R$ é um retângulo arbitrário segue da observação feita 
acima que a prova está concluída.
\end{proof}





\begin{teorema}
Sejam $(\Omega,\F,\P)$ um espaço de probabilidade
e $X_1,\ldots,X_n$ variáveis aleatórias {\bf simples}. Então
\begin{itemize}
	\item[a)]
	A $\sigma$-álgebra $\sigma(X_1,\ldots,X_n)$ 
	consiste exatamente dos conjuntos da forma
	\[
		\{(X_1,\ldots,X_n)\in H\} 
		\equiv
		\{\omega\in\Omega; (X_1(\omega),\ldots,X_n(\omega))\in H\},
	\]
	onde $H\subset \R^n$. 
	Observamos que nesta representação, 
	podemos considerar apenas $H$'s finitos.
	
	\item[b)]
	Uma variável aleatória $Y$ é mensurável segundo 
	$\sigma(X_1,\ldots,X_n)$ se, e somente se,
	$Y=f(X_1,\ldots,X_n)$ para alguma função 
	$f:\R^n\to\R$.	
\end{itemize}
\end{teorema}


\begin{proof}
Prova do item a). Seja $\mathcal{C}$ a coleção de todos os conjuntos da forma 
$\{(X_1,\ldots,X_n)\in H\}$, quando $H$ varia sobre todos os 
subconjuntos de $\R^n$. 
Dentre este subconjuntos estão todos da forma 
$\{(X_1,\ldots,X_n)=(x_1,\ldots,x_n)\}=\cap_{i=1}^n \{X_i=x_i\}$,
que certamente pertencem a $\sigma(X_1,\ldots,X_n)$.
Como $X_1,\ldots,X_n$ são v.a.'s simples, segue que os conjuntos da 
forma $\{(X_1,\ldots,X_n)\in H\}$, com $H\in \R^n$ são uniões 
finitas de elementos de $\sigma(X_1,\ldots,X_n)$.
Portanto $\mathcal{C}\subset \sigma(X_1,\ldots,X_n)$.


Por outro lado, temos que $\mathcal{C}$ é uma $\sigma$-álgebra,
pois: 
\begin{itemize}
\item
$\{(X_1,\ldots,X_n)\in \R^n\}=\Omega$; 

\item
$\{(X_1,\ldots,X_n)\in H\}^c = \{(X_1,\ldots,X_n)\in H^c\}$;

\item 
$
\left\{ (X_1,\ldots,X_n)\in \bigcup_{n\in \N} H_n \right\} 
=
\bigcup_{n\in \N}\{(X_1,\ldots,X_n)\in H_n\}.
$
\end{itemize}
Observamos que cada $X_i$ é mensurável com respeito a
$\mathcal{C}$. De fato, $\{X_i=x\}$ pode ser expresso 
como uma união finita de conjuntos da forma 
$\{(X_1,\ldots,X_n)=(x_1,\ldots,x_n)\}$, com $x_i=x$.
Logo $\sigma(X_1,\ldots,X_n)\subset \mathcal{C}$, 
o que junto com a continência demonstrada acima
implica que $\sigma(X_1,\ldots,X_n)= \mathcal{C}$.

Já que a interceptar $H$ com a imagem do 
vetor $(X_1,\ldots,X_n)$ (que é composta por um número 
finito de pontos de $\R^n$) não modifica a coleção 
$\{(X_1,\ldots,X_n)\in H\}$, podemos considerar 
que $H$ é finto.  



Prova do item b). Suponha, para todo $\omega\in\Omega$, que
$Y(\omega)=f(X_1(\omega),\ldots,X_n(\omega))$ para
alguma função $f:\R^n\to\R$. 
Já $\{Y=y\}$ pode ser escrito na forma $\{(X_1,\ldots,X_n)\in H\}$,
onde $H=\{(x_1,\ldots,x_n)\in \R^n ;\ f(x_1,\ldots,x_n)=y\}$
segue que $Y$ é mensurável segundo $\sigma(X_1,\ldots,X_n)$.


Reciprocamente, assuma que $Y$ é $\sigma(X_1,\ldots,X_n)$-mensurável.
Seja $y_1,\ldots,y_r$ todos os valores que a variável aleatória 
$Y$ assume. Pelo item a) podemos afirmar que existem 
$H_1,\ldots,H_r\subset \R^n$ tais que 
	\[
	\{\omega\in\Omega;\ Y(\omega)=y_i\}
	=
	\{\omega\in\Omega;\ (X_1(\omega),\ldots,X_n(\omega))\in H_i\}.
	\]
Agora defina a seguinte função $f:\R^n\to\R$
	\[
		f(x_1,\ldots,x_n)=\sum_{i=1}^r y_i 1_{H_i}(x_1,\ldots,x_n).
	\] 
Embora {\it a priori} não seja sempre verdadeiro que $H_1,\ldots,H_r$
são mutuamente disjuntos, temos que se 
$(X_1(\omega),\ldots,X_n(\omega))\in H_i\cap H_j$,
então $Y(\omega)=y_i$ e $Y(\omega)=y_j$ o que é impossível.
Portanto cada ponto da forma $(X_1(\omega),\ldots,X_n(\omega))$
pertence a exatamente um dos conjuntos $H_i$ e deste fato segue 
que $f(X_1(\omega),\ldots,X_n(\omega)=Y(\omega)$.
\end{proof}






















\begin{teorema}\label{teo-f(X1,...,X_n)-sigma(X1,...,X_n)}
	Sejam $(\Omega,\F,\P)$ um espaço de probabilidade,
	$X_1,\ldots,X_n$ variáveis aleatórias, então 
\begin{itemize}
	\item[a)]
	A $\sigma$-álgebra $\sigma(X_1,\ldots,X_n)$ 
	consiste exatamente dos conjuntos da forma
	\[
		\{(X_1,\ldots,X_n)\in E\} 
		\equiv
		\{\omega\in\Omega; (X_1(\omega),\ldots,X_n(\omega))\in E\},
	\]
	onde $E\in \mathscr{B}(\R^n)$.
	
	\item[b)]
	Uma variável aleatória $Y$ é mensurável segundo 
	$\sigma(X_1,\ldots,X_n)$ se, e somente se,
	$Y=f(X_1,\ldots,X_n)$ para alguma função 
	$f:\R^n\to\R$ borel mensurável,
	isto é, $f^{-1}(B)\in\mathscr{B}(\R^n)$ para todo
	$B\in\mathscr{B}(\R)$.
\end{itemize}
\end{teorema}


\begin{proof}
Prova do item a). 
A coleção $\mathcal{C}$ de todos os subconjuntos 
de $\Omega$ da forma 
$\{(X_1,\ldots,X_n)\in E\}$, onde $E\in\mathscr{B}(\R^n)$
é uma $\sigma$-álgebra. 
Pela Proposição \ref{prop-caracterizacao-vetor-aleatorio}
temos que a aplicação $\omega\mapsto (X_1(\omega),\ldots,X_n(\omega))$
é mensurável segundo a $\sigma$-álgebra $\sigma(X_1,\ldots,X_n)$.
Logo $\mathcal{C}\subset \sigma(X_1,\ldots,X_n)$. 
Como $\omega\mapsto (X_1(\omega),\ldots,X_n(\omega))$ é claramente 
mensurável segundo $\mathcal{C}$ segue que 
$\sigma(X_1,\ldots,X_n)\subset\mathcal{C}$ e portanto o 
item a) está provado.


Prova do item b).
Se $Y=f(X_1,\ldots,X_n)$ então $Y=f\circ \mathbf{X}$,
onde $\mathbf{X}(\omega)=(X_1(\omega),\ldots,X_n(\omega))$.
Logo para todo $B\in\mathscr{B}(\R)$ temos que 
$Y^{-1}(B)= \mathbf{X}^{-1}(f^{-1}(B))= \mathbf{X}^{-1}(E)$
que pelo item a) implica que $Y$ 
é $\sigma(X_1,\ldots,X_n)$-mensurável.


Para provar a recíproca, suponha inicialmente 
que $Y$ é uma variável aleatória simples assumindo 
os valores $y_1,\ldots,y_m$. Por hipótese temos que 
$
A_i
=
\{\omega\in\Omega: Y(\omega)=y_i\}
\in 
\sigma(X_1,\ldots,X_n).
$
Pelo item a) temos que $A_i=\{(X_1,\ldots,X_n)\in E_i\}$ 
para algum boreliano $E_i$ de $\R^n$. Defina a função 
$f:\R^n\to\R$ dada por 
	\[
	f(x_1,\ldots,x_n)=\sum_{i=1}^r y_i 1_{E_i}(x_1,\ldots,x_n).
	\]
Note que $f$ é Borel mensurável. 
Já que os conjuntos $A_i$'s são disjuntos nenhum ponto 
de $\R^n$ da forma $\mathbf{X}(\omega)$ pertence a mais 
de um $E_i$ 
(observe que não estamos afirmando que os conjuntos 
$E_i$'s são mutuamente disjuntos)
e portanto temos $f(\mathbf{X}(\omega))=Y(\omega)$.


Para provar o teorema no caso geral lembramos que 
se $Y$ é uma v.a. então 
existe uma sequência de v.a.'s $\{Y_n\}$ simples
tal que $Y_n(\omega)\to Y(\omega)$, 
para todo $\omega\in\Omega$
(para provar este fato basta aplicar 
o Teorema \ref{teo:aproximacao-monotona-por-func-simples}
às partes positiva e negativa de $Y$).
Para cada $n\in\N$ segue do argumento acima que 
existe $f_n:\R^n\to\R$ tal que $Y_n=f_n(\mathbf{X})$.
Seja $M\subset \R^n$ o conjunto dos pontos $x\in\R^n$ 
tais que a sequência $\{f_n(x)\}$ converge.
Uma vez que 
 \[
 	M
 	=
 	\left\{
 		x\in\R^n:\ \limsup_{n\to\infty}f_n(x)=\liminf_{n\to\infty}f_n(x)
 	\right\}\in \mathscr{B}(\R^n).
 \]
a seguinte função  
	\[
	f(x_1,\ldots,x_n)
	=
	\begin{cases}
	\displaystyle
	\lim_{n\to\infty} f_n(x_1,\ldots,x_n),& \text{se}\ (x_1,\ldots,x_n)\in M;
	\\[0.3cm]
	0,&\text{se}\ (x_1,\ldots,x_n)\in \R^n\setminus M.
	\end{cases}
	\]
é Borel mensurável pois $\lim_{n\to\infty} f_n1_{M}$ e $f_n1_{M}$ são 
Borel mensuráveis.
Observe que para cada $\omega\in\Omega$ temos que 
$Y(\omega)=\lim_{n\to\infty} f_n(\mathbf{X}(\omega))$.
Logo $\mathbf{X}(\omega)\in M$ e portanto temos finalmente
que 
	\[
	Y(\omega) 
	=
	\lim_{n\to\infty} f_n(\mathbf{X}(\omega))
	=
	f(\mathbf{X}(\omega)).
	\] 
\end{proof}





















\section{Lei Zero-Um de Komolgorov}

Seja $(\Omega,\F,\P)$ um espaço de probabilidade e 
$\{X_n\}$ uma sequência de v.a.'s. 
Para cada $n\in\N$ defina
	\[
		\F_n' = \sigma( X_{n+1},X_{n+2},\ldots).
	\] 
A $\sigma$-álgebra caudal $\mathcal{T}$
é definida como sendo 
	\[
		\mathcal{T}
		=
		\bigcap_{n=1}^{\infty}
		\F_n'
		=
		\lim_{n\to\infty} \sigma(X_n,X_{n+1},\ldots).
	\]
Um evento $A\in\mathcal{T}$ é chamado de evento caudal.
Uma variável aleatória mensurável com respeito a $\mathcal{T}$
é chamada de variável aleatória caudal. 
No que segue apresentamos
alguns exemplos de eventos e v.a.'s caudais.

\begin{exemplo}
	Sejam $(\Omega,\F,\P)$ um espaço de probabilidade e
	$\{X_n\}$ uma sequência de v.a.'s. Então o evento 
		\[
			\left\{
			\omega\in \Omega: \sum_{n=1}^{\infty} X_n(\omega)<\infty
			\right\}
		\]
	é um evento caudal.
	Para prova este fato primeiro observamos 
	que para todo $m\in\N$ temos a seguinte 
	igualdade de conjuntos 
		\[
			\left\{
			\omega\in \Omega: \sum_{n=1}^{\infty} X_n(\omega)<\infty
			\right\}
			=
			\left\{
			\omega\in \Omega: \sum_{n=m}^{\infty} X_n(\omega)<\infty
			\right\}.
		\]
	Fixado $m\in \N$ temos que 
		\[
		\sum_{n=m}^{\infty} X_n(\omega)
		=
		\lim_{k\to\infty} \sum_{n=m}^{m+k} X_n(\omega).
		\]
	Pelo Teorema \ref{teo-f(X1,...,X_n)-sigma(X1,...,X_n)}
	para qualquer $m\in\N$ 
	temos que $\sum_{n=m}^{m+k} X_n \in \sigma(X_{m},\ldots,X_{m+k})
	\subset \sigma(X_{m},X_{m+1},\ldots)$. 
	Logo
	$
	\lim_{k\to\infty}\sum_{n=m}^{m+k} X_n
	\in
	\sigma(X_{m},X_{m+1},\ldots),
	$
	para todo $m\in \N$. 
	Daí segue que 
		\[
			\left\{
			\omega\in \Omega: \sum_{n=1}^{\infty} X_n(\omega)<\infty
			\right\}
			\in
			\mathcal{F}_m'
			\qquad
			\forall m\in\N.			
		\]
	O que implica imediatamente pela definição que o evento acima
	é um evento caudal.
\end{exemplo}


\begin{exemplo}
	Sejam $(\Omega,\F,\P)$ um espaço de probabilidade e
	$\{X_n\}$ uma sequência de v.a.'s. Então as seguintes
	v.a.'s são mensuráveis segundo a $\sigma$-álgebra caudal. 
		\begin{itemize}
			\item 
			$\displaystyle \limsup_{n\to\infty} X_n$.
			
			\item
			$\displaystyle \liminf_{n\to\infty} X_n$.
		\end{itemize}		 

A prova de ambos os itens são semelhantes, portanto vamos 
mostrar apenas o argumento no primeiro caso.  
Pela definição de $\limsup$ temos que 
		\[
			\limsup_{n\to\infty} X_n
			=
			\inf_{n\geq 1} \sup_{k\geq n} X_k.
		\]
Para todo $n\geq 2$ fixado, 
afirmamos que $\sup_{k\geq n} X_k\in \mathcal{F}_{n-1}'$.
De fato,  
\[
	\sup_{k\geq n} X_k  
	= 
	\lim_{m\to\infty} \sup_{n\leq k\leq m} X_k.
\]
Pelo Teorema \ref{teo-f(X1,...,X_n)-sigma(X1,...,X_n)} 
temos que  
$\sup_{n\leq k\leq m} X_k
\in 
\sigma(X_n,X_{n+1},\ldots,X_{m})
\subset
\mathcal{F}_{n-1}'$.
Tomando o limite quando $m\to\infty$ a afirmação está provada. 
Defina 
	\[
		Y_{n} = \sup_{k\geq n} X_k.
	\]
Pela afirmação acima temos que $Y_n$ é $\mathcal{F}_{n-1}'$ mensurável.
Já que  $Y_{n+r}\leq Y_n$, para todo $r\in\N$ temos que 
$\inf_{n} Y_{n} = \inf_{n} Y_{n+r}$. Escrevendo o ínfimo 
a direita novamente como limite é fácil ver que 
$\inf_{n} Y_{n+r}\in \mathcal{F}_{n+r-1}'$ para todo 
$r\in\N$. Portanto 
	\[
		\inf_{n} Y_{n}\in 
		\bigcap_{n=1}^{\infty}
		\mathcal{F}_{n}'.
	\]
\end{exemplo}







\begin{exemplo}
	Seja $(\Omega,\F,\P)$ um espaço de probabilidade e
	$\{X_n\}$ uma sequência de v.a.'s. Então 
		\[
			\left\{
				\omega\in \Omega: 
				\exists \lim_{n\to\infty} X_n
			\right\}
			\in
			\mathcal{T}.
		\]
	Para provar este fato vamos usar o exemplo anterior.
	Pela definição de limite temos que 
	\begin{align*}
		\left\{
			\omega\in \Omega: 
			\exists \lim_{n\to\infty} X_n
		\right\}^c
		&=
		\left\{
			\omega\in \Omega: 
			\liminf_{n\to\infty} X_n 
			< 
			\limsup_{n\to\infty} X_n
		\right\}
		\\[0.3cm]
		&\hspace*{-1cm}=
		\bigcup_{r\in \Q}
		\left(		
		\left\{
			\omega\in \Omega: 
			\liminf_{n\to\infty} X_n 
			\leq r 
		\right\}
		\cap
		\left\{
			\omega\in \Omega: 
			\limsup_{n\to\infty} X_n 
			> r 
		\right\}
		\right).
	\end{align*}
Como ambas v.a.'s $\limsup X_n$ e $\liminf X_n$ são $\mathcal{T}$-mensuráveis
segue que o lado esquerdo da igualdade acima é um evento pertencente 
a $\sigma$-álgebra caudal e portanto seu complementar, o que desejamos,
também é um evento caudal.
\end{exemplo}





\begin{exemplo}
	Sejam $(\Omega,\F,\P)$ um espaço de probabilidade,
	$\{X_n\}$ uma sequência de v.a.'s. e $S_n=X_1+\ldots+X_n$.
	Então 
	\[
		\left\{
			\omega\in \Omega: 
			\lim_{n\to\infty} \frac{S_n}{n}=0
		\right\}
		\in
		\mathcal{T}.
	\]
	De fato, para qualquer $m\in\N$ temos que
		\[
			\limsup_{n\to\infty} \frac{S_n}{n}
			=
			\limsup_{n\to\infty} \frac{\sum_{i=1}^n X_i}{n}
			=
			\limsup_{n\to\infty} \frac{\sum_{i=m+1}^n X_i}{n}
			\in
			\mathcal{F}_{m}'.
		\]
	Segue dos exemplos anteriores que o conjunto de pontos 
	de $\Omega$ para o qual existe $\lim_{n\to\infty} S_n/n$ 
	é mensurável segundo $\mathcal{T}$. Portanto
	\[
		\left\{
			\omega\in \Omega: 
			\lim_{n\to\infty} \frac{S_n}{n}=0
		\right\}
		=
		\left\{
			\omega\in \Omega: 
			\limsup_{n\to\infty} \frac{S_n}{n}=0
		\right\}
		\cap
		\left\{
			\omega\in \Omega: 
			\exists \lim_{n\to\infty} \frac{S_n}{n}
		\right\}
		\in
		\mathcal{T}.
	\]
\end{exemplo}





Seja $(\Omega,\F,\P)$ um espaço de probabilidade.
Chamamos uma sub-$\sigma$-álgebra $\mathcal{B}$ de
{\bf quase trivial} se todos seus eventos têm probabilidade
zero ou um. Um exemplo de sub-$\sigma$-álgebra 
quase trivial é $\mathcal{B}=\{\emptyset,\Omega\}$.




\begin{lema}[$\sigma$-Álgebras quase-triviais]
\label{lema-sigma-algebras-quase-triviais}
Sejam $(\Omega,\F,\P)$ um espaço de probabilidade,
$\mathcal{B}$ uma sub-$\sigma$-álgebra de $\F$
quase-trivial e $X$ uma v.a. $\mathcal{B}$-mensurável.
Então existe uma constante $c\in\R$ tal que 
$\P(X=c)=1$.
\end{lema}



\begin{proof}
Seja $F(x)=\P(X\leq x)$. Como já sabemos $F$ é uma 
função monótona não-decrescente e já que 
$\{X\leq x\}\in\sigma(X)\subset \mathcal{B}$ 
temos que $F(x)\in\{0,1\}$, para qualquer 
que seja $x\in\R$. Seja $c =\sup\{x\in\R: F(x)=0\}$.
Pelas propriedades de função distribuição segue que 
$c<\infty$. Pela definição de $c$ e pelo fato de 
$F(x)$ assumir apenas os valores $0$ ou $1$
temos para todo $\varepsilon>0$ que
$F(c+\varepsilon)=\P(X\leq c+\varepsilon)=1$.
Tomando o limite quando $\varepsilon\to 0$ segue da 
continuidade a direita de $F$ que $F(c)=1$.
Já que $1=F(c)= \P(X<c)+\P(X=c)$ e $P(X<c)=0$
temos que $\P(X=c)=1$.
\end{proof}















\begin{teorema}
[Lei Zero-Um de Kolmogorov]
\label{teo-lei-zero-um-kolmogorov}
Sejam $(\Omega,\F,\P)$ um espaço de probabilidade
e $\{X_n\}$ é uma sequência de v.a.'s independentes
com $\sigma$-álgebra caudal $\mathcal{T}$, então 
para todo evento $E\in\mathcal{T}$ temos que 
$\P(E)=0$ ou $1$, isto é, $\mathcal{T}$ 
é uma $\sigma$-álgebra quase-trivial.
\end{teorema}


\begin{proof}
A ideia da prova é mostrar que se $E\in\mathcal{T}$
então $E$ é independente de si mesmo, ou seja,
$\P(E)=\P(E\cap E)=\P(E)\P(E)$. Desta igualdade 
temos que $\P(E)=\P(E)^2$ e logo $\P(E)$ é solução 
da equação quadrática $x=x^2$ e portanto só pode 
assumir os valores zero ou um.   


Para cada $n\in\N$ defina $\mathcal{F}_n=\sigma(X_1,\ldots,X_n)$.
Como $\mathcal{F}_n\subset \mathcal{F}_{n+1}$ temos que  
\[
	\mathcal{F}_{\infty}
	\equiv
	\sigma(X_1,X_2,\ldots)
	=
	\sigma\left( \bigcup_{n\in\N} \sigma(X_n) \right)
	=
	\sigma\left( \bigcup_{n\in\N} \mathcal{F}_n \right).
\]
Observamos que
\begin{equation}\label{eq-aux1-teo-lei-zero-um-kolmogorov}
	E\in \mathcal{T}\subset \mathcal{F}_n'
	=
	\sigma(X_{n+1},X_{n+2},\ldots)
	\subset 
	\mathcal{F}_{\infty}.
\end{equation}
Em particular, $E\in \mathcal{F}_n',\ \forall n\in\N$.
Já que a sequência $\{X_n\}$ é independente temos que
as $\sigma$-álgebras $\mathcal{F}_n$ e $\mathcal{F}_n'$
são claramente independentes e assim $E$ é independente
de qualquer evento em $\mathcal{F}_n$ para qualquer 
que seja $n\in\N$. Isto implica que $E$ é independente
de $\cup_{n\in\N} \mathcal{F}_n$.
Considere as seguintes coleções $\mathcal{C}_1=\{E\}$ 
e $\mathcal{C}_2=\cup_{n\in\N} \mathcal{F}_n$. Claramente
$\mathcal{C}_1$ e $\mathcal{C}_2$ são $\pi$-sistemas e 
além do mais este $\pi$-sistemas são independentes,
então pelo Teorema \ref{teo-criterio-basico-independencia}
segue que as $\sigma$-álgebras
	\[
	\sigma(\mathcal{C}_1)=\{\emptyset,E,E^c,\Omega\}
	\qquad
	\text{e}
	\qquad
	\sigma(\mathcal{C}_2)
		=\sigma\left( \bigcup_{n\in\N} \mathcal{F}_n \right)
		=\mathcal{F}_{\infty}
	\]
são independentes. Da primeira 
igualdade acima temos que 
$E\in \sigma(\mathcal{C}_1)$ e de 
\eqref{eq-aux1-teo-lei-zero-um-kolmogorov} 
segue que 
$E\in \mathcal{F}_{\infty}=\sigma(\mathcal{C}_2)$, 
portanto $E$ é independente de si mesmo.
\end{proof}













\begin{corolario}
Sejam $(\Omega,\F,\P)$ um espaço de probabilidade e 
$\{X_n\}$ é uma sequência de v.a.'s independentes, 
então 
	\begin{itemize}
		\item[a)]
		O evento $\{ \sum_{n\in\N} X_n\ \text{converge}\}$ tem probabilidade
		zero ou um.		
		
		\item[b)]
		As v.a.'s $\liminf X_n$ e $\limsup X_n$ são contantes quase certamente.
		
		\item[c)]
		O evento $\{S_n/n\to 0\}$ tem probabilidade zero ou um.		
		
	\end{itemize}
\end{corolario}


\begin{proof}
Basta usar os exemplos desta seção para concluir que os eventos
em a) e c) são caudais e para b) usar também os exemplos desta
seção  junto com o Lema \ref{lema-sigma-algebras-quase-triviais}. 
\end{proof}



















\section{A medida Produto em $\{0,1,2,\ldots,s\}^{\mathbb{N}}$}

Seja $s\geq 1$ um número natural fixado. 
Denote por $S=\{0,1,2,\ldots,s\}$, 
o conjunto dos primeiros $s+1$ inteiros não negativos 
e seja $\Omega = S^{\mathbb{N}}$, isto é,
o produto cartesiano de infinitas cópias de $S$. 
Este produto cartesiano pode ser definido 
de várias formas. Por simplicidade vamos 
adotar nesta seção a seguinte definição 
\[
S^{\mathbb{N}}
\equiv 
\{
	(\omega_1,\omega_2,\ldots)
	:\omega_i\in S\ \forall \ i\in \mathbb{N}
\}.
\]
Note que podemos pensar em $\Omega$ como 
o conjunto de todas as sequências tomando 
valores em $S$. O conjunto $\Omega$ é 
muitas vezes chamado na literatura de 
espaço simbólico.

Será de grande utilidade considerar algumas
funções de ``projeção''. 
Tais funções são definidas de maneira mais precisa
da seguinte forma: para cada inteiro $n\geq 1$
\[
	\pi_n: S^{\mathbb{N}}\to S
	\ \ \
	\text{é definida por}\ \ \
	\pi_n\big( (\omega_1,\omega_2,\ldots)  \big)
	=
	\omega_n.
\]
As aplicações $\pi_n$'s são chamadas de 
{\bf funções coordenadas} ou {\bf projeções naturais}.
De maneira usual denotamos por $S^n$ o produto cartesiano
de $n$ cópias de $S$, isto é, $S^n=S\times S\times\ldots \times S$.

Um cilindro $n$-dimensional em $\Omega=S^{\mathbb{N}}$ é um conjunto 
$\mathcal{C}$ da forma 
\[
\mathcal{C}
=
\{
\omega\in \Omega:
(\pi_1(\omega),\ldots,\pi_n(\omega)) \in H
\},
\ \ 
\text{onde}\ \ H\subset S^n.
\]
Note que $\mathcal{C}$ é um conjunto não-vazio 
se $H$ é não vazio.
Um conjunto $\mathcal{C}$ é chamado de 
um cilindro finito dimensional se $\mathcal{C}$
é um cilindro $n$-dimensional para algum $n\in\mathbb{N}$. 


\begin{lema}
A coleção $\mathscr{C}_0$ de todos os cilindros finito dimensionais de 
$\Omega = S^{\mathbb{N}}$ é uma álgebra de conjuntos.
\end{lema}

\begin{proof}
Para ver que $\Omega$ e $\emptyset$ pertencem a 
$\mathscr{C}_0$ basta observar que 
\[
\Omega 
= 
\{\omega\in \Omega: \pi_1(\omega)\in S\}
\in
\mathscr{C}_0
%
\quad
\text{e}
\quad
%
\emptyset
=
\{\omega\in \Omega: \pi_1(\omega)\in \emptyset\}
\in
\mathscr{C}_0.
\]
%
Seja $\mathcal{C}\in \mathscr{C}_0$ um cilindro finito dimensional
da forma 
$$
\mathcal{C}
=
\{
\omega\in \Omega:
(\pi_1(\omega), \ldots,\pi_n(\omega)) \in H
\},
$$
onde
$H\subset S^n$ para algum $n\in\mathbb{N}$.
Vamos mostrar que $\Omega\setminus \mathcal{C}\in \mathscr{C}_0$.
De fato, temos que 
$$
\Omega\setminus \mathcal{C}
=
\{
\omega\in \Omega:
(\pi_1(\omega), \ldots,\pi_n(\omega)) \in S^n\setminus H
\}.
$$
Resta mostrar que a coleção $\mathscr{C}_0$ é fechada 
para uniões finitas. Para fazer isto é suficiente mostrar
que dados quaisquer dois cilindros finito dimensionais 
$\mathcal{C},\mathcal{D}\in\mathscr{C}_0$ temos que 
$\mathcal{C}\cup \mathcal{D}\in\mathscr{C}_0$.
Suponha que $\mathcal{C}$ é um cilindro $n$-dimensional
da forma 
$
\mathcal{C}
=
\{\omega\in \Omega: (\pi_1(\omega),\ldots,\pi_n(\omega))\in H\},
$
onde $H\subset S^n$
e $\mathcal{D}$ é um cilindro $m$-dimensional
da forma 
$
\mathcal{D}
=
\{\omega\in \Omega: (\pi_1(\omega),\ldots,\pi_m(\omega))\in J\},
$
onde $J\subset S^m$.
Sem perda de generalidade, podemos assumir que $m\leq n$.
Vamos considerar primeiro o caso $m=n$. Neste caso 
temos que 
\[
\mathcal{C}\cup\mathcal{D}
=
\{\omega\in \Omega: (\pi_1(\omega),\ldots,\pi_n(\omega))\in G\cup J\}.
\] 
Caso $m<n$, podemos olhar para o cilindro $\mathcal{D}$
como um cilindro $n$-dimensional, pois 
\[
\mathcal{D}
=
\mathcal{D}'
\equiv 
\{
\omega\in \Omega: 
(\pi_1(\omega),\ldots,\pi_m(\omega),\ldots,\pi_n(\omega))
\in J\times S^{n-m}
\}.
\] 
Usando a igualdade acima e o caso anterior segue que 
$\mathcal{C}\cup\mathcal{D}\in\mathscr{C}_0$. 
\end{proof}



\begin{definicao}
Sejam $s\geq 1$ um inteiro positivo fixado, $S=\{0,1,2,\ldots,s\}$ 
e $\Omega=S^{\mathbb{N}}$. A $\sigma$-álgebra gerada pela coleção
dos cilindros finito dimensionais $\F =\sigma(\mathscr{C}_0)$ 
é chamada de $\sigma$-álgebra gerada pelos cilindros.
\end{definicao}

Em seguida, vamos mostrar como construir diversas medidas 
de probabilidade definidas em 
$(\Omega,\F)\equiv (S^{\mathbb{N}},\sigma(\mathscr{C}_0))$ 
tendo várias
propriedades interessantes. A ideia é definir estas medidas
inicialmente na coleção dos cilindros finito dimensionais 
e em seguida usar o Teorema da Extensão de Carathéodory.

\begin{proposicao}
Sejam $s\geq 1$ inteiro positivo fixado,
$\mathscr{C}_0$ a coleção dos cilindros finito 
dimensionais de $\Omega=S^{\mathbb{N}}$
e $(p_1,\ldots,p_s)\in [0,1]^s$ tais que 
$p_1+\ldots +p_s=1$. 
Para cada $\mathcal{C}\in\mathscr{C}_0$
da forma 
$
\mathcal{C}
=
\{
\omega\in \Omega:
(\pi_1(\omega), \ldots,\pi_n(\omega)) \in H
\},
$
onde $H\subset S^n$ defina 
\end{proposicao}