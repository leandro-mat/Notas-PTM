\chapter[Aula 8]{Lei Zero-Um de Komolgorov}
\chaptermark{}

\section{Uma Propriedade de Sequências de v.a.'s Exponenciais Independentes}


Nesta seção vamos assumir que $\{X_n\}$ é uma sequência 
de v.a.'s iid com distribuição exponencial de parâmetro 1,
isto é, $X_n\geq 0$ e  
\[
	\P(X_n>x)=e^{-x}.
\] 

O objetivo desta seção é provar que 
%
	\begin{equation}\label{exemplo-seq-exp-par-1-indep-cresce-logn}
		\P
		\left(
			\limsup_{n\to\infty}
			\frac{X_n}{\log n} 
			=1
		\right)	
		=1.
	\end{equation}

Este resultado é as vezes considerado surpreendente.
Esta tendência está associada ao erro comum que alguns
matemáticos cometem pensando que sequências iid 
são ``moralmente constantes'' e portanto a divisão 
por $\log n$ deveria fazer com que o limite fosse
zero. 
Entretanto muito frequentemente a sequência $\{X_n\}$
toma valores que vão aumentando com o índice e 
este aumento é aproximadamente dado por $\log n$.

Para provar este resultado vamos precisar de um fato 
bastante simples que se $\{A_n\}$ é uma sequência 
de eventos em um espaço de probabilidade $(\Omega,\F,\P)$
tal que $\P(A_n)=1$ então $\P(\cap_{n=1}^{\infty} A_n)=1$,
veja Lista 1 de exercícios.


\medskip

\noindent
{\bf Prova de \eqref{exemplo-seq-exp-par-1-indep-cresce-logn} }.
Se $\omega\in \Omega$ é tal que 
	\[
		\limsup_{n\to\infty}
		\frac{X_n(\omega)}{\log n} 
		=1,
	\]
isto significa que 
\begin{itemize}
	\item[a)]
	para todo $\varepsilon>0$ temos que se $n\geq n_0\equiv n_0(\omega)$ então
	$\displaystyle\frac{X_n(\omega)}{\log n}\leq 1+\varepsilon$.
	
	
	\item[b)]
	para todo $\varepsilon>0$ para infinitos valores de $n$ 
	temos que 	
	$1-\varepsilon<\displaystyle\frac{X_n(\omega)}{\log n}$.
		
\end{itemize}




Observe que o item a) diz que não existe nenhuma 
subsequência com limite maior que $1+\varepsilon$ enquanto
b) diz que sempre existe uma subsequência limitada inferiormente
por $1-\varepsilon$. 




Para qualquer sequência $\varepsilon_k\downarrow 0$, 
note que temos a seguinte igualdade
de conjuntos
\begin{align*}
\left\{ \limsup_{n\to\infty} \frac{X_n}{\log n} =1  \right\}
&
\\[0.3cm]
&
\hspace*{-2cm}
=\bigcap_{k=1}^{\infty}
\left[ 
	\liminf_{n\to\infty} 
		\left\{ 
			\frac{X_n}{\log n} \leq 1+\varepsilon_k 
		\right\}  
\right]
\quad
\bigcap
\quad
\bigcap_{k=1}^{\infty}
\left[ 
	\limsup_{n\to\infty} 
		\left\{ 
			\frac{X_n}{\log n} > 1-\varepsilon_k  
		\right\}  
\right].
\end{align*}
%
%
%
%
Para provar que o lado direito acima tem probabilidade
um, basta mostrar que cada um dos eventos 
entre colchetes tem probabilidade um. 

Para todo $k\in\N$ fixado temos 
\begin{align*}
\sum_{n=1}^{\infty} 
\P\left(  
	\frac{X_n}{\log n} > 1-\varepsilon_k 
\right)
&=
\sum_{n=1}^{\infty} 
\P\left(  
	X_n > (1-\varepsilon_k)\log n  
\right)
\\
&=
\sum_{n=1}^{\infty} 
\exp\left(  
	 -(1-\varepsilon_k)\log n  
\right)
\\
&=
\sum_{n=1}^{\infty} 
\frac{1}{n^{1-\varepsilon_k}}
\\
&=
\infty,
\end{align*}
onde na última desigualdade usamos que $k$ 
está fixado. Assim segue da Lei Zero-Um de Borel
que 
	\[
		\P\left( 
			\limsup_{n\to\infty} 
			\left\{ 
				\frac{X_n}{\log n} > 1-\varepsilon_k  
			\right\}  
		\right)
		=
		1.
	\]

De maneira análoga podemos verificar que
\begin{align*}
\sum_{n=1}^{\infty} 
\P\left(  
	\frac{X_n}{\log n} > 1+\varepsilon_k 
\right)
&=
\sum_{n=1}^{\infty} 
\P\left(  
	X_n > (1+\varepsilon_k)\log n  
\right)
\\
&=
\sum_{n=1}^{\infty} 
\exp\left(  
	 -(1+\varepsilon_k)\log n  
\right)
\\
&=
\sum_{n=1}^{\infty} 
\frac{1}{n^{1+\varepsilon_k}}
\\
&<
\infty.
\end{align*}
Aplicando novamente a Lei Zero-Um de Borel
temos que 
	\[
		\P\left( 
			\limsup_{n\to\infty} 
			\left\{ 
				\frac{X_n}{\log n} > 1+\varepsilon_k  
			\right\}  
		\right)
		=
		0.
	\]
Tomando complementar obtemos 
	\[
		\P\left( 
			\liminf_{n\to\infty} 
			\left\{ 
				\frac{X_n}{\log n} \leq 1+\varepsilon_k  
			\right\}  
		\right)
		=1.
	\]
O que encerra a demonstração.















\section{Lei Zero-Um de Komolgorov}

Seja $(\Omega,\F,\P)$ um espaço de probabilidade e 
$\{X_n\}$ uma sequência de v.a.'s e defina
	\[
		\F_n' = \sigma( X_{n+1},X_{n+2},\ldots)
	\] 