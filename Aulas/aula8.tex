\chapter[Aula 8]{Lei Zero-Um de Komolgorov}
\chaptermark{}

\section{Uma Propriedade de Sequências de v.a.'s Exponenciais Independentes}


Nesta seção vamos assumir que $\{X_n\}$ é uma sequência 
de v.a.'s iid com distribuição exponencial de parâmetro 1,
isto é, $X_n\geq 0$ e  
\[
	\P(X_n>x)=e^{-x}.
\] 

O objetivo desta seção é provar que 
%
	\[
		\P
		\left(
			\limsup_{n\to\infty}
			\frac{X_n}{\log n} 
			=1
		\right)	
		=1.
	\]

Este resultado é as vezes considerado surpreendente.
Esta tendência está associada ao erro comum que alguns
matemáticos cometem pensando que sequências iid 
são ``moralmente constantes'' e portanto a divisão 
por $\log n$ deveria fazer com que o limite fosse
zero. 
Entretanto muito frequentemente a sequência $\{X_n\}$
toma valores que vão aumentando com o índice e 
este aumento é aproximadamente dado por $\log n$.

Para provar este resultado vamos precisar de um fato 
bastante simples que se $\{A_n\}$ é uma sequência 
de eventos em um espaço de probabilidade $(\Omega,\F,\P)$
tal que $\P(A_n)=1$ então $\P(\cap_{n=1}^{\infty} A_n)=1$,
veja Lista 1 de exercícios. 
 