\chapter[Aula 8]{A Integral de Lebesgue}
\chaptermark{}

\section{A Integral}

A principal diferença entre a integral de Riemann e a integral
de Lebesgue é que na integral de Riemann é feita uma 
partição do dominío da função, enquanto que na construção da
 integral de Lebesgue é feita uma partição da imagem da função.
Esse novo  "aproach" permite construir um conceito de integral muito
mais geral e com propriedades muito mais fortes  que a integral de Riemman, 
como veremos. Também discutiremos quando esses dois conceitos 
de integral coincidem.

O objeto que desenrola um papel fundamental na 
contrução da integral de Lebesgue é o conceito de função simples: 
\begin{definicao}
Uma função real é dita uma função simples
 quando sua imagem possue apenas um número
  finito de elementos.
\end{definicao}
Uma função real  mensurável simples $f\in \mathcal{M}^{+}(\Omega,\mathcal{F})$ pode ser escrita na forma 
\begin{equation}\label{Func. Simp.}
f=\sum_{j=1}^{n}a_j 1_{E_j}
\end{equation}
onde $1_{E_j}$ é a função característica de um conjunto $E_j\in \mathcal{F}$. 
 Não é difícil ver que existem várias maneiras de escrever uma mesma função
  simples na forma de (\ref{Func. Simp.}). Entretanto existe uma maneira de
  representar  que será nossa preferida, que será chamada a
   \emph{representação standart } de $f$, 
   que é caracterizada do seguinte modo: os $a_j$ 
   devem ser distintos e os $E_j$ devem ser disjuntos,  não vazios e tais que 
   $\bigcup_{j=1}^nE_j=\Omega$. É fácil ver que a representação standart de $f$ é única.