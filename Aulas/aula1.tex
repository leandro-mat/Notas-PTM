\chapter[Aula 1]{Conjuntos e Limite de Sequências de Conjuntos}
\chaptermark{}

  
\section{Teoria Básica de Conjuntos}

Nesta seção são introduzidos alguns fatos básicos da teoria
conjuntos. O ponto de vista adotado aqui é o da 
``teoria ingênua de conjuntos'' e não temos objetivos de discutí-la do ponto 
vista axiomático.


Em geral, vamos trabalhar em um espaço que será denotado 
por $\Omega$. Assim, operações de conjuntos serão sempre 
consideradas com relação a este espaço. 
A coleção de todos os subconjuntos de $\Omega$ será 
denotada por $\mathcal{P}(\Omega)\equiv \{A: A\subset \Omega\}$ e 
será chamado de conjunto das partes de $\Omega$.
Em geral, usaremos as letras maiúsculas $A,B$ e etc$\ldots$ para denotar
um subconjunto arbitrário de $\Omega$. Quando quisermos nos referir
a uma coleção de subconjuntos de $\Omega$ usaremos letras maiúsculas 
caligráficas como $\mathcal{A}, \mathcal{B}$ e etc. Finalmente,
usaremos, na maioria das vezes, a notação $\omega\in\Omega$ 
para denotar um ponto do espaço $\Omega$.



O complementar de um conjunto $A$ ou simplesmente complemento de $A$ 
será denotado por $A^{c}\equiv \{w \in \Omega; w \notin A\}$. 
Se $T$ é um conjunto de índices arbitrários e 
para cada $t\in T$ temos que $A_t\subset \Omega$ então 
a interseção e união dos conjuntos $A_t$'s sobre a coleção
$T$ são dados, respectivamente, por 
\[ 
\bigcap_{t \in T}{A_t}
= 
\{ w \in \Omega:\  \ w \in A_t,\ \forall t \in T\}.
\]
e
\[
\bigcup_{t \in T}{A_t}
= 
\{ w \in \Omega:\ w \in A_t,\ \text{para algum}\ t \in T \}.
\]
%
%
%
Se $A \cap B = \emptyset$, dizemos que $A$ e $B$ são disjuntos.
De maneira mais geral, uma família de conjuntos 
$\{A_t\}_{t\in T}$ é dita mutuamente disjunta (ou dois a dois disjunta) 
se $A_i \cap A_j = \emptyset$ para todo $i,j\in T$ com $i\neq j$. 
Por questão de simplicidade vamos usar 
a notação 
\[
	\bigsqcup_{t \in T} A_t
\]
para indicar a união de uma 
família $\{A_t\}_{t\in T}$ de conjuntos
mutuamente disjunta.

\bigskip
\noindent
\textbf{Observação}. 
Alguns autores usam simplesmente a notação $AB$ 
para indicar a interseção $A\cap B$. 
\bigskip


Vamos usar as notações $A \setminus B$ ou $A-B$
para denotar o conjunto diferença entre $A$ e $B$,
que é definido por $A\setminus B\equiv  A \cap B^c$.
É importante observar que em geral, 
$A - B \neq B - A$.
Outra operação, entre conjuntos, que vamos 
considerar com frequência é 
a diferença simétrica entre dois conjuntos $A$ e $B$. 
Esta operação é definida por 
$A\triangle B = (A \setminus B) \cup (B\setminus A).$
Segue da comutatividade da união, 
que a diferença simétrica é uma operação comutativa,
isto é,  $A\triangle B = B\triangle A$.
Por último, vamos convencionar que 
$\emptyset^c = \Omega$ e $\Omega^c = \emptyset$.




\begin{exercicio}[Associatividade] 
Mostre que para quaisquer subconjuntos $A,B$ e $C$ 
de $\Omega$ valem as seguintes igualdades:
\begin{enumerate}
	\item $(A\cup B)\cup C= A \cup ( B \cup C)$.
	\item $(A\cap B)\cap C= A \cap ( B \cap C)$.
\end{enumerate}
\end{exercicio}






\begin{exercicio}[Leis de de Morgan] 
Sejam $I$ um conjunto arbitrário de índices
e $A_i\subset \Omega$ para todo $i\in I$. Mostre que
as seguintes igualdades são válidas:
\begin{enumerate}
\item 
$
\left( \displaystyle\bigcup_{i \in I}{A_i} \right)^c 
= 
\displaystyle\bigcap_{i \in I}{{A_i}^c}
$.
\vspace*{0.3cm}
\item
$
\left( \displaystyle\bigcap_{i \in I}{A_i} \right)^c 
= 
\displaystyle\bigcup_{i \in I}{A_i^c}
$.
%
\end{enumerate}
\end{exercicio}





\begin{exercicio}[Distributiva] 
Sejam $I$ um conjunto arbitrário de índices, 
$B\subset \Omega$ e $A_i\subset \Omega$ para todo $i\in I$. 
Mostre que as seguintes igualdades são válidas:
%
\begin{enumerate}
\item 
$
B \cap \left( \displaystyle\bigcup_{i \in I}{A_i} \right) 
= 
\displaystyle\bigcup_{i \in I}{(B\cap A_i)} 
$.
%
\vspace*{0.3cm}
\item
$
B \cup \left( \displaystyle\bigcap_{i \in I}{A_i} \right) 
= 
\displaystyle\bigcap_{i \in I}{(B\cup A_i)} 
$.
\end{enumerate}
%
\end{exercicio}







\begin{definicao}[Função Indicadora]\label{def-funcao-indicadora}
	Seja $A \subseteq \Omega$. 
	A função indicadora \index{Função!Indicadora} 
	de $A$ é a função $1_A: \Omega \to \R$ definida por 
	\[
		1_A(w) =
			\begin{cases}
				1, & \text{se}\ w \in A; \\
				0, & \text{caso contrário.}
			\end{cases}
	\]
\end{definicao}








\begin{observacao} 
	Da definição de função indicadora, podemos mostrar facilmente 
	as seguintes relações,
	as quais serão importantes ao longo do texto
	\begin{enumerate}
		\item 
		$1_A \leqslant 1_B \Leftrightarrow A \subseteq B$.

		\item
		$1_{A^c}= 1- 1_A$.
\end{enumerate}
\end{observacao}







\section{Limite de Conjuntos}

Nesta seção queremos introduzir uma noção de 
convergência de conjuntos. 
A definição de convergência que vamos dar
abaixo é baseada em algumas ideias da noção 
de limite de uma sequência de números reais. 
Esta ideia consiste
em definir limites superior e inferior de uma
sequência de conjuntos e dizer que uma sequência 
de conjuntos converge para um determinado conjunto
se os limites superior e inferior coincidem.
De maneira mais precisa, 
seja $\{A_n\}_{n\in\mathbb{N}}$ uma sequência 
arbitrária de subconjuntos de $\Omega$ 
e para cada $n\in\mathbb{N}$ fixado considere
os seguintes conjuntos: 
\[
\inf \limits_{k\geqslant n} A_k 
	\equiv 
	\displaystyle\bigcap_{k=n}^{\infty}{A_k}
\qquad\text{e}\qquad
\sup \limits_{k\geqslant n} A_k 
	\equiv 
	\displaystyle\bigcup_{k=n}^{\infty}{A_k}.
\]
Os conjuntos limite inferior e superior da sequência 
$\{A_n\}_{n\in\mathbb{N}}$ são definidos, respectivamente,
por
\[
	\liminf \limits_{n \to \infty} A_n 
	\equiv 
	\displaystyle\bigcup_ {n\geqslant 1} 
		\left(\displaystyle\bigcap_{k=n}^{\infty}{A_k} \right)
\qquad\text{e}\qquad
	\limsup \limits_{n \to \infty} A_n 
	\equiv 
	\displaystyle\bigcap_ {n\geqslant 1} 
		\left(\displaystyle\bigcup_{k=n}^{\infty}{A_k} \right).
\]






\begin{definicao}[Limite de uma Sequência de Conjuntos]
	Seja $\{A_n\}_{n\in\mathbb{N} }$  uma sequência de conjuntos.
	Se existe um conjunto $A$ tal que 
	\[
		\liminf \limits_{n \to \infty} A_n 
		= 
		A
		=
		\limsup \limits_{n \to \infty} A_n, 
	\]	
	então dizemos que a sequência de conjuntos 
	$\{A_n\}_{n\in\mathbb{N}}$ tem limite $A$  
	e usamos a seguinte notação para indicar este fato 
	$\lim \limits_{n \to \infty} A_n = A$.
\end{definicao}



Abaixo apresentamos um lema que caracteriza os 
conceito de limite inferior e superior em 
termos de funções tomando valores no 
conjunto dos números reais. Esta caracterização 
é também muito útil para que possamos interpretar
de maneira intuitiva ambos conjuntos.  


\begin{lema}
Seja $\{A_n\}_{n\in\mathbb{N}}$ uma sequência de subconjuntos de $\Omega$.
Então as seguintes igualdades são válidas:
\begin{enumerate}
\item 
$\displaystyle\limsup_{n\to\infty} A_n 
= 
\left\{ 
w \in \Omega: \ \sum_{n \geqslant 1} 1_{A_n}(w)= \infty 
\right\}
$.

\vspace*{0.3cm}


\item  
$
\begin{aligned}[t]
\liminf_{n\to\infty} A_n 
&=
	\{ w \in \Omega:
		\ w \in A_n\ \mathrm{para\ todo}\ n\ \mathrm{exceto\ uma\ quantidade\ finita} 
	\} 
\\
&= 
	\left\{w \in \Omega:
		\ \sum_{n\geqslant 1} 1_{A_n^c}(w) < \infty 
	\right\} 
\\[0.2cm]
&= 
	\{w \in \Omega:
		\ w \in A_n, \forall n \geqslant n_0(w) 
	\}. 
\end{aligned} 
$
\end{enumerate}
\end{lema}

\begin{proof}
Prova do item $1$. Suponha que $w \in \limsup A_n$, então 
$w \in \cup_{k \geq n} A_k,
\forall n \in \N$, logo existe $k_n \geqslant n$ tal que $w \in A_{k_n}$. 
Assim
%
	\[
		\sum \limits_{n \geqslant 1} 1_{A_n}(w) 
		\geqslant 
		\sum \limits_{n\geqslant 1} 1_{A_{k_n}}(w) 
		= 
		\infty.
	\]
%
Reciprocamente, 
se 
$ w \in \{ w \in \Omega;\ \sum_{n \geqslant 1} 1_{A_n}(w)= \infty \}$, 
então para infinitos valores de $k$ temos que $w \in A_k$. 
Portanto $w \in \limsup A_n$.

A prova do item $2$ segue diretamente do item $1$. 
\end{proof}



\begin{exercicio} 
Seja $\{A_n\}$ uma sequência de subconjuntos de $\Omega$.
Mostre que as seguintes igualdades são verdadeiras:
%
\begin{enumerate}
\item $\liminf A_n \subseteq \limsup A_n$.
\item $\left( \liminf A_n \right)^c = \limsup A_n^c$. 
\end{enumerate}
\end{exercicio}







\begin{observacao} Este comentário pode ser omitido 
por leitores que nunca fizeram um curso introdutório 
de Probabilidade.
Seja $\{X_n,\ n\geqslant 0\}$ uma sequência de v.a.'s.
Uma das maneiras de mostrar que $X_n \to X$ q.c. 
é provar que   
\[
	\mathbb{P}( |X_n-X| > \epsilon \ \text{infinitas vezes} )=0,
\] 
em outras palavras, 
se denotamos por $A_n= \{ |X_n-X|> \epsilon\}$ 
então basta provar que $\mathbb{P} (\limsup A_n )=0$.
Voltaremos a este critério posteriormente e apresentaremos
sua prova no momento apropriado.
\end{observacao}







\begin{definicao}[Sequências Monótonas de Conjuntos]
Seja $\{A_n\}$ é uma sequência de conjuntos de $\Omega$. 
Dizemos que $\{A_n\}$ é monótona não-decrescente 
se $A_1 \subseteq A_2 \subseteq \ldots$.
Analogamente definimos sequência não-crescente.
Usaremos as notações $A_n \nearrow$ ou $A_n \uparrow$ 
(analogamente $A_n\searrow$ ou $A_n \downarrow$)
para indicar que $A_n$ é uma sequência 
não-decrescente (não-crescente).
\end{definicao}







\begin{proposicao}
 Suponha que $\{A_n\}$ é uma sequência monótona.
 \begin{enumerate}
 \item Se $A_n \nearrow$ então 
 	$\exists \lim \limits_{n \to \infty} A_n
 	= 
 	\displaystyle\bigcup_{n\geqslant 1} {A_n}$.
 	
 \item Se $A_n \searrow$ então 
 	$\exists \lim \limits_{n \to \infty} A_n
 	= 
 	\displaystyle\bigcap_{n\geqslant 1} {A_n}$.
 \end{enumerate}
\end{proposicao}


\begin{proof}
Vamos provar inicialmente o item 1. 
Neste caso queremos mostrar que 
%
\[
	\liminf_{n\to\infty} A_n
	= 
	\limsup_{n\to\infty} A_n
	=
	\displaystyle\bigcup_{n\geqslant 1} {A_n}.
\] 
%
Já que $A_j \subseteq A_{j+1}$
então $\cap_{k\geqslant n} {A_k}=A_n$.
Assim segue da definição que 
\[
\liminf_{n\to\infty} A_n 
= 
\bigcup_{n\geqslant 1} {A_n}.
\]
Usando a definição de $\limsup$ e que a interseção de 
uma sequência de conjuntos
está contida em qualquer elemento da sequência temos 
\[
	\limsup_{n\to\infty} A_n 
	=
	\displaystyle\bigcap_{n\geqslant 1} 
		\left(\displaystyle\bigcup_{k\geqslant n}{A_k} \right)
	\subseteq 
	\displaystyle\bigcup_{k\geqslant 1} {A_k} 
	=
	\liminf_{n\to\infty} A_n 
	\subseteq 
	\limsup_{n\to\infty} A_n.
\]
Para provar o item 2 basta proceder de maneira análoga feita acima 
e usar as Leis de De Morgan.
\end{proof}







\begin{exercicio}
Com um resultado sobre limite de sequências monótonas de conjuntos, 
mostraremos mais a frente que
		 \[
		 	\liminf \limits_{n\to \infty}A_n 
			=
			\lim \limits_{n\to \infty}\left(\inf \limits_{k\geqslant n}A_k \right).
		 \]
\end{exercicio}






\section{Relações de ``Dualidade''}

Nesta seção deixamos como exercício para o leitor a prova de
seis relações 
entre a $\limsup,\ \liminf$ de conjuntos e de suas respectivas 
funções indicadoras; bem como algumas relações ligadas as operações
básicas de conjuntos. Estas relações explicam por si só a escolha 
do título desta seção. 

\begin{exercicio}
Sejam $\{A_n\}$ uma sequência de subconjuntos de um espaço $\Omega$, 
$A$ e $B$ dois subconjuntos arbitrários de $\Omega$. Mostre que 
\begin{enumerate}
\item 
$1_{\inf_{k\geqslant n} A_k} = \inf \limits_{k\geqslant n} 1_{A_k}$.

\item
$1_{\sup_{k\geqslant n} A_k} = \sup \limits_{k\geqslant n} 1_{A_k}$.

\item
$1_{\cup_{n\geqslant 1} A_n} 
\leqslant 
\sum \limits_{n \geqslant 1} 1_{A_n}$.

\item
$1_{\limsup A_n} = \limsup 1_{A_n}$.

\item
$1_{\liminf A_n} = \liminf 1_{A_n}$.

\item
$ 1_{A \triangle B} = 1_A + 1_B (\text{mod 2})$.
\end{enumerate}
\end{exercicio}

