\chapter[Aula 10]{Medida Produto}
\chaptermark{}

\section{Espaço Produto}

Sejam $(X, \mathscr{X})$ e $(Y, \mathscr{Y})$ espaços mensuráveis. 
Uma vez que sabemos que produto cartesiano de sigma-álgebras não 
constitui uma sigma álgebra, a sigma álgebra mais natural a ser 
considerada em $X\times Y$ passa a ser a sigma álgebra gerada pelos
\emph{retângulos mensuráveis}, a saber, a classe $\mathscr{R}=\{
A\times B;~~A\in \mathscr{X}, B\in \mathscr{Y}\}.$ Chamamos a 
sigma-álgebra gerada por $\mathscr{R}$ de sigma-álgebra produto e 
a denotamos por $\mathscr{X}\times \mathscr{Y}$.



\begin{proposicao}\label{Prod. 1}
Sejam $(X, \mathscr{X})$ e $(Y, \mathscr{Y})$ espaços mensuráveis e 
$\mathscr{X}\times \mathscr{Y}$ a sigma-álgebra produto em $X\times Y$.
Então as seguintes afirmações são verdade:
\begin{itemize}
\item[(a)] Se $E\in \mathscr{X}\times \mathscr{Y}$ então, para cada $x$ fixado,
 o conjunto $E_x=\{y: (x,y)\in E\}$ pertence a $\mathscr{Y}$. De modo análogo 
, fixado y,  o conjunto $E_y=\{x: (x,y)\in E\}$ pertence a $\mathscr{X}$.



\item[(b)] Se $f$ é uma função $\mathscr{X}\times \mathscr{Y}$ mensurável 
então, para cada $x$ fixado, a função $f(x, \cdot)$ é $\mathscr{Y}$-mensurável. De 
modo análogo, para cada $y$ fixo, a função $f(\cdot, y)$ é $\mathscr{X}$-mensurável
\end{itemize}
\end{proposicao}


Os conjunto $E_x$, e $E_y$ são chamados as secções de $E$ determinadas por $x$ e 
por $y$ respectivamente. Do mesmo modo as funções $f(x, \cdot)$ e $f(\cdot, y)$ 
são chamadas as secções de $f$ determinadas por $x$ e por $y$ respectivamente.
\medskip

\begin{proof}
 \emph{Item (a):} Fixe $x$ e considere o mapa $T_x:Y\to X\times Y$ dado por $T_x(y)=(x,y)$.
 Note que se $E=A\times B\in \mathscr{X}\times \mathscr{Y}$
é um retângulo então $T_{y}^{-1}E=B~\textnormal{ou}~\varnothing$ 
dependendo apenas se 
$A$ contém ou não o ponto $x$.  De qualquer modo $T_x^{-1}E\in \mathscr{Y}$.
\medskip

\noindent \emph{Item (b)}:  Neste item o primeiro passo é  mostrar 
que $T_x$ é $\mathscr{Y}$-$\mathscr{X}\times \mathscr{Y}$ 
mensurável. 
\medskip

\noindent \emph{Afirmação:} A classe
 $\mathcal{C}=\{E\in \mathscr{X}\times \mathscr{Y};~~T_x^{-1}E\in \mathscr{Y} \}$ 
 é uma sigma-álgebra.
\medskip 
 
\noindent É claro que a classe  $\mathcal{C}$ é
 não vazia pois $X\times Y\in \mathcal{C}$. 
 Se $E\in \mathcal{C}$ então 
$T_x^{-1}(X\times Y\setminus E)=Y\setminus T_x^{-1}E\in \mathscr{Y}.$  
Se $E_1, \ldots, E_n, \ldots \in \mathcal{C}$ entã
o $T_x^{-1}(\bigcup E_n)=\bigcup T_x^{-1}E_n\in \mathscr{Y}$. Portanto 
$C$ é uma sigma álgebra.
\medskip

Por outro lado mostramos acima que $\mathcal{C}$ contém a
 classe $\mathscr{R}$ dos retângulos mensuráveis. Sendo assim $\mathcal{C}$ contém a 
 sigma-álgebra gerada por $\mathscr{R}$ que para nossa sorte é $\mathscr{X}\times \mathscr{Y}$ 
 provando que $T_x$ é $\mathscr{Y}$-$\mathscr{X}\times \mathscr{Y}$ 
mensurável.  Agora, como $f$ é $\mathscr{X}\times \mathscr{Y}$-$\mathscr{B}(\mathbb{R})$ mensurável 
 segue que a composição $f\circ T_x$ é $\mathscr{Y}$-$\mathscr{B}(\mathbb{R})$ mensurável.

\end{proof}


\section{Medidas Produto}


Sejam $(X, \mathscr{X}, \mu)$ e $(Y, \mathscr{Y}, \nu)$
 espaços de medida com $\mu$ e $\nu$ finitas. 
 Temos pela proposição (\ref{Prod. 1}) que dado
  $E\in \mathscr{X}\times \mathscr{Y}$ as funções 
  $X\ni x\mapsto \nu(E_x)\in [0, \infty)$ e
   $Y\ni y\mapsto \mu(E_y)\in [0, \infty)$ estão bem definidas.
Denote por $\mathscr{L}$ a classe dos $E\in \mathscr{X}\times \mathscr{Y}$
para os quais as funções acima são $\mathscr{X}$ e $\mathscr{Y}$ mensuráveis 
respectivamente.

\begin{proposicao}
A classe $\mathscr{L}$ coincide com $\mathscr{X}\times \mathscr{Y}$.
\end{proposicao}
\begin{proof}
Começamos por mostrar que $\mathscr{L}$ é um $\lambda$ sistema. De fato, temos que
$\nu_x(X\times Y)=\nu(Y)=1$ para todo $x\in X$ portanto
 a função $X\ni x\mapsto \nu_x(X\times Y)\in [0,\infty)$ é a função constante
igual a $1$, que como sabemos é mensurável.
  Agora note que $\nu_x(E^c)=\nu(E^c_x)=\nu((E_x)^c)=1-\nu(E_x).$
 Assim, se $X\ni x\mapsto \nu_x(E)\in [0, \infty)$  é mensurável temos que 
 $X\ni x\mapsto \nu_x(E^c)=1-\nu_x(E)\in [0, \infty)$ é mensurável, sendo pois 
 a diferença de duas funções mensuráveis.
\end{proof}











