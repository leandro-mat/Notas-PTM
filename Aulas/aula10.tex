\chapter[Aula 10]{Medida Produto}
\chaptermark{}

\section{Espaço Produto}

Sejam $(X, \mathscr{X})$ e $(Y, \mathscr{Y})$ espaços mensuráveis. 
Uma vez que sabemos que produto cartesiano de sigma-álgebras não 
constitui uma sigma álgebra, a sigma álgebra mais natural a ser 
considerada em $X\times Y$ passa a ser a sigma álgebra gerada pelos
\emph{retângulos mensuráveis}, a saber, a classe $\mathscr{R}=\{
A\times B;~~A\in \mathscr{X}, B\in \mathscr{Y}\}.$ Chamamos a 
sigma-álgebra gerada por $\mathscr{R}$ de sigma-álgebra produto e 
a denotamos por $\mathscr{X}\times \mathscr{Y}$.



\begin{proposicao}
Sejam $(X, \mathscr{X})$ e $(Y, \mathscr{Y})$ espaços mensuráveis e 
$\mathscr{X}\times \mathscr{Y}$ a sigma-álgebra produto em $X\times Y$.
Então as seguintes afirmações são verdade:
\begin{itemize}
\item[(a)] Se $E\in \mathscr{X}\times \mathscr{Y}$ então, para cada $x$ fixado,
 o conjunto $E_x=\{y: (x,y)\in E\}$ pertence a $\mathscr{Y}$. De modo análogo 
, fixado y,  o conjunto $E_y=\{x: (x,y)\in E\}$ pertence a $\mathscr{X}$.



\item[(b)] Se $f$ é uma função $\mathscr{X}\times \mathscr{Y}$ mensurável 
então, para cada $x$ fixado, a função $f(x, \cdot)$ é $\mathscr{Y}$-mensurável. De 
modo análogo, para cada $y$ fixo, a função $f(\cdot, y)$ é $\mathscr{X}$-mensurável
\end{itemize}
\end{proposicao}


Os conjunto $E_x$, e $E_y$ são chamados as secções de $E$ determinadas por $x$ e 
por $y$ respectivamente. Do mesmo modo as funções $f(x, \cdot)$ e $f(\cdot, y)$ 
são chamadas as secções de $f$ determinadas por $x$ e por $y$ respectivamente.