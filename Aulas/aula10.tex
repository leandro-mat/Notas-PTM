\chapter[Aula 10]{Medida Produto}
\chaptermark{}

\section{Espaço Produto}

Sejam $(X, \mathscr{X})$ e $(Y, \mathscr{Y})$ espaços mensuráveis. 
Uma vez que sabemos que produto cartesiano de sigma-álgebras não 
constitui uma sigma álgebra, a sigma álgebra mais natural a ser 
considerada em $X\times Y$ passa a ser a sigma álgebra gerada pelos
\emph{retângulos mensuráveis}, a saber, a classe $\mathscr{R}=\{
A\times B;~~A\in \mathscr{X}, B\in \mathscr{Y}\}.$ Chamamos a 
sigma-álgebra gerada por $\mathscr{R}$ de sigma-álgebra produto e 
a denotamos por $\mathscr{X}\times \mathscr{Y}$.



\begin{proposicao}
Sejam $(X, \mathscr{X})$ e $(Y, \mathscr{Y})$ espaços mensuráveis e 
$\mathscr{X}\times \mathscr{Y}$ a sigma-álgebra produto em $X\times Y$.
Então as seguintes afirmações são verdade:
\begin{itemize}
\item[(a)]
\item[(b)]

\end{itemize}
\end{proposicao}