\chapter[Aula 10]{Medida Produto}
\chaptermark{}

\section{Espaço Produto}

Sejam $(X, \mathscr{X})$ e $(Y, \mathscr{Y})$ espaços mensuráveis. 
Uma vez que sabemos que produto cartesiano de sigma-álgebras não 
constitui uma sigma álgebra, a sigma álgebra mais natural a ser 
considerada em $X\times Y$ passa a ser a sigma álgebra gerada pelos
\emph{retângulos mensuráveis}, a saber, a classe $\mathscr{R}=\{
A\times B;~~A\in \mathscr{X}, B\in \mathscr{Y}\}.$ Chamamos a 
sigma-álgebra gerada por $\mathscr{R}$ de sigma-álgebra produto e 
a denotamos por $\mathscr{X}\times \mathscr{Y}$.



\begin{proposicao}\label{Prod. 1}
Sejam $(X, \mathscr{X})$ e $(Y, \mathscr{Y})$ espaços mensuráveis e 
$\mathscr{X}\times \mathscr{Y}$ a sigma-álgebra produto em $X\times Y$.
Então as seguintes afirmações são verdade:
\begin{itemize}
\item[(a)] Se $E\in \mathscr{X}\times \mathscr{Y}$ então, para cada $x$ fixado,
 o conjunto $E_x=\{y: (x,y)\in E\}$ pertence a $\mathscr{Y}$. De modo análogo 
, fixado y,  o conjunto $E_y=\{x: (x,y)\in E\}$ pertence a $\mathscr{X}$.



\item[(b)] Se $f$ é uma função $\mathscr{X}\times \mathscr{Y}$ mensurável 
então, para cada $x$ fixado, a função $f(x, \cdot)$ é $\mathscr{Y}$-mensurável. De 
modo análogo, para cada $y$ fixo, a função $f(\cdot, y)$ é $\mathscr{X}$-mensurável
\end{itemize}
\end{proposicao}


Os conjunto $E_x$, e $E_y$ são chamados as secções de $E$ determinadas por $x$ e 
por $y$ respectivamente. Do mesmo modo as funções $f(x, \cdot)$ e $f(\cdot, y)$ 
são chamadas as secções de $f$ determinadas por $x$ e por $y$ respectivamente.
\medskip

\begin{proof}
 \emph{Item (a):} Fixe $x$ e considere o mapa $T_x:Y\to X\times Y$ dado por $T_x(y)=(x,y)$.
 Note que se $E=A\times B\in \mathscr{X}\times \mathscr{Y}$
é um retângulo então $T_{y}^{-1}E=B~\textnormal{ou}~\varnothing$ 
dependendo apenas se 
$A$ contém ou não o ponto $x$.  De qualquer modo $T_x^{-1}E\in \mathscr{Y}$.
\medskip

\noindent \emph{Item (b)}:  Neste item o primeiro passo é  mostrar 
que $T_x$ é $\mathscr{Y}$-$\mathscr{X}\times \mathscr{Y}$ 
mensurável. 
\medskip

\noindent \emph{Afirmação:} A classe
 $\mathcal{C}=\{E\in \mathscr{X}\times \mathscr{Y};~~T_x^{-1}E\in \mathscr{Y} \}$ 
 é uma sigma-álgebra.
\medskip 
 
\noindent É claro que a classe  $\mathcal{C}$ é
 não vazia pois $X\times Y\in \mathcal{C}$. 
 Se $E\in \mathcal{C}$ então 
$T_x^{-1}(X\times Y\setminus E)=Y\setminus T_x^{-1}E\in \mathscr{Y}.$  
Se $E_1, \ldots, E_n, \ldots \in \mathcal{C}$ entã
o $T_x^{-1}(\bigcup E_n)=\bigcup T_x^{-1}E_n\in \mathscr{Y}$. Portanto 
$C$ é uma sigma álgebra.
\medskip

Por outro lado mostramos acima que $\mathcal{C}$ contém a
 classe $\mathscr{R}$ dos retângulos mensuráveis. Sendo assim $\mathcal{C}$ contém a 
 sigma-álgebra gerada por $\mathscr{R}$ que para nossa sorte é $\mathscr{X}\times \mathscr{Y}$ 
 provando que $T_x$ é $\mathscr{Y}$-$\mathscr{X}\times \mathscr{Y}$ 
mensurável.  Agora, como $f$ é $\mathscr{X}\times \mathscr{Y}$-$\mathscr{B}(\mathbb{R})$ mensurável 
 segue que a composição $f\circ T_x$ é $\mathscr{Y}$-$\mathscr{B}(\mathbb{R})$ mensurável.

\end{proof}


\section{Medidas Produto}
Sejam $(X, \mathscr{X}, \mu)$ e $(Y, \mathscr{Y}, \nu)$
 espaços de medida. A nossa  vontade é encontrar uma medida $\pi$ definida 
 em $\mathscr{X}\times \mathscr{Y}$ tal que se $E=A\times B$ seja um retângulo mensurável 
 tenhamos que $\pi(E)=\mu(A)\nu(B)$. A seguir vamos ver que quando 
 $ \mu$ e $\nu$ são sigma finitas a \emph{medida produto} não só existe como é única.
 \medskip
 

Sejam $(X, \mathscr{X}, \mu)$ e $(Y, \mathscr{Y}, \nu)$
 espaços de medida com $\mu$ e $\nu$ finitas. 
 Temos pela proposição (\ref{Prod. 1}) que dado
  $E\in \mathscr{X}\times \mathscr{Y}$ as funções 
  $X\ni x\mapsto \nu(E_x)\in [0, \infty)$ e
   $Y\ni y\mapsto \mu(E_y)\in [0, \infty)$ estão bem definidas.
Denote por $\mathscr{L}$ a classe dos $E\in \mathscr{X}\times \mathscr{Y}$
para os quais as funções acima são $\mathscr{X}$ e $\mathscr{Y}$ mensuráveis 
respectivamente.

\begin{proposicao}\label{Mens. Prod}
A classe $\mathscr{L}$ coincide com $\mathscr{X}\times \mathscr{Y}$.
\end{proposicao}
\begin{proof}
Começamos por mostrar que $\mathscr{L}$ é um $\lambda$-sistema. De fato, temos que
$\nu_x(X\times Y)=\nu(Y)=1$ para todo $x\in X$ portanto
 a função $X\ni x\mapsto \nu_x(X\times Y)\in [0,\infty)$ é a função constante
igual a $1$, que como sabemos é mensurável.
  Agora note que $\nu_x(E^c)=\nu(E^c_x)=\nu((E_x)^c)=1-\nu(E_x).$
 Assim, se $X\ni x\mapsto \nu_x(E)\in [0, \infty)$  é mensurável temos que 
 $X\ni x\mapsto \nu_x(E^c)=1-\nu_x(E)\in [0, \infty)$ é mensurável, sendo pois 
 a diferença de duas funções mensuráveis.
  Sejam $E_1, E_2, \ldots E_n, \ldots$ em $\mathscr{L}.$ E suficiente 
  considerarmos o caso em que os $E_n^{'s}$ são disjuntos. Note que $E_x=(\bigcup E_n)_x=\bigcup (E_n)_x$. Assim temos que 
  $$
  \nu_x(E)=\nu(E_x)=\nu(\bigcup (E_n)_x)=\sum_{n\geq 1}\nu_x(E_n).
  $$
  Como, por hipótese, para cada $n$, o mapa $X\ni x\mapsto \nu_x(E_n)\in [0,\infty)$ é mensurável, segue que 
  o mapa 
  $$
 X\ni x\mapsto \nu_x(E)=\sum_{n\geq 1}\nu_x(E_n)\in [0,\infty),
 $$ 
 sendo pois o limite de uma sequência de funções mensuráveis, também é mensurável. Isto conclui que $\mathscr{L}$ é um $\lambda$-sistema.
 
 Para finalizar veja  que $\mathscr{L}$ contém os retângulos mensuráveis $\mathscr{R}$, com efeito se $E=A\times B\in \mathscr{R}$ então o mapa $x\mapsto \nu(E_x)$ pode ser escrito como $1_A(x)\nu(B)$ que é claramente mensurável. Sendo assim temos pelo Teorema $\pi$-$\lambda$ que $\mathscr{X}\times \mathscr{Y}= \sigma(\mathscr{R})\subset \mathscr{L}\subset \mathscr{X}\times \mathscr{Y}$ concluindo a demonstração.
\end{proof}
\medskip


Passemos à construção da medida produto. Comecemos definindo funções de conjunto $\pi_1$ e $\pi_2$ como segue
\begin{equation}\label{M.Prod. 1}
\pi_1(E)=\int_X \nu_x(E) d\mu(x),~~~~~E \in \mathscr{X}\times \mathscr{Y}
\end{equation}
e 
\begin{equation}\label{M. Prod. 2}
\pi_2(E)=\int_Y \mu_y(E)d\nu(y),~~~~~E \in \mathscr{X}\times \mathscr{Y}
\end{equation}
é fácil ver que tanto $\pi_1$ quanto $\pi_2$ são medidas finitas em $\mathscr{X}\times \mathscr{Y}$. Agora note que se $E=A\times B\in\mathscr{X}\times \mathscr{Y}$  é um retângulo mensurável então temos que  $\nu_x(E)=1_A(x)\nu(B)$ e $\mu_y(E)=1_B\mu(A)$, de modo que 
\begin{equation}
\pi_1(A\times B)=\mu(A)\nu(B)=\pi_2(A\times B)
\end{equation}
de modo que as medidas $\pi_1$ e $\pi_2$ coincidem no $\pi$-sistema $\mathscr{R}$.  Denote por $\mathscr{L}$ a classe dos elementos $E$ de 
$\mathscr{X}\times \mathscr{Y}$  para os quais $\pi_1(E)=\pi_2(E)$.

\begin{proposicao}
A classe $\mathscr{L}$ coincide com $\mathscr{X}\times \mathscr{Y}$
\end{proposicao}


\begin{proof}
Seja $E=X\times Y$, então $\nu_x(E)=1_X(x)\nu(Y)=1_X(x)$ e $\mu_y(E)=1_Y(y)\mu(X)=1_Y(y)$. Portanto 
$$
\pi_1(X\times Y)=1=\pi_2(X\times Y)
$$
donde $X\times Y\in \mathscr{L}$. Seja $E\in \mathscr{L}$, então temos que
$$
\begin{array}{rcl}
\pi_1(E^c)&=&\displaystyle\int_X \nu_x(E^c)d\mu(x)
=
\int_X (1-\nu_x(E))d\mu(x)
\\[5mm]
&
=
&
\displaystyle 1-\int_X\nu_x(E) d\mu (x)
=
1-\int_Y \mu_y(E)d\nu(y)
\\[5mm]
&
=
&
1-\pi_2(E)
\\[5mm]
&
=
&
\pi_2(E^c)
\end{array}
$$
portanto $E^c\in \mathscr{L}$.  Finalmente sejam $E_1, \ldots, E_n, \ldots $ em $\mathscr{L}$, vamos mostrar que $E=\bigcup E_n\in \mathscr{L}.$ É suficiente considerar o caso em que os $E_n^{'s}$ são disjuntos. Para cada $n$ considere os conjunto $\bigcup_{j=1}^nE_n$, é claro que   $\bigcup_{j=1}^nE_n\nearrow E.$ Note também que para cada $n$ a função $x\mapsto \nu_x(E_n)$ é uma função não negativa. Sendo assim a temos que  $f_n=\sum_{j=1}^n\nu_{(\cdot)}(E_j)=\nu_x(\bigcup_{j=1}^{n}E_j)$ e $g_n=\sum_{j=1}^n\mu_{(\cdot)}(E_j)=\mu_y(\bigcup_{j=1}^{n}E_j)$ são  uma famílias não decrescentes de funções não negativas. Portanto segue do teorema da convergência monótona
que  $\bigcup_{j=1}^nE_n$

$$
\begin{array}{rcl}
\pi_1(E)&=&\displaystyle\lim_{n\to \infty}\pi_1(\bigcup_{j=1}^{n}E_j)
=
\lim_{n\to \infty}\int_X \nu_x(\bigcup_{j=1}^{n}E_j))d\mu(x)
\\[5mm]
&
=
&
\displaystyle \lim_{n\to \infty}\int_X (\sum_{j=1}^n\nu_x(E_j)) d\mu(x)
=
\lim_{n\to \infty}\int_Y(\sum_{j=1}^{n}\mu_y(E_j)d\nu(y)
\\[5mm]
&
=
&
\displaystyle\lim_{n\to \infty} \int_Y \mu_y(\bigcup_{j=1}^{n}E_j)d\nu(y)
\\
&=&
\pi_2(E)
\end{array}
$$
concluindo que $E\in \mathscr{L}$. Portanto $\mathscr{L}$ é um $\lambda$-sistema 
que contém o $\pi$-sistema $\mathscr{R}$, logo $\mathscr{X}\times \mathscr{Y}=\sigma(\mathscr{R})\subset \mathscr{L}\subset \mathscr{X}\times \mathscr{Y}.$

\end{proof}

\begin{corolario}
As medidas definidas em (\ref{M.Prod. 1}) e (\ref{M. Prod. 2}) estão bem definidas e  coincidem mesmo que as medidas 
$\mu$ e $\nu$ sejam apenas $\sigma$-fintas.
\end{corolario}

\begin{proof}
Sejam $\{A_n\}$ e $\{B_n\}$ decomposição de $X$ e $Y$ respectivamente em subconjuntos 
disjuntos de medida finita. Para cada $n,m$ defina as medidas $\mu_n$ e $\nu_m$ 
em  $\mathscr{X}$ e $ \mathscr{Y}$ respectivamente  pondo  $\mu_n(A)=\mu(A\cap A_n)$ e 
$\nu_m(B)=\nu(B\cap B_m)$. 
\medskip

\noindent \emph{Afirmação 1:} Seja $E=A\times B\in \mathscr{X}\times \mathscr{Y}$ um retângulo mensurável, temos que o mapa 
$X\ni x\mapsto \nu_x(E)\in [0, \infty)$ é mensurável.
\medskip


 De fato, temos para cada $n$ que  $\nu_n(E_x)=\nu(E_x\cap B_n)=1_A(x)  \nu(B\cap B_n)$, logo 
 $x\mapsto \nu_n(E_x)$ é mensurável. Como $\nu(E_x)=\sum_{n\geq 1}\nu_n(E_x)$ temos que 
 $x\mapsto \nu_x(E)$ é mensurável. De modo análogo mostra-se que $y\mapsto \mu_y(E)$ é mensurável.

Agora defina a classe $\mathscr{L}$ dos elementos de $E$ de $\mathscr{X}\times \mathscr{Y}$ tais que 
o mapa $x\mapsto \nu_x(E)$ é mensurável.

\noindent \emph{Afirmação 2:} A classe $\mathscr{L}$ coincide com $\mathscr{X}\times \mathscr{Y}$.
\medskip

A demonstração consiste em mostrar que $\mathscr{L}$ é um $\lambda$-sistema que contém o $\pi$-sistema dos 
retângulos mensuráveis.
 O fato de $\mathscr{L}$ conter os retângulos 
 mensuráveis é o conteúdo da afirmação anterior.
 Para provar que $\mathscr{L}$ é um $\lambda$-sistema basta procedermos como na proposição (\ref{Mens. Prod}).
  Tem-se uma afirmação análoga para o mapa $x\mapsto \mu_y(E)$. 
  
  Temos então que tanto $\pi_1$ quando $\pi_2$ estão bem definidas. Para ver que $\pi_1$ e $\pi_2$ coincidem 
{\red continua}

\section{Teorema de Fubini}  
 Sejam $(X, \mathscr{X}, \mu)$ e $(Y, \mathscr{Y}, \nu)$
 espaços de medida com $\mu$ e $\nu$ sigma finitas. Considere na sigma álgebra
 produto $\mathscr{X}\times \mathscr{Y}$ a medida produto $\mu\times \nu$. Nosso sonho 
 é verificar que valem expressões do tipo 
 \begin{equation}\label{Fubini1}
 \int_{X\times Y}f(x,y) d\mu\times d\nu=\int_X(\int_Yf(x,y) d\nu)d\mu
 \end{equation}
 \begin{equation}\label{Fubini2}
 \int_{X\times Y}f(x,y) d\mu\times d\nu=\int_Y(\int_Xf(x,y) d\mu)d\nu
 \end{equation}
é claro, pois a vida não é uma poesia, que as expressões (\ref{Fubini1}) e (\ref{Fubini2}) não devem ser
verificadas para todas as funções mensuráveis. Verificaremos abaixo que elas valem para uma classe bastante 
grande de funções, primeiro verificaremos que valem para funções mensuráveis positivas(Teorema de Tonneli), e mais geralmente
para funções em $L^1(\mu\times \nu)$(Teorema de Fubini).

\end{proof}


