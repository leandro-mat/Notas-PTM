\chapter[Aula 14]{Decomposição de Lebesgue e Funções de Densidade de Probabilidade}
\chaptermark{}

\section{O Teorema da Decomposição de Lebesgue}

Vamos continuar com algumas aplicações do Teorema 
de Radon-Nikodým para a teoria da probabilidade.
O objetivo é obter uma decomposição de qualquer 
função de distribuição em termos de uma combinação 
convexa de três outras funções de distribuição 
com propriedades especiais que vamos apresentar
ao longo desta seção. O primeiro passo nesta
direção será o Teorema da Decomposição de Lebesgue 
e para apresentar seu enunciado preciso vamos 
introduzir algumas definições. 


\begin{definicao}
Sejam $(\Omega,\F)$ um espaço mensurável,
$\lambda$ e $\mu$ duas medidas definidas 
sobre $(\Omega,\F)$.
Dizemos que $\lambda$ e $\mu$ são 
{\bf mutuamente singulares} 
se existem conjuntos $A$ $B$ não-vazios 
tais que $\Omega = A\cup B$ e $\lambda(A)=0=\mu(B)$.
Usamos a notação $\lambda\perp \mu$ para indicar 
que as medidas $\lambda$ e $\mu$ são mutuamente 
singulares. 
\end{definicao}


Embora esta relação de singularidade seja simétrica 
em $\lambda$ e $\mu$ costumamos dizer que $\lambda$
é {\bf singular com respeito a} $\mu$.


\begin{teorema}[Teorema da Decomposição de Lebesgue]
\label{teo-decomposicao-lebesgue}
Sejam $(\Omega,\F)$ um espaço mensurável,
$\lambda$ e $\mu$ duas medidas $\sigma$-finitas 
definidas sobre $(\Omega,\F)$.
Então existem medidas 
$\lambda_1$ e $\lambda_2$ definidas sobre  $(\Omega,\F)$
tais que $\lambda_1\perp \mu$, $\lambda_2\ll \mu$ 
e $\lambda = \lambda_1+\lambda_2$. Além do mais esta 
decomposição é única.
\end{teorema}




