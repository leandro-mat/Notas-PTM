\chapter[Aula 14]{Decomposição de Lebesgue e Funções de Densidade de Probabilidade}
\chaptermark{}

\section{O Teorema da Decomposição de Lebesgue}

Vamos continuar com algumas aplicações do Teorema 
de Radon-Nikodým para a teoria da probabilidade.
O objetivo é obter uma decomposição de qualquer 
função de distribuição em termos de uma combinação 
convexa de três outras funções de distribuição 
com propriedades especiais que vamos apresentar
ao longo desta seção. O primeiro passo nesta
direção será o Teorema da Decomposição de Lebesgue 
e para apresentar seu enunciado preciso vamos 
introduzir algumas definições. 


\begin{definicao}
Sejam $(\Omega,\F)$ um espaço mensurável,
$\lambda$ e $\mu$ duas medidas $\sigma$-finitas
quaisquer definidas sobre $(\Omega,\F)$.
Dizemos que $\lambda$ e $\mu$ são 
{\bf mutuamente singulares} 
se existem conjuntos $A$ $B$ não-vazios 
tais que $\Omega = A\cup B$ e $\lambda(A)=0=\mu(B)$.
Usamos a notação $\lambda\perp \mu$ para indicar 
que as medidas $\lambda$ e $\mu$ são mutuamente 
singulares. 
\end{definicao}


