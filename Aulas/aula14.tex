\chapter[Aula 14]{Decomposição de Lebesgue e Funções de Densidade de Probabilidade}
\chaptermark{}

\section{O Teorema da Decomposição de Lebesgue}

Vamos continuar com algumas aplicações do Teorema 
de Radon-Nikodým para a teoria da probabilidade.
O objetivo é obter uma decomposição de qualquer 
função de distribuição em termos de uma combinação 
convexa de três outras funções de distribuição 
com propriedades especiais que vamos apresentar
ao longo desta seção. O primeiro passo nesta
direção será o Teorema da Decomposição de Lebesgue 
e para apresentar seu enunciado preciso vamos 
introduzir algumas definições. 


\begin{definicao}
Sejam $(\Omega,\F)$ um espaço mensurável,
$\lambda$ e $\mu$ duas medidas definidas 
sobre $(\Omega,\F)$.
Dizemos que $\lambda$ e $\mu$ são 
{\bf mutuamente singulares} 
se existem conjuntos $A$ $B$ não-vazios 
tais que $\Omega = A\cup B$ e $\lambda(A)=0=\mu(B)$.
Usamos a notação $\lambda\perp \mu$ para indicar 
que as medidas $\lambda$ e $\mu$ são mutuamente 
singulares. 
\end{definicao}


Embora esta relação de singularidade seja simétrica 
em $\lambda$ e $\mu$ costumamos dizer que $\lambda$
é {\bf singular com respeito a} $\mu$.


\begin{teorema}[Teorema da Decomposição de Lebesgue]
\label{teo-decomposicao-lebesgue}
Sejam $(\Omega,\F)$ um espaço mensurável,
$\lambda$ e $\mu$ duas medidas $\sigma$-finitas 
definidas sobre $(\Omega,\F)$.
Então existem medidas 
$\lambda_1$ e $\lambda_2$ definidas sobre  $(\Omega,\F)$
tais que $\lambda_1\perp \mu$, $\lambda_2\ll \mu$ 
e $\lambda = \lambda_1+\lambda_2$. Além do mais esta 
decomposição é única.
\end{teorema}

\begin{proof}
Defina a medida $\nu:\F\to [0,+\infty]$ por
$\nu = \lambda+\mu$. Claramente $\nu$ define uma
medida $\sigma$-finita. 
Já que ambas $\lambda$ e $\mu$ são absolutamente contínuas
com respeito $\nu$ segue do Teorema de Radon-Nikodým 
que existem funções $f,g:\Omega\to [0,+\infty]$ mensuráveis
segundo $\F$ tais que para todo $E\in\F$ temos 
\[
\lambda(E)= \int_{E} f\, d\nu,
\quad\text{e}\quad
\mu(E)= \int_{E} g\, d\nu.
\] 
Considere os conjuntos $A=\{\omega\in\Omega: g(\omega)=0 \}$
e $B=\{\omega\in\Omega: g(\omega)>0\}$. Claramente temos 
$A\cap B =\emptyset$ e como $g$ é uma função 
não-negativa segue que $\Omega = A\cup B$. Com auxílio destes
dois conjuntos vamos agora construir a decomposição 
desejada. Defina para cada $E\in \F$ as seguintes medidas
\[
\lambda_1(E) = \lambda(E\cap A)
\quad\text{e}\quad
\lambda_2(E)= \lambda(E\cap B).
\]
Já que $\mu(A)=\int_A g\, d\nu = 0$
e $\lambda_1(B) = \lambda(B\cap A) =0 $ temos que 
$\mu \perp \lambda_1$.

Vamos provar agora que $\lambda_2\ll \mu$. 
Seja $E\in \F$ tal que $\mu(E)=0$. Então temos
$0= \mu(E)=\int_{E} g\, d\nu$. Desta igualdade 
concluímos que $g=0\ \nu$-q.t.p. em $E$. 
Como $E\cap B\subseteq E$ temos imediatamente 
que $\nu(E\cap B)=0$. Usando agora que $\lambda\ll \nu$
temos que $\lambda_2(E)=\lambda(E\cap B) = 0$  
e assim provamos que $\lambda_2\ll \mu$.
Claramente $\lambda =\lambda_1+\lambda_2$ e assim 
a existência da decomposição está estabelecida. 
Vamos chamar tal decomposição de decomposição 
de Lebesgue de $\lambda$ com respeito a $\mu$.

Resta mostrar a unicidade. 
Na prova da unicidade vamos usar o seguinte 
exercício 
\begin{exercicio}
Mostre que se $\mu$ e $\gamma$ são medidas 
sobre $(\Omega,\F)$ tais que  
$\gamma\ll \mu$ e $\gamma\perp \mu$, então 
$\gamma= 0$.
\end{exercicio}
%
A prova da unicidade será divida em 
dois casos. O primeiro caso consideramos que $\lambda$
é uma medida finita em $(\Omega,\F)$ e o segundo
caso consideramos que $\lambda$ é apenas $\sigma$-finita. 

Suponha que   
$\lambda = \rho+\sigma$ 
também seja uma decomposição de Lebesgue de $\lambda$ 
com respeito a $\mu$. Considere o conjunto $A$
definido acima. Já que $\mu(A)=0$ e $\sigma \ll \mu$
temos que $\sigma(A)=0$. Pela definição de $\lambda_1$
temos para todo $E\in\F$ que
$\lambda_1(E) = \lambda(E\cap A) = \rho(E\cap A)+\sigma(E\cap A)
\leq \rho(E)$. Como $\lambda_1+\lambda_2 = \rho+\sigma$
temos da desigualdade anterior que $\sigma(E)\leq \lambda_2(E)$ 
para todo $E\in\F$. Desta forma $\rho-\lambda_1$ e 
$\lambda_2-\sigma$ são medidas ambas absolutamente
contínuas e singulares com respeito a $\mu$ e 
portanto ambas devem ser medidas identicamente nulas.
\end{proof}

