\chapter[Aula 3]{Medida de Lebesgue em $\mathbb{R}$}
\chaptermark{}






\section*{Medida de Lebesgue}


Queremos construir em $\mathbb{R}$ uma medida que 
generalize a noção de comprimento de  
um intervalo, isto é, queremos uma medida  
$\mu$ definida em alguma classe razoável de 
subconjuntos de $\mathbb{R}$ tal que se I é um intervalo
com fecho $\overline{I}=[a,b]$ então $\mu(I)=b-a$ 
e se $A=\bigsqcup_{j=1}^nI_j$ onde cada $I_j$ 
é um intervalo do tipo $(a_j,b_j), [a_j,b_j],(a_j,b_j]$ ou $[a_j,b_j)$,  
então $\mu(A)=\sum_{j=1}^n(b_j-a_j)$ e
além do mais $\mu(\emptyset)=0$.



Pois bem, dado $A\subset \mathbb{R}$ 
definimos a medida exterior de $A$ e 
denotamos $\tn{Leb}^{*}(A)$ como sendo 
$$
\tn{Leb}^{*}(A)=\inf 
				\left\{
					\sum_{j=1}^\infty \mu(I_j):~ A\subset \bigcup_{j=1}^\infty I_j 
				\right\}
$$
onde o ínfimo é tomado sobre todas as coberturas de $A$ por intervalos fechados.  

Segue da Proposição \ref{prop-med-ext} que 
$\tn{Leb}^{*}:\mathcal{P}(\mathbb{R})\to[0,\infty]$ é de fato uma medida exterior.

\begin{lema}\label{LEB 2}
$\tn{Leb}^{*}([a,b])=b-a$
\end{lema}

\begin{proof}
É claro que $\tn{Leb}^{*}([a,b])\leq b-a$, 
precisamos apenas provar a desigualdade oposta. 
Comecemos tomando uma cobertura de 
$[a,b]\subset \bigcup_{k\in \mathbb{N}} (c_k,d_k)$  
por intervalos abertos, por compacidade podemos considerar 
apenas uma subcobertura finita de $[a,b]$, 
$$
[a,b]\subset \bigcup_{k=1}^M(c_k,d_k)
$$
suponha sem perda de generalidade que esta subcobertura seja minimal, 
isto é, se tirarmos qualquer intervalo que compõe a subcobertura 
não teremos mais uma subcobertura. Para facilitar o raciocínio 
ordenaremos os intervalos de modo que 
$$
c_0<c_1<\cdots<c_{M-1}
$$
note que a minimalidade da subcobertura garante então que 
$$
d_0<d_1<\cdots<d_{M-1}
$$
para fixar raciocínio  suponha que tivessemos $d_2<d_1$ 
então teríamos que $(c_2,d_2)\subset (c_1,d_1)$ o 
que contraria a minimalidade da cobertura. 
({\red um desenho aqui deixa tudo óbvio}) 
Agora defina 
$e_0=a$, $e_1=\frac{c_1+d_0}{2},$ $e_2=\frac{c_2+d_1}{2}$,
$e_{M-1}=\frac{c_{M-1}+d_{M-2}}{2}$ e $e_{M}=b.$  
Agora note que  para cada $k$, 
tem-se $[e_k,e_{k+1}]\subsetneqq [c_k,d_k]$ assim 
$$
\sum_{k=0}^{M-1} (d_k-c_k)>\sum_{k=0}^{M-1} (e_{k+1}-e_k)=b-a
$$ 
donde $\tn{Leb}^{*}([a,b])\geq b-a$.

\end{proof}

\begin{lema}

$\tn{Leb}^{*}((a,b))=b-a$
\end{lema}


\begin{proof}
Como o intervalo $(a,b)$ está contido no seu 
fecho $[a,b]$ portanto temos que  $\tn{Leb}^{*}((a,b))\leq b-a.$ 
Agora considere um intervalo $[a+\epsilon,b-\epsilon]$,
para algum $\epsilon>0$, 
contido em $(a,b)$ note que qualquer cobertura de 
$(a,b)$ por intervalos fechados é também uma cobertura 
de $[a+\epsilon,b-\epsilon]$ de modo que 
$$
b-a-2\epsilon= \tn{Leb}^{*}([a_0,b_0])\leq  \tn{Leb}^{*}((a,b))\leq b-a
$$
como $\epsilon$ é arbitrário segue que  $\tn{Leb}^{*}((a,b))=b-a$.
\end{proof}
Usando a mesma técnica da demonstração acima pode-se mostrar 
resultados análogos considerando intervalos  
da forma $[a,b)$ e $(a,b]$.





\begin{lema}\label{LEB 3}
Seja $A\subset \mathbb{R}$,  aberto e $A=\bigsqcup_{k=1}^M(a_k,b_k)$ então temos 
$$
\tn{Leb}^{*}(A)=\sum_{k=1}^M(b_k-a_k).
$$
\end{lema}

\begin{proof}
Segue diretamente da defininção que 
$\tn{Leb}^{*}(A)\leq\sum_{k=1}^M(b_k-a_k)$. 
Suponha que vale a desigualdade estrita, 
isto é, $\tn{Leb}^{*}(A)<\sum_{k=1}^M(b_k-a_k)$. 
Seja $a=\inf A$ e $b=\sup A$
e considere o conjunto $[a,b]\setminus A=\bigsqcup_{k=1}^{N}[c_l,d_l]$ 
então temos
$$
[a,b]=\left(~\bigsqcup_{k=1}^M(a_k,b_k)~\right)
\cup \left(~\bigsqcup_{l=1}^{N}(c_l,d_l)~\right)
$$
então pela propriedade da medida exterior 
\begin{eqnarray*}
\tn{Leb}^{*}([a,b])
&\leq &
\tn{Leb}^*\left(~\bigsqcup_{k=1}^M(a_k,b_k)~\right)
+\tn{Leb}^*\left(~\bigsqcup_{l=1}^{N}(c_l,d_l)~\right)
\\
&
<
&
\sum_{k=1}^{M} (b_k-a_k)+\sum_{l=1}^N(d_l-c_l)
\\
&
=
&
b-a
\end{eqnarray*}
contradizendo o lema (\ref{LEB 2}).

\end{proof}



\begin{proposicao}
Todo conjunto aberto $A\subset \mathbb{R}$ pode ser 
escrito, de modo único, como uma união enumerável de 
intervalos dois a dois disjuntos.
\end{proposicao}


\begin{proof}
Seja $x\in A$  e considere os números $a_x$ e $b_x$ 
definidos como a seguir
$$
a_x
=
\inf\{x;~x<a~\tn{e}~(x,a)\subset A\},~~b_x
=
\sup \{x;~x>a~\tn{e}~(a,x)\subset A\}
$$
note que  tanto $a_x$ quanto $b_x$ estão bem definidos 
pois como $A$ é  aberto sempre existe um intervalo $I$ 
satisfazendo $a\in I\subset A$. 
Então para cada $x\in A$ considere o intervalo 
$I_x=(a_x,b_x)$. Sejam $x,y\in A$, afirmamos que ou $I_x=I_y$ 
ou $I_x\cap I_y=\varnothing.$ 
De fato, suponha que $I_x\cap I_y\neq \varnothing$ 
então a reunião $I_x\cup I_y$ é um intervalo que 
contém tanto $x$ quanto $y$ de modo que temos, 
por definição que $  I_ x\cup I_y\subset I_y$ 
e $I_x\cup I_y\subset I_x$ portanto 
$$
I_x\subset I_x\cup I_y\subset I_y~~ I_y\subset I_x\cup I_y\subset I_x
$$
donde $I_x=I_y$.



Para ver que a família $\{I_x\}_{x\in A}$ é 
contável escolha em cada intervalo $I_x$ um 
racional $r_x$ e considere a aplicação  
$f:\{I_x,x\in A\}\to \mathbb{Q}$ definida por 
$f(I_x)=r_x$ e note que $f$ é injetiva, com efeito se 
$r_y=f(I_y)=f(I_x)=r_x$ temos que 
$I_x\cap I_y\neq \varnothing$ donde $I_x=I_y$ e portanto $r_x=r_y$.

\end{proof}




\begin{lema}\label{LEB 4}
Seja $A\subset \mathbb{R}$,  aberto e 
$A=\bigsqcup_{k\in \mathbb{N}}(a_k,b_k)$ então temos 
$$
\tn{Leb}^{*}(A)=\sum_{k\in \mathbb{N}}(b_k-a_k).
$$
\end{lema}

\begin{proof}
Temos pela definição de medida exterior que 
$\tn{Leb}^{*}(A)\leq \sum_{k\in \mathbb{N}}(b_k-a_k).$ 
Suponha que tenhamos na verdade 
$\tn{Leb}^{*}(A)< \sum_{k\in \mathbb{N}}(b_k-a_k)$ 
segue dai que existe $K\in \mathbb{N}$ tal que 
$$
\tn{Leb}^{*}(A)< \sum_{k<K}(b_k-a_k)
$$
 por outro lado pelo lema (\ref{LEB 3}) 
 que $\sum_{k<K}(b_k-a_k)=\tn{Leb}^{*}(A_0)$ 
 onde $A_0=\bigcup_{k<K}(a_k,b_k)$.
Temos então uma contradição pois 
$A_0\subset A$ e $\tn{Leb}^{*}(A)<\tn{Leb}^{*}(A_0).$
\end{proof}

Note que em última análise o resultado acima 
vale para outras uniões enumeráveis disjuntas 
de intervalos não necessariamente abertos, 
isto é, vale para uniões disjuntas 
$A=\bigsqcup_{n\in\mathbb{N}} I_n$ onde 
cada intervalo $I_n$ pode ser qualquer um 
dos intervalos $[a,b),[b,a)$ ou $[a,b]$. 



%Veja que estamos considerando a semi-álgebra $\mathcal{S}$ dos cubos fechados de $\mathbb{R}^d$ e a medida finitamente aditiva $\mu$ caracterizada por 
%$\mu(C)=|C|$ onde $C$ é um cubo fechado.  Então $\tn{Leb}^{*}$ é a medida exterior induzida por esta medida finitamente aditiva.

\begin{proposicao}
Sejam $A_1, \ldots, A_n, \ldots$ elementos da semi-álgebra 
$\mathcal{S}$ tais que $A=\bigcup_{i=1}^{\infty}A_n$


\end{proposicao}




\begin{proposicao}
Seja $A\subset \mathbb{R}$ então 
$ 
\tn{Leb}^{*}(A)
=
\inf_{\mathcal{O}\supset A}\tn{Leb}^{*}(\mathcal{O})
$ 
onde $\mathcal{O}$ é um aberto contendo $A$.
\end{proposicao}


\begin{proposicao}
Sejam $A,B\subset \mathbb{R}$ com $\tn{dist}(A,B)>0$ 
então 
$
\tn{Leb}^{*}(A\cup B)
=
\tn{Leb}^{*}(A)+\tn{Leb}^{*}(B)
$.
\end{proposicao}



\subsubsection{Uma construç\~ao autom\'atica da medida de Lebesgue}


Considere álgebra $\mathcal{A}$ consistindo dos 
intervalos e uniões de intervalos de   $\mathbb{R}.$ 
Seja  $\mu$ a  função de conjunto  nessa álgebra 
caracterizada do seguinte modo, se $[a,b]$ é um 
intervalo  então $\mu([a,b])=b-a$ 
e se $A=\cup_{i=1}^n(a_i,b_i)$ então 
$\mu(A)=\sum_{i=1}^n(b_i-a_i).$  
Note que $\mu$ é claramente uma medida 
finitamente aditiva em $\mathcal{A}$, 
além do mais o Lema (\ref{LEB 4}) mostra 
que $\mu$ é $\sigma$-aditiva. 
O teorema de extensão de Caratheodory diz que a 
medida $\mu$ se estende a uma medida definida na 
$\sigma$-álgebra  $\mathcal{M}$  
caracterizada pela condição de Caratheodory. 
Além do mais se considerarmos a restrição 
da medida obtida  à $\sigma$-álgebra
$\sigma(\mathcal{S})\subset \mathcal{M}$ 
esta restrição é única. A medida que é a 
extensão de $\mu$ é dita a medida de Lebesgue 
de $\mathbb{R}$ e denotada por $\tn{Leb}$, 
a sigma álgebra $\mathcal{M}$ nesse caso particular
é dita a $\sigma$-álgebra de Lebesgue e é denotada por $\mathcal{L}$.






































\subsection*{Uma outra construç\~ao da medida de Lebesgue}













\subsubsection{Caratheodory mensur\'avel e Lebesgue mensur\'avel}




A condição que define mensurabilidade no sentido de 
Caratheodory parece muito misteriosa, ou em 
um portugês mais claro 
parece ``tirada do chapeu", o que está por 
trás disso são anos de pesquisa árdua. 
Entretanto em $\mathbb{R}^d$ é possível definir 
mensurabilidade de maneira bastante razoável. 
Diremos que $A\subset \mathbb{R}^d$ é Lebesgue  
mensurável quando para cada  $\epsilon>0$ 
existe um aberto $\mathcal{O}\supset A$ 
tal que $\tn{Leb}^{*}(\mathcal{O}\setminus A)\leq \epsilon$. 
Esta definição parece portanto bastante razoável, 
um conjunto $A\subset \mathbb{R}^d$ é mensurável 
se sua medida exterior  pode ser bem aproximada 
pela medida exterior de abertos contendo $A$.

A grande pergunta então é se o conceito de 
mensurabilidade dado acima coincide com o 
conceito de mensurabilidade 
no sentido dado por Caratheodory. 
Bom  a resposta é sim como mostraremos.



Seja $A\subset \mathbb{R}^d$ e suponha que $A$ 
seja Lebesgue mensurável. 
Note que todo aberto 
é Caratheodory mensurável. 
Dado $\epsilon>0$ seja $\mathcal{O}$ um 
conjunto aberto contendo $A$ 
tal que 
$\tn{Leb}^{*}(\mathcal{O}\setminus A)\leq \epsilon$  
então dado $E\subset \mathbb{R}^d$ 
arbitrário temos 
\begin{eqnarray*}
\tn{Leb}^{*}(E\cap A)+\tn{Leb}^{*}(E\cap A^c)
&\leq& 
\tn{Leb}^{*}(E\cap \mathcal{O})+\tn{Leb}^{*}(E\cap A^c)
\\
&
\leq 
&
\tn{Leb}^{*}(E\cap\mathcal{O})+
\tn{Leb}^{*}(E\cap (A^c\setminus \mathcal{O}^c))
\\
&
+
&
 \tn{Leb}^{*}(E\cap \mathcal{O}^c)
 \\
 &
 \leq 
 &
\tn{Leb}^{*}(E)+\epsilon
\end{eqnarray*}
onde usamos acima que 
$
A\cap E\subset \mathcal{O}\cap E,$ $A^c
=
(A^c\setminus \mathcal{O}^c)\cup \mathcal{O}^c
$ 
e que 
$
A^c\setminus \mathcal{O}^c=\mathcal{O}\setminus A$ como 
$\epsilon>0$ é arbitrário concluímos que
$$
\tn{Leb}^{*}(E)\geq \tn{Leb}^{*}(E\cap A)+\tn{Leb}^{*}(E\cap A^c)
$$
donde $A$ é Caratheodory mensurável.


A recíproca é um tanto mais simples. De fato seja 
$A\subset \mathbb{R}^d$ Caratheodory mensurável 
tal que $\tn{Leb}^{*}(A)<\infty$ dado $\epsilon>0$  
existe uma família $\{C_i\}_{i=1}^{\infty}$ de 
cubos fechados tais que 
$A\subset \bigcup_{i=1}^{\infty}C_i$ e 
$\sum_{i=1}^{\infty}|C_i|\leq \tn{Leb}^{*}(A)+\epsilon$.
Agora para cada $i$ consideremos um cubo $\tilde{C}_i$ aberto 
tal que $|\tilde{C}_i|<|C_i|+\epsilon/2^{i}.$
Como cada $\tilde{C}_i$ é Caratheodory mensurável sua 
união enumerável também o é de modo que podemos escrever 
\begin{eqnarray*}
\tn{Leb}^{*}(\bigcup_{i=1}^{\infty} \tilde{C}_i\setminus A)
&=&
\tn{Leb}^{*}(\bigcup_{i=1}^{\infty} \tilde{C}_i)-\tn{Leb}^{*}(A)
\\
&
=
&
\sum_{i=1}^{\infty}|\tilde{C}_i|-\tn{Leb}^{*}(A)
\\
&
\leq 
&
\sum_{i=1}^{\infty}|C_i|-\tn{Leb}^{*}(A)+\sum_{i=1}^{\infty}\epsilon/2^i
\\
&
\leq 
&
2\epsilon
\end{eqnarray*}
então tomando  $\mathcal{O}=\bigcup_{i=1}^{\infty}\tilde{C}_i$ 
temos que $\tn{Leb}^{*}(\mathcal{O}\setminus A)\leq \epsilon$.
 
Fazendo um apanhado da pequena discussão acima, 
concluímos que a noção de conjunto mensurável no 
sentido de Caratheodory quando partimos da semi-álgebra 
dos retângulos em $\mathbb{R}^d$ e a noção de mensurabilidade
no sentido de Lebesgue são as mesmas, 
em particular provamos que os mensuráveis no 
sentido de Lebesgue também formam uma sigma álgebra.




\subsubsection{Uma construção artesanal da medida de Lebesgue}

A seguir refaremos em $\mathbb{R}^d$ todo o processo 
de extensão da medida finitamente aditiva feita na 
seção anterior para a única diferença e que 
usaremos apenas o conceito de mensurabilidade 
no sentido de Lebesgue.
É claro que do ponto de vista 
rigoroso toda a construção que virá é 
desnecessária tendo em vista o que já foi discutido acima, 
por outro lado do ponto de vista didático esta 
discussão será  muito útil.









\begin{definicao}
Um subconjunto $A\subset \mathbb{R}^d$ é dito 
Lebesgue mensurável, ou simplesmente mensurável,  
se para cada $\epsilon>0$ existe um 
conjunto aberto $\mathcal{O}\supset A$ 
tal que $\tn{Leb}^{*}(\mathcal{O}\setminus A)\leq \epsilon$.
\end{definicao}


\begin{proposicao}
Todo conjunto aberto $A\subset \mathbb{R}^d$ é mensurável.
\end{proposicao}






\begin{proposicao}
Se $\tn{Leb}^{*}(A)=0$ então $A$ é  mensurável, em particular 
todo subconjunto de $A$ é mensurável.
\end{proposicao}
\begin{proof}
Temos pela proposição (??) que dado $\epsilon>0$ existe 
um aberto $\mathcal{O}\supset A$ tal que 
$\tn{Leb}^{*}(\mathcal{O})\leq \epsilon$, 
como $\mathcal{O}\setminus E\subset \mathcal{O}$ 
temos por monotonicidade da medida exterior 
que $\tn{Leb}^{*}(\mathcal{O}\setminus E)\leq \epsilon$. 
\end{proof}





\begin{proposicao}
Toda  união enumerável de conjuntos mensuráveis é mensurável.
\end{proposicao}

\begin{proof}
Seja $A=\bigcup_{j=1}^{\infty} A_j$ onde cada $A_j$ é mensurável. 
Dado $\epsilon>$ temos para cada $j$ que existe um 
aberto $\mathcal{O}_j\supset A_j$ e 
tal que $\tn{Leb}^{*}(\mathcal{O}_j\setminus A_j)\leq \epsilon/2^j$. 
Temos então que  a união 
$\mathcal{O}=\bigcup_{j=1}^{\infty}\mathcal{O}_j$ 
é um conjunto aberto,   $A\subset \mathcal{O}$ e
que 
$$
\mathcal{O}\setminus A=\bigcup_{j=1}^{\infty} \mathcal{O}_j\setminus A_j
$$
por monotonicidade e subaditividade que 
$$
\tn{Leb}^{*}(\mathcal{O}\setminus A)
\leq 
\sum_{n=1}^\infty \tn{Leb}^{*}(\mathcal{O}_j\setminus A_j)
\leq 
\sum_{n=1}^\infty \epsilon/2^j=\epsilon.
$$

\end{proof}


\begin{proposicao}
Conjuntos fechados de $\mathbb{R}^d$ são mensuráveis.
\end{proposicao}





\begin{proposicao}
Complementos de conjuntos mensuráveis são mensuráveis
\end{proposicao}


\begin{proposicao}

\end{proposicao}
  

