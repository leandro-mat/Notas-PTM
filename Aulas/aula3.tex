\chapter[Aula 3]{Medida de Lebesgue em $\mathbb{R}$}
\chaptermark{}








\section{Medida de Lebesgue em $(0,1]$}


Nesta seção vamos mostrar como construir 
a medida de Lebesgue no 
intervalo semi-aberto $(0,1]$ via o Teorema 
da Extensão de Carathéodory. 
Para isto 
vamos considerar a coleção de 
subconjuntos do intervalo $(0,1]$ dada por 
\[
\mathcal{A}
=
\left\{ 
A\subset (0,1]:\ 
\begin{array}{c}
A\
\text{é uma união finita de intervalos disjuntos}
\\
\text{da forma}\ (a,b]
\ \text{com}\ 0\leq a<b\leq 1
\end{array}
\right\}
\cup
\{\emptyset\}.
\]
Afirmamos que $\mathcal{A}$ é uma álgebra de subconjuntos 
de $(0,1]$.
Para provar a afirmação observamos 
primeiro que ambos conjuntos $\emptyset$ e $(0,1]$ pertencem
a $\mathcal{A}$.  
Seja $A=(a_1,b_1]\cup\ldots (a_n,b_n]\in\mathcal{A}$. 
Sem perda de generalidade, podemos assumir que 
$a_1\leq \ldots \leq a_n$.
Por questão de conveniência 
para $b_i=a_{i+1}$
vamos convencionar que $(b_i,a_{i+1}]=\emptyset$. 
Já que $A\in\mathcal{A}$ 
temos que os $n$ intervalos que formam o conjunto 
$A$ são mutuamente disjuntos, logo
$
A^c=(0,a_1]\cup(b_1,a_2]
\cup\ldots\cup (b_{n-1},a_n]\cup (a_n,1] 
\in\mathcal{A}.
$
Se $B=(c_1,d_1]\cup\ldots (c_m,d_m]\in\mathcal{A}$
temos pelas propriedades de elementares de conjuntos
que 
\[
A\cap B
=
\bigcup_{i=1}^n \bigcup_{j=1}^{m}
\Big( (a_i,b_i]\cap (c_j,d_j] \Big)
=
\bigcup_{i=1}^n \bigcup_{j=1}^{m}
\Big( (\max\{a_i,c_j\},\min\{b_i,d_j\}] \Big).
\]
Como a coleção de intervalos aparecendo acima é
mutuamente disjunta segue que $A\cap B\in \mathcal{A}$
e assim encerramos a prova de que $\mathcal{A}$ 
é uma álgebra de subconjuntos de $(0,1]$.

Embora, a coleção $\mathcal{A}$ seja uma álgebra
ela não é $\sigma$-álgebra, para ver isto basta 
observar que a coleção $\mathcal{A}$ não possui 
nenhum dos conjuntos unitários $\{c\}\subset (0,1]$
que são obtidos, por exemplo, como uma interseção 
enumerável de subconjuntos de $\mathcal{A}$ da seguinte maneira
$\cap_{n>c^{-1}} (c-1/n,c]$.
A álgebra $\mathcal{A}$ também não contém conjuntos que 
são uniões enumeráveis de intervalos que não pode ser 
expressa como uma união finita de intervalos disjuntos.


Considere a função de conjuntos 
$\mathrm{Leb}:\mathcal{A}\to [0,+\infty]$ definida por 
\[
\mathrm{Leb}
\left(
	\bigcup_{i=1}^n (a_i,b_i]
\right) 
= 
\sum_{i=1}^{n} (b_i-a_i).
\]


\begin{teorema}\label{teo-leb-sigma-aditiva-algebra}
Sejam $I=(a,b]$ e $I_k=(a_k,b_k]$ 
uma sequência arbitrária de intervalos 
todos contidos em $(0,1]$.
\begin{itemize}
	\item[1)] 
	Se $\cup_{k}I_k\subset I$ e 
	$I_k$'s são disjuntos, 
	então $\sum_{k}\mathrm{Leb}(I_k)\leq \mathrm{Leb}(I)$.
	
	\item[2)]
	Se $I\subset \cup_k I_k$ (os $I_k$'s não necessariamente disjuntos),
	então temos que $\mathrm{Leb}(I)\leq \sum_{k}\mathrm{Leb}(I_k)$.
	
	\item[3)]
	Se $I=\cup_k I_k$ e $I_k$'s são disjuntos, então 
	$\mathrm{Leb}(I)= \sum_{k}\mathrm{Leb}(I_k)$.
\end{itemize}
\end{teorema}



\begin{proof}
Evidentemente 3) segue de 1) e 2).
Vamos provar 1). Primeiro vamos 
estabelecer este fato para o caso 
em que $\cup_{k} I_k$ é uma união finita. 
A prova deste caso será feita por indução 
no número de intervalos.
Para o caso $n=1$ o resultado é óbvio. 
Suponha que 1) vale para união de $n$ intervalos
disjuntos contida em $I$. Sem perda de generalidade 
podemos supor que 
$I_1=(a_1,b_1],\ldots, I_n=(a_n,b_n], I_{n+1}=(a_{n+1},b_{n+1}]$
são tais que $a_1,\ldots, a_n <a_{n+1}$. 
Já que todos $I_k$'s estão contidos em $(a,b]$
temos que 
$\cup_{k=1}^n (a_k,b_k]\subset (a,a_{n+1}]$.
Pela hipótese de indução temos que
\[
\sum_{k=1}^n (b_k-a_k)
\equiv
\sum_{k=1}^n \mathrm{Leb}(I_k)
\leq
\mathrm{Leb}((a,a_{n+1}])
\equiv
a_{n+1}-a.
\] 
Usando a definição de $\mathrm{Leb}$ e a desigualdade 
acima obtemos
\begin{align*}
\mathrm{Leb}(I_{n+1})
+ 
{\textstyle \sum_{k=1}^n \mathrm{Leb}(I_k)}
&\leq
(b_{n+1}-a_{n+1})+(a_{n+1}-a)
\\
&=
b_{n+1}-a
\\
&\leq
b-a
\\
&=
\mathrm{Leb}(I).
\end{align*}

No caso em que a união em 1) é infinita basta aplicar o 
resultado acima para cada subcoleção finita obtendo 
$\sum_{k=1}^n \mathrm{Leb}(I_k)\leq b-a$ para todo $n\in\mathbb{N}$.
Já que a cota superior é uniforme em $n$ segue que 
$\sum_{k=1}^{\infty} \mathrm{Leb}(I_k)\leq b-a$.



Passamos agora para a prova do item 2). 
Por questão de simplicidade vamos dividir a 
prova em dois casos. Primeiro consideramos 
que a união em 2) é finita.
Vamos novamente argumentar por indução. 
Suponha que o resultado seja válido para
uma coleção de $n$ intervalos e assuma que 
$(a,b]\subset \cup_{k=1}^{n+1} (a_k,b_k]$.
Sem perda de generalidade podemos 
supor que $b\in (a_{n+1},b_{n+1}]$, 
isto é, $a_{n+1}<b\leq b_{n+1}$.
Se $a_{n+1}<a$ o resultado é óbvio.
Caso contrário temos $(a,a_{n+1}]\subset \cup_{k=1}^n (a_k,b_k]$.
Pela hipótese de indução segue que
\[
a_{n+1}-a
\equiv
\mathrm{Leb}((a,a_{n+1}])
\leq
\sum_{k=1}^n \mathrm{Leb}((a_k,b_k])
\equiv
\sum_{k=1}^n (b_k-a_k).
\]
Somando $b_{n+1}-a_{n+1}$ em ambos termos
da desigualdade acima e lembrando que 
$b\leq b_{n+1}$ ficamos com 
\[
\mathrm{Leb}(I)\equiv 
b-a
\leq
b_{n+1}-a
\leq  
\sum_{k=1}^{n+1} (b_k-a_k)
\equiv
\sum_{k=1}^{n+1} \mathrm{Leb}((a_k,b_k]).
\]
e portanto o item 2) está provado 
para coleções finitas. Vamos agora analisar o caso de coleções
infinitas. Suponha que $(a,b]\subset \cup_{k=1}^{\infty}(a_k,b_k]$.
Dado $\varepsilon>0$ tal que $0<\varepsilon<b-a$ temos que 
\[
[a+\varepsilon,\ b]
\subset 
\bigcup_{k=1}^{\infty} \left( a_k,\ b_{k}+\frac{\varepsilon}{2^k}\right).
\]
Já que $[a+\varepsilon,\ b]$ é um compacto existe uma subcobertura
finita da cobertura por abertos dada acima.
Desta forma para algum $n\in\mathbb{N}$ podemos afirmar que 
\[
[a+\varepsilon,\ b]
\subset 
\bigcup_{k=1}^{n} \left( a_k,\ b_{k}+\frac{\varepsilon}{2^k}\right).
\]
Deste fato podemos concluir imediatamente que 
\[
(a+\varepsilon,\ b]
\subset 
\bigcup_{k=1}^{n} \left( a_k,\ b_{k}+\frac{\varepsilon}{2^k}\right].
\]
Usando agora o resultado provado acima para o caso finito 
podemos afirmar que
\[
b-(a+\varepsilon)
\leq 
\sum_{k=1}^n 
\Big(b_k+\frac{\varepsilon}{2^k} -a_k\Big)
\leq
\sum_{k=1}^{\infty} (b_k-a_k)+\varepsilon.
\]
Como $\varepsilon$ é arbitrário ficamos com
\[
\mathrm{Leb}(I)
\leq
\sum_{k=1}^{\infty} \mathrm{Leb}(I_k). 
\qedhere
\]
\end{proof}


Com os Teoremas 
\ref{teo-leb-sigma-aditiva-algebra}
e
\ref{teorema-caratheodory}
em mãos
podemos apresentar finalmente a  medida de Lebesgue
em $(0,1]$.
Pois bem, dado $A\subset (0,1]$ 
definimos a medida exterior de $A$ como
em \eqref{def-medida-exterior}, ou seja,
\[
\tn{Leb}^{*}(A)=\inf 
	\left\{
		\sum_{j=1}^\infty \mathrm{Leb}(I_j):
		\ A\subset \bigcup_{j=1}^\infty I_j,
		\ \text{com}\ I_j\in \mathcal{A}\ \ \forall  j\in\mathbb{N}
	\right\}.
\]

Pelo Teorema \ref{teo-leb-sigma-aditiva-algebra} 
temos que $\mathrm{Leb}:\mathcal{A}\to [0,\infty]$
é uma medida na álgebra $\mathcal{A}$.
Pelo item 2) do Teorema de Carathéodory podemos
afirmar que  $\sigma(\mathcal{A})\subset \mathcal{M}$,
onde $\mathcal{M}$ é a coleção dos conjuntos 
mensuráveis com respeito a 
medida exterior $\mathrm{Leb}^*$,
além do mais o teorema afirma que $\mathrm{Leb}^*\big|_{\mathcal{M}}$
é uma medida e que para todo intervalo $I=(a,b]\in\mathcal{A}$ 
que $\mathrm{Leb}^*(I)=\mathrm{Leb}(I)=b-a$.
Esta medida é conhecida como a {\bf Medida de Lebesgue} 
em $(0,1]$ e será denotada por $\lambda$.

A $\sigma$-álgebra $\sigma(\mathcal{A})$ é a $\sigma$-álgebra
gerada pela coleção dos intervalos abertos contidos em $(0,1]$
chamada normalmente de $\sigma$-álgebra de Borel de $(0,1]$.
Já a $\sigma$-álgebra $\mathcal{M}$ é conhecida
como a $\sigma$-álgebra de Lebesgue de $(0,1]$.
Uma pergunta natural neste momento é como se relacionam 
as $\sigma$-álgebras de Borel e Lebesgue. Vamos mostrar 
mais a frente que 
\[
\sigma(\mathcal{A})
\subsetneq 
\mathcal{M}
\subsetneq 
\mathcal{P}((0,1]).
\]
A prova que a primeira continência é estrita 
será dada mostrando que a $\sigma$-álgebra 
de Borel $\sigma(\mathcal{A})$ 
tem a cardinalidade de $\mathbb{R}$ 
enquanto que a $\sigma$-álgebra de Lebesgue 
$\mathcal{M}$ tem a cardinalidade das 
partes de $\mathbb{R}$.
Para provar que a segunda continência é estrita
vamos usar o famoso exemplo de Giuseppe Vitali
de 1905, que permanece até nos dias de hoje 
sendo o exemplo mais simples.


\begin{teorema}
	Seja $\lambda:\mathcal{M}\to [0,+\infty]$ a medida de Lebesgue em $(0,1]$
	e suponha que $\nu:\mathcal{M}\to [0,+\infty]$ seja uma medida satisfazendo 
	$\nu\big|_{\mathcal{A}} = \lambda\big|_{\mathcal{A}}$.
	Então $\nu(E) = \lambda(E)$ para todo conjunto $E\in \mathcal{M}$. 
\end{teorema}

\begin{observacao}
	No teorema acima, se desejamos mostrar apenas 
	$\nu\big|_{\sigma(\mathcal{A})} = \lambda\big|_{\sigma(\mathcal{A})}$
	basta aplicar o item 3) do Teorema da Extensão de Carathéodory. 
	Porém para estender a conclusão a 
	todos os conjuntos Lebesgue mensuráveis, como afirmado no enunciado,
	é necessário um pouco mais de trabalho.	
\end{observacao}










\section{Medida de Lebesgue em $\mathbb{R}$}


Existem várias maneira equivalentes de se definir a medida 
de Lebesgue em $\mathbb{R}$.
Para exercitar as técnicas apresentadas na seção anterior
vamos mostrar como fazer esta construção a partir da extensão 
de uma medida definida em uma álgebra de conjuntos de $\mathbb{R}$.
Na verdade esta seção apresenta apenas um roteiro para 
esta construção com grande parte dos detalhes sendo deixados 
como exercício.

Considere a seguinte coleção de subconjuntos de $\mathbb{R}$
\begin{equation}\label{algebra-conjuntos-R}
\mathcal{A}
=
\left\{ 
A\subset \mathbb{R}:\ 
\begin{array}{l}
A\
\text{é uma união finita de intervalos disjuntos}
\\
\text{dos seguintes tipos:}
\\
I)\phantom{II} \ \ (a,b]
\ \text{com}\ -\infty< a<b< +\infty;
\\
II)\phantom{I} \ \ (a,+\infty) \ \text{com}\  a\in\mathbb{R};
\\
III) \ \ (-\infty,a]  \ \text{com}\  a\in\mathbb{R};
\end{array}
\right\}
\cup
\{\mathbb{R},\emptyset\}.
\end{equation}



\begin{exercicio}
	Mostre que a coleção $\mathcal{A}$ definida acima é uma 
	álgebra de conjuntos de $\mathbb{R}$.
\end{exercicio}

Defina a função de conjuntos $\textrm{Leb}:\mathcal{A}\to[0,+\infty]$ 
da seguinte maneira:

\[
\textrm{Leb}(A)
=
\begin{cases}
\displaystyle
\sum_{i=1}^n (b_i-a_i),
	&
	\begin{array}{l}
	\text{se} \ A=\bigsqcup_{j=1}^n I_j, \ 
	\text{onde}\  I_j\equiv (a_j,b_j]
	\\
	\text{com}\ -\infty< a_j<b_j< +\infty \ \forall j=1,\ldots,n;
	\end{array}
\\[0.8cm]
+\infty,
	& 
\begin{array}{l}
	\text{se} \ A=\bigsqcup_{j=1}^n I_j\ \text{e existe} \ 1\leq q\leq n 
	\\
	\ \text{tal que}\ I_q\ \text{é do tipo} \ II\ \text{ou}\ III.
\end{array}
\end{cases}
\]



\begin{exercicio}
	Mostre que a função $\mathrm{Leb}:\mathcal{A}\to [0,+\infty]$ como
	definida acima é uma medida na álgebra $\mathcal{A}$.
\end{exercicio}
\vspace*{-0.3cm}
\noindent{\bf Dica.} Generalize o Teorema \ref{teo-leb-sigma-aditiva-algebra}.
\bigskip

Para finalizar a construção da medida de Lebesgue em 
$\mathbb{R}$, consideramos a medida exterior 
$\mathrm{Leb}^*:\mathcal{P}(\mathbb{R})\to [0,+\infty]$ 
definida como na Proposição \ref{prop-med-ext}.
Temos pelo item $1)$ do 
Teorema da Extensão de Carathéodory 
que a medida exterior $\mathrm{Leb}^*$ restrita a coleção $\mathcal{M}$
de todos os subconjuntos de $\mathbb{R}$ que são
$\mathrm{Leb}^*$-mensuráveis é de fato uma medida. 
Esta medida é chamada de {\bf medida de Lebesgue} sobre $\mathbb{R}$ e será denotada
simplesmente por $\lambda$. A $\sigma$-álgebra $\mathcal{M}$ é 
chamada de $\sigma$-álgebra de Lebesgue de $\mathbb{R}$.
 Utilizando o item $2)$ do 
Teorema da Extensão de Carathéodory podemos concluir também 
para quaisquer $a,b\in \mathbb{R}$ que 
\[
\lambda(\{a\})=0,
\quad
\lambda([a,b])=b-a,
\quad
\lambda\big((-\infty,a]\big)=+\infty
\quad
\text{e}
\quad 
\lambda\big([a,+\infty)\big)=+\infty.
\]

Observamos que ao longo do texto vamos utilizar
com mais frequência a restrição da 
medida de Lebesgue $\lambda$ à $\sigma(\mathcal{A})$
que é a $\sigma$-álgebra de Borel de $\mathbb{R}$ e
por simplicidade (e tradição)
vamos chamar esta restrição também de medida de Lebesgue
sobre $\mathbb{R}$.








\section[A Invariância por Translação da Medida de Lebesgue em $\mathbb{R}$]
{A Invariância por Translação da Medida de\\ Lebesgue em $\mathbb{R}$}


Seja $x\in\mathbb{R}$ fixado. Definimos a 
translação por $x$ de subconjunto $A\subset \mathbb{R}$ 
como sendo o conjunto 
\[
x\oplus A \equiv \{x+a\in \mathbb{R}: a\in A\}.
\]
O objetivo desta seção é provar que a medida de 
Lebesgue é invariante por translações, isto é, 
para quaisquer $x\in\mathbb{R}$ e $A\in\mathcal{M}$ fixados,
temos que $\lambda(A)=\lambda(x\oplus A)$.
Uma maneira interessante e que fornece uma 
perspectiva mais geral da afirmação feita acima
é considerar o conjunto $A$ como a {\bf imagem inversa} do 
conjunto $x\oplus A$ pela aplicação $T:\mathbb{R}\to\mathbb{R}$ 
dada por $T(y)=y-x$.
Neste contexto a igualdade 
$\lambda(A)=\lambda(x\oplus A)$ 
se expressa da seguinte forma 
$\lambda(A)=\lambda(T^{-1}(A))$. 


Uma pergunta natural sobre esta última maneira de expressar
a invariância translacional é: porque escrever
esta equação usando a imagem inversa ao invés de 
escrevê-la usando a imagem direta 
da função inversa de $T$ (o que parece ser mais simples). 
A resposta é que desta
maneira podemos definir o conceito de 
medida invariante para aplicações que não são 
inversíveis e que podem ser apenas mensuráveis.
De fato, dada qualquer aplicação $T:\mathbb{R}\to\mathbb{R}$
Borel mensurável a função de conjuntos definida 
sobre $\mathscr{B}(\mathbb{R})$ 
($\sigma$-álgebra de Borel de $\mathbb{R}$) por
\[
A\mapsto \lambda(T^{-1}(A))
\]
define sempre uma medida sobre os borelianos. 
Por causa desta observação podemos
definir que $\lambda$ (ou qualquer outra medida em $\mathbb{R}$)
é invariante por $T$ se a medida definida acima coincide 
com a medida $\lambda$. 


Antes de provar a invariância por translações  
da medida de Lebesgue vamos provar um fato 
relacionado que se refere  
a invariância por translações da medida exterior 
$\mathrm{Leb}^*$. 

\begin{teorema}
[Invariância por Translações de $\mathrm{Leb}^*$]
Para qualquer $x\in\mathbb{R}$ fixado e 
$E\in \mathcal{P}(\mathbb{R})$ temos que 
\[
\mathrm{Leb}^*(x\oplus E) 
=
\mathrm{Leb}^*(E).
\]
\end{teorema}

\begin{proof}
Seja $\bigcup_{j=1}^\infty I_j$ uma cobertura de $E$, 
onde cada $I_j\in\mathcal{A}$ e $\mathcal{A}$ é a álgebra 
de conjuntos de $\mathbb{R}$ 
definida em \eqref{algebra-conjuntos-R}. 
É fácil ver que  $\bigcup_{j=1}^\infty (x\oplus I_j)$
é uma cobertura de $x\oplus E$ e além do mais que 
$(x\oplus I_j) \in \mathcal{A}$. 
\begin{exercicio}
Mostre que as duas afirmações feitas acima são verdadeiras
e que 
$
\mathrm{Leb}(x\oplus I_j)
=
\mathrm{Leb}(I_j).
$
\end{exercicio}
Usando a definição da medida exterior
$\mathrm{Leb}^*$ em seguida, o exercício acima 
obtemos a seguinte estimativa 
\begin{align*}
\mathrm{Leb}^{*}(x\oplus E)
&\leq
\inf 
	\left\{
		\sum_{j=1}^\infty \mathrm{Leb}(x\oplus I_j):
		\ x\oplus E\subset \bigcup_{j=1}^\infty (x\oplus I_j)
	\right\}
\\
&=
\inf 
	\left\{
		\sum_{j=1}^\infty \mathrm{Leb}(I_j):
		\ E\subset \bigcup_{j=1}^\infty I_j
	\right\}
\\
&=
\mathrm{Leb}^{*}(E).
\end{align*}
Já que toda cobertura $\{R_j\}_{j\in\mathbb{N}}$ de $x\oplus E$
pode ser obtida por translação de uma cobertura 
$\{I_j\}_{j\in\mathbb{N}}$ de $E$, obtemos de maneira 
semelhante a que fizemos acima a cota 
$\mathrm{Leb}^{*}(E)\leq \mathrm{Leb}^{*}(x\oplus E)$ e
finalmente que  
$\mathrm{Leb}^{*}(E)=\mathrm{Leb}^{*}(x\oplus E)$.
\end{proof}










\begin{teorema}
[Invariância por Translações da Medida de Lebesgue]
Sejam $E$ um conjunto Lebesgue mensurável e $x\in\mathbb{R}$
fixados. Então $x\oplus E$ é Lebesgue mensurável e além do 
mais temos que 
\[
\lambda(x\oplus E)= \lambda(E).
\]
\end{teorema}

\begin{proof}
Pelo teorema anterior é suficiente mostrar que 
para todo conjunto Lebesgue mensurável $E$ temos 
que $x\oplus E$ é Lebesgue mensurável.
\begin{exercicio}
Sejam $A,B\subset \mathbb{R}$ subconjuntos arbitrários
e $x\in\mathbb{R}$ fixado. Mostre que as seguintes igualdades
são válidas
\[
(x\oplus A)\cap B 
= 
x\oplus 
\Big(A\cap \big((-x)\oplus B\big)\Big)
\quad\text{e}\quad
(x\oplus A^c) = (x\oplus A)^c.
\]
\end{exercicio}
Do teorema anterior e da primeira igualdade do exercício 
acima, para quaisquer que sejam $A,B\subset \mathbb{R}$ 
e $x\in\mathbb{R}$ fixados temos a seguinte igualdade  
\begin{equation}\label{eq-aux-invariancia-med-lebesgue}
\mathrm{Leb}^*\big((x\oplus A)\cap B\big)
= 
\mathrm{Leb}^*
\Big(A\cap \big((-x)\oplus B\big)\Big).
\end{equation}
Já que estamos assumindo que $E$ 
é um conjunto Lebesgue mensurável, então 
podemos afirmar que 
\[
\mathrm{Leb}^*\big((-x)\oplus A\big)
=
\mathrm{Leb}^*\Big( \big((-x)\oplus A\big) \cap E\Big)
+
\mathrm{Leb}^*\Big( \big((-x)\oplus A\big) \cap E^c\Big)
\]
Aplicando a identidade 
\eqref{eq-aux-invariancia-med-lebesgue} nas duas parcelas
do lado direito da igualdade acima, ficamos com 
\[
\mathrm{Leb}^*\big((-x)\oplus A\big)
=
\mathrm{Leb}^*\Big( \big(x\oplus E\big) \cap A\Big)
+
\mathrm{Leb}^*\Big( \big(x\oplus E^c\big) \cap A\Big)
\]
Usando a invariância por translações da medida exterior 
$\mathrm{Leb}^*$ e a segunda igualdade do exercício,
na segunda parcela acima, ficamos com 
\[
\mathrm{Leb}^*(A)
=
\mathrm{Leb}^*\big((-x)\oplus A\big)
=
\mathrm{Leb}^*\Big( \big(x\oplus E\big) \cap A\Big)
+
\mathrm{Leb}^*\Big( \big(x\oplus E\big)^c \cap A\Big).
\]
Como $A\subset \mathbb{R}$ é arbitrário segue da igualdade acima que 
$x\oplus E$ é Lebesgue mensurável.
\end{proof}







\begin{exercicio}
Sejam $x\in\mathbb{R}$ e $B$ um boreliano da reta fixados.
Mostre que $x\oplus B$ é um boreliano.
Em seguida, usando que 
$
	\mathrm{Leb}(A)=\mathrm{Leb}(x\oplus A),
$
para todo $A$ elemento da 
álgebra $\mathcal{A}$ definida em \eqref{algebra-conjuntos-R}
e o Teorema $\pi-\lambda$ de Dynkin,
mostre que $\lambda\big|_{\mathscr{B}(\mathbb{R})}$
é invariante por translações. 

\end{exercicio}