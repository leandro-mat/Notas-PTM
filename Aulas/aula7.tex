\chapter[Aula 7]{Independência e Leis Zero-Um}
\chaptermark{}

\section{Expansões Diádicas de Números Aleatórios Distribuídos Uniformemente}

Apresentamos nesta seção outro exemplo interessante
de v.a.'s independentes. Desta vez nosso espaço 
de probabilidade será 
$(\Omega,\F,\P)=\Big((0,1],\mathscr{B}((0,1]),\lambda\Big)$,
onde $\lambda$ é a medida de Lebesgue em $(0,1]$.

Para definir a v.a.'s que estamos interessados,
precisamos fazer primeiro algumas observações. 
Para cada $\omega\in (0,1]$ existem
números $d_n(\omega)\in\{0,1\}$, com $n\in\N$  
tais que 
	\[
		\omega = \sum_{n=1}^{\infty} \frac{d_n(\omega)}{2^n}.
	\] 
Observamos que os números $d_n(\omega)$ na igualdade 
acima não são unicamente determinados por $\omega$. Por exemplo, 
para $\omega=1/2$ temos 
	\[
		\frac{1}{2} = \frac{0}{2^1}+\frac{1}{2^2}+\frac{1}{2^3}+\ldots
		\qquad
		\text{e}
		\qquad
		\frac{1}{2}= \frac{1}{2^1}+\frac{0}{2^2}+\frac{0}{2^3}+\ldots.
	\]  
Um fato importante é que este é o único tipo 
de não unicidade que ocorre na expansão diádica de um 
número no intervalo $(0,1]$. 
Este fato permite fazer uma escolha apropriada
de forma que $d_n:\Omega\to\{0,1\}$ seja de fato 
uma função. Esta escolha é feita da seguinte maneira:
sempre que $\omega$ admitir mais de uma 
expansão diádica, 
seja $d_n(\omega)$ o $n$-ésimo digito da expansão 
diádica de $\omega$ que termina em uma sequência 
infinita de uns consecutivos.
Esta escolha no exemplo acima determina uma 
única expansão diádica para $1/2$ que é dada por $d_1(1/2)=0$ e
$d_n(1/2)=1$, para todo $n\geq 2$.

\begin{proposicao}
	Para todo $n\in\N$ a função $d_n:\Omega\to\{0,1\}$ 
	como definida acima é uma variável aleatória. 
\end{proposicao}



\begin{proof}
Como $d_n$ assume apenas dois valores, 
para provar a proposição basta mostrar 
que 
	\[
		\{d_n=0\}\in\mathscr{B}((0,1])
		\qquad
		\text{e}
		\qquad
		\{d_n=1\}\in\mathscr{B}((0,1]),
	\]
para todo $n\in\N$.
Já que $\{d_n=0\} = \{d_n=1\}^c$, é suficiente
mostrar que $\{d_n=1\}\in \mathscr{B}((0,1])$.



Para ajudar na compreensão do argumento mostramos 
qual é a ideia da prova em um 
caso mais simples, onde $n=1$. Neste caso 
temos 
	\begin{align*}
		\{d_n=1\}
		&=
		\left(
		\frac{1}{2^1}+\frac{0}{2^2}+\frac{0}{2^3}+\ldots
		\right.
		\ \ \mathbf{,} \ \ 
		\left.
		\frac{1}{2^1}+\frac{1}{2^2}+\frac{1}{2^3}+\ldots
		\right]
		\\[0.3cm]
		&=
		\left(\frac{1}{2},1\right]
		\in \mathscr{B}((0,1]).
	\end{align*}
Observe que o intervalo acima é aberto no extremo esquerdo
por causa da escolha que fizemos da expansão e portanto 
$1/2\in \{d_1=0\}$.

Para $n\geq 2$ temos que 
	\begin{align*}
		\{d_n=1\}
		&=
		\bigcup_{(u_1,\ldots,u_{n-1})\in\{0,1\}^{n-1}}
		\left(
		\frac{u_1}{2^1}+\frac{u_2}{2^2}+\ldots+\frac{u_{n-1}}{2^{n-1}}+\frac{1}{2^{n}}
		+\frac{0}{2^{n+1}}+\frac{0}{2^{n+2}}+
		\ldots
		\right.
		\\
		&\qquad\qquad\qquad
		\ \ \mathbf{,} 
		\left.
		\frac{u_1}{2^1}+\frac{u_2}{2^2}+\ldots+\frac{u_{n-1}}{2^{n-1}}+\frac{1}{2^{n}}
		+\frac{1}{2^{n+1}}+\frac{1}{2^{n+2}}+
		\ldots
		\right].
	\end{align*}
O lado direito da igualdade acima é uma união 
disjunta de $2^{n-1}$ intervalos e portanto este
conjunto pertence a $\mathscr{B}((0,1])$.

Por exemplo, 
	\[
	\{d_2=1\} 
	= \left( \frac{1}{4},\frac{1}{2} \right]
	\cup 
	\left( \frac{3}{4},1  \right].
	\]
\end{proof}




\begin{proposicao}
	Para todo $n\in\N$ temos que $\lambda(\{d_n=1\})=1/2$.
\end{proposicao}



\begin{proof}
Usando a representação obtida na proposição acima e 
a aditividade de $\lambda$ temos que 
%
	\begin{align*}
		\lambda(\{d_n=1\})
		&=
		\lambda
		\left(		
			\bigcup_{(u_1,\ldots,u_{n-1})\in\{0,1\}^{n-1}}
			\left(
				\frac{u_1}{2^1}
				+\frac{u_2}{2^2}
				+\ldots+\frac{u_{n-1}}{2^{n-1}}+\frac{1}{2^{n}}
				+\frac{0}{2^{n+1}}+\ldots
			\right.
		\right.
		\\
		&\qquad\qquad\qquad
		\ \ \mathbf{,} 
		\left.
			\left.
				\frac{u_1}{2^1}+\frac{u_2}{2^2}
				+\ldots+\frac{u_{n-1}}{2^{n-1}}+\frac{1}{2^{n}}
				+\frac{1}{2^{n+1}}+\frac{1}{2^{n+2}}
				+\ldots
			\right]
		\right).
		\\[0.8cm]
%
%	SEGUNDA IGUALDADE
%
		&=
		\sum_{(u_1,\ldots,u_{n-1})\in\{0,1\}^{n-1}}
		\lambda
		\left(		
			\left(
				\frac{u_1}{2^1}
				+\frac{u_2}{2^2}
				+\ldots+\frac{u_{n-1}}{2^{n-1}}+\frac{1}{2^{n}}
				+\frac{0}{2^{n+1}}+\ldots
			\right.
		\right.
		\\
		&\qquad\qquad\qquad
		\ \ \mathbf{,} 
		\left.
			\left.
				\frac{u_1}{2^1}+\frac{u_2}{2^2}
				+\ldots+\frac{u_{n-1}}{2^{n-1}}+\frac{1}{2^{n}}
				+\frac{1}{2^{n+1}}+\frac{1}{2^{n+2}}
				+\ldots
			\right]
		\right)
		\\[0.8cm]
%
%	TERCEIRA IGUALDADE
%
		&=
		\sum_{(u_1,\ldots,u_{n-1})\in\{0,1\}^{n-1}}
		\sum_{k=n+1}^{\infty} \frac{1}{2^k}
		=
		\sum_{(u_1,\ldots,u_{n-1})\in\{0,1\}^{n-1}}
		\frac{1}{2^{n+1}(1-1/2)}
		=\frac{1}{2}.
	\end{align*}
\end{proof}

















\begin{proposicao}
A sequência $\{d_n\}$ é uma sequência de variáveis aleatórias iid no
espaço de probabilidade
$\Big((0,1],\mathscr{B}((0,1]),\lambda\Big)$.
\end{proposicao}


\begin{proof}
Já que $d_n$ assume apenas os valores $0$ e $1$ segue da
proposição anterior que $\{d_n\}$ é identicamente distribuída.
Portanto basta provar que esta sequência é independente. 
Para isto é suficiente mostrar que para todo $n\in\N$ 
as variáveis aleatórias $\{d_1,\ldots,d_n\}$ são independentes.
\end{proof}