\chapter[Aula 7]{Independência e Leis Zero-Um}
\chaptermark{}

\section{Expansões Diádicas de Números Aleatórios Distribuídos Uniformemente}

Apresentamos nesta seção outro exemplo interessante
de v.a.'s independentes. Desta vez nosso espaço 
de probabilidade será 
$(\Omega,\F,\P)=\Big((0,1],\mathscr{B}((0,1]),\lambda\Big)$,
onde $\lambda$ é a medida de Lebesgue em $(0,1]$.

Para definir a v.a.'s que estamos interessados,
precisamos fazer primeiro algumas observações. 
Para cada $\omega\in (0,1]$ existem
números $d_n(\omega)\in\{0,1\}$, com $n\in\N$  
tais que 
	\[
		\omega = \sum_{n=1}^{\infty} \frac{d_n(\omega)}{2^n}.
	\] 
Observamos que os números $d_n(\omega)$ na igualdade 
acima não são unicamente determinados por $\omega$. Por exemplo, 
para $\omega=1/2$ temos 
	\[
		\frac{1}{2} = \frac{0}{2^1}+\frac{1}{2^2}+\frac{1}{2^3}+\ldots
		\qquad
		\text{e}
		\qquad
		\frac{1}{2}= \frac{1}{2^1}+\frac{0}{2^2}+\frac{0}{2^3}+\ldots.
	\]  
Um fato importante é que este é o único tipo 
de não unicidade que ocorre na expansão diádica de um 
número no intervalo $(0,1]$. 
Assim $d_n$ estará
bem definida como função 
se escolhemos sempre $d_n(\omega)$ como 
o $n$-ésimo digito da expansão diádica de $\omega$ 
que termina em uma sequência infinita de uns consecutivos.
No exemplo acima escolheríamos a primeira, isto é,
a expansão diádica de $1/2$ é dada por $d_1(1/2)=0$ e
$d_n(1/2)=1$, para todo $n\geq 2$.


