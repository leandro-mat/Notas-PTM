\chapter[Aula 7]{Independência e Leis Zero-Um}
\chaptermark{}

\section{Expansões Diádicas de Números Aleatórios Distribuídos Uniformemente}

Apresentamos nesta seção outro exemplo interessante
de v.a.'s independentes. Desta vez nosso espaço 
de probabilidade será 
$(\Omega,\F,\P)=\Big((0,1],\mathscr{B}((0,1]),\lambda\Big)$,
onde $\lambda$ é a medida de Lebesgue em $(0,1]$.

Para definir a v.a.'s que estamos interessados,
precisamos fazer primeiro algumas observações. 
Para cada $\omega\in (0,1]$ existem
números $d_n(\omega)\in\{0,1\}$, com $n\in\N$  
tais que 
	\[
		\omega = \sum_{n=1}^{\infty} \frac{d_n(\omega)}{2^n}.
	\] 
Observamos que os números $d_n(\omega)$ na igualdade 
acima não são unicamente determinados por $\omega$. Por exemplo, 
para $\omega=1/2$ temos 
	\[
		\frac{1}{2} = \frac{0}{2^1}+\frac{1}{2^2}+\frac{1}{2^3}+\ldots
		\qquad
		\text{e}
		\qquad
		\frac{1}{2}= \frac{1}{2^1}+\frac{0}{2^2}+\frac{0}{2^3}+\ldots.
	\]  
Um fato importante é que este é o único tipo 
de não unicidade que ocorre na expansão diádica de um 
número no intervalo $(0,1]$. 
Este fato permite fazer uma escolha apropriada
de forma que $d_n:\Omega\to\{0,1\}$ seja de fato 
uma função. Esta escolha é feita da seguinte maneira:
sempre que $\omega$ admitir mais de uma 
expansão diádica, 
seja $d_n(\omega)$ o $n$-ésimo digito da expansão 
diádica de $\omega$ que termina em uma sequência 
infinita de uns consecutivos.
Esta escolha no exemplo acima determina uma 
única expansão diádica para $1/2$ que é dada por $d_1(1/2)=0$ e
$d_n(1/2)=1$, para todo $n\geq 2$.

\begin{proposicao}
	Para todo $n\in\N$ a função $d_n:\Omega\to\{0,1\}$ 
	como definida acima é uma variável aleatória. 
\end{proposicao}



\begin{proof}
Como $d_n$ assume apenas dois valores, 
para provar a proposição basta mostrar 
que 
	\[
		\{d_n=0\}\in\mathscr{B}((0,1])
		\qquad
		\text{e}
		\qquad
		\{d_n=1\}\in\mathscr{B}((0,1]),
	\]
para todo $n\in\N$.
Já que $\{d_n=0\} = \{d_n=1\}^c$, é suficiente
mostrar que $\{d_n=1\}\in \mathscr{B}((0,1])$.



















\end{proof}





