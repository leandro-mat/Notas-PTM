\chapter[Aula 12]{Teorema de Radom-Nikodým}
\chaptermark{}


\section{O Teorema da Decomposição de Hahn}

Lembramos que uma carga em  $(\Omega,\F)$ é uma 
função de conjuntos $\lambda:\F\to\R$ tal que
$\lambda(\emptyset)=0$ e $\lambda$ 
é $\sigma$-aditiva, isto é, se $\{E_n\}$ é uma
sequência de conjuntos mutuamente disjuntos de $\F$ 
então 
\[
\lambda\left(\bigcup_{n=1}^{\infty}E_n\right)
=
\sum_{n=1}^{\infty} \lambda(E_n).
\]
Como já mencionamos o lado esquerdo da igualdade acima é 
independente de qualquer reordenamento que se
faça da coleção $\{E_n\}$ e portanto a série que aparece a 
direita é incondicionalmente somável. 

Exemplos importantes de cargas, são aquelas 
dadas por 
\[
\lambda(E)=\int_{E} f\, d\mu,
\]
onde $f\in L^1(\Omega,\F,\mu)$. A prova de que 
a função de conjuntos $\lambda$, definida acima, é uma 
carga foi dada no Lema \ref{lema-int-f-dmu-define-uma-carga}.

