\chapter[Aula 5]{Funções Mensuráveis e Reais e Complexas}
\chaptermark{}

\section{Funções Mensuráveis}

A partir de agora usamos a notação $\overline{\R}$ 
para denotar a reta estendida, isto é, o conjunto
$\overline{\R} = \R\cup\{-\infty,+\infty\}$. 
Observe que podemos estender naturalmente conceito 
de soma e produto  de números reais
para os seguintes pares de elementos 
de $\overline{\R}$:
\begin{enumerate}
	\item 
	para quaisquer $x,y\in \R$ a soma é a soma usual 
	e o mesmo para o produto. 
	
	\item para todo $x\in \overline{\R}$ temos 
	$0\cdot x=0$.
	
	\item Para todo $x\in \overline{\R}$ diferente de 
	$-\infty$ definimos $x+(+\infty)=+\infty$.

	\item Para todo $x\in \overline{\R}$ diferente de 
	$+\infty$ definimos $x+(-\infty)=-\infty$.
	
	\item 
	$\pm \infty \cdot \pm \infty = +\infty$ 
	e
	$\pm \infty \cdot \mp \infty = -\infty$.
	
\end{enumerate}

Não vamos nos preocupar, neste momento, em munir $\overline{\R}$ 
de uma topologia, mas vamos definir a $\sigma$-álgebra
de Borel de $\overline{\R}$,
como sendo a coleção formada pela reunião da
coleção $\mathscr{B}(\R)$ 
e de todos conjuntos da forma 
$B\cup\{-\infty\}$, $B\cup\{+\infty\}$ e $B\cup\{-\infty,+\infty\}$,
onde $B$ varia sobre todos os elementos de $\mathscr{B}(\R)$.
Esta coleção que acabamos de definir
é de fato uma $\sigma$-álgebra e será chamada de $\sigma$-álgebra
de borel de $\overline{\R}$ e denotada por $\mathscr{B}(\overline{\R})$.

\begin{exercicio}
	Mostre que a coleção $\mathscr{B}(\overline{\R})$ 
	é uma $\sigma$-álgebra.
\end{exercicio}

\begin{exercicio}
	Podemos ver $\overline{\R}$ como um conjunto totalmente ordenado se 
	consideramos a relação de ordem ``$<$'' obtida pela extensão natural 
	da relação de ordem em $\R$.
	Seja $\tau$ a topologia da ordem definida por ``$<$'' 
	em $\overline{\R}$. Mostre que a $\sigma$-álgebra gerada pelos 
	abertos de $\tau$ coincide com a coleção $\mathscr{B}(\overline{\R})$
	definida acima. 
	
\end{exercicio}










\subsection*{Funções a Valores Reais Mensuráveis}

Nesta subseção vamos considerar funções que
saem de um espaço mensurável $(\Omega,\F)$ e tomam 
valores em $\mathbb{R}$. O caso mais geral, onde
as funções assumem valores em $\overline{R}$ será
tratado na subseção seguinte.


\begin{definicao} 
Seja $(\Omega,\F)$ um espaço mensurável. 
Uma função $f:\Omega\to\R$ é dita $\F$-mensurável
se para todo número real $\alpha$ temos que
\[
	\{\omega \in\Omega : f(\omega)>\alpha \} \in \F.
\]
\end{definicao}

O próximo lema fornece três maneira alternativas de
definir funções mensuráveis. 

\begin{lema}
As seguintes afirmações são equivalentes para uma função 
$f:\Omega\to\R$.
\begin{enumerate}
	\item 
	Para todo $\alpha\in\R$ o conjunto 
	$A_{\alpha}\equiv\{\omega\in\Omega : f(\omega)>\alpha \} \in \F$.

	\item 
	Para todo $\alpha\in\R$ o conjunto 
	$B_{\alpha}\equiv\{\omega\in\Omega : f(\omega)\leq \alpha \} \in \F$.

	\item 
	Para todo $\alpha\in\R$ o conjunto 
	$C_{\alpha}\equiv\{\omega\in\Omega : f(\omega)\geq\alpha \} \in \F$.

	\item 
	Para todo $\alpha\in\R$ o conjunto 
	$D_{\alpha}\equiv\{\omega\in\Omega : f(\omega)<\alpha \} \in \F$.
	
\end{enumerate}
\end{lema}


\begin{proof}
	Já que $A_{\alpha}$ e $B_{\alpha}$ são complementares um do outro
	temos imediatamente que 1 e 2 são equivalentes. Pelo mesmo 
	motivo segue que 3 e 4 são equivalentes. Vamos assumir agora que 
	1 seja válido. Então temos que $A_{\alpha-1/n}\in \F$ para todo
	$n\in\N$. Já que 
	\[ 
		C_{\alpha} = \bigcap_{n=1}^{\infty} A_{\alpha} 
	\]
	temos que 1 implica 3. Por outro lado, a igualdade
	\[
		A_{\alpha} = \bigcup_{n=1}^{\infty} C_{\alpha+1/n}
	\]
	mostra que 3 implica 1.
\end{proof}



\begin{exemplo}\label{exemplo-func-const.mensuravel}
	Toda função constante $f:\Omega\to\R$ é mensurável.
	De fato, se $f(\omega)=c$ para todo $\omega\in\Omega$ 
	então temos que se $\alpha\geq c$ então 
	$\{\omega\in\Omega: f(\omega)>\alpha\} =\emptyset$.
	Por outro lado, se $\alpha<c$ então 
	$\{\omega\in\Omega: f(\omega)>\alpha\} =\Omega$.
\end{exemplo}





\begin{exemplo}
	Seja $(\Omega,\F)$ um espaço mensurável. 
	Se $E\in\F$ então a função $1_E:\Omega\to\R$, a função indicadora de $E$,
	é mensurável.
	Para ver que a afirmação é verdadeira basta notar que 
	$\{\omega\in\Omega: 1_{E}(\omega)>\alpha \}$ é igual $\Omega, E$
	ou $\emptyset$.
\end{exemplo}



\begin{exemplo}
	Considere o espaço mensurável $(\R,\mathscr{B}(\R))$. 
	Então qualquer função $f:\R\to\R$ contínua é 
	$\mathscr{B}(\R)$-mensurável.
	Basta notar que para todo $\alpha\in \R$ temos
	por continuidade que $\{x\in\R: f(x)>\alpha\}$ é 
	um aberto da reta e portanto pertence a $\sigma$-álgebra
	de borel de $\R$.
\end{exemplo}




\begin{exemplo}
	Considere o espaço mensurável $(\R,\mathscr{B}(\R))$ e 
	seja $f:\R\to\R$ uma função monótona não-decrescente.
	Para qualquer $\alpha\in\R$, 
	temos que $\{x\in\R: f(x)<\alpha\}$ ou é uma semi-reta
	da forma $(-\infty,a)$ ou $(-\infty,a]$ ou $\R$ ou $\emptyset$.
	E portanto todo função monótona não-crescente é borel 
	mensurável, isto é, $\mathscr{B}(\R)$-mensurável.
\end{exemplo}


Vamos mostrar na sequência que algumas operações algébricas
entre funções mensuráveis fornece também um função mensurável.

\begin{lema}\label{lema-operacoes-alg-com-func-mensuraveis}
	Sejam $(\Omega,\F)$ um espaço de medida e 
	$f,g:\Omega\to\R$ funções $\F$-mensuráveis 
	e $c$ uma constante real. Então as funções 
	\[
		cf,\ f^2,\ f+g,\ fg,\ |f|
	\]
 	são também funções $\F$-mensuráveis.
\end{lema} 


\begin{proof}
Vamos mostrar primeiro que $cf$ é mensurável. 
Se $c=0$ então $cf\equiv 0$ e a mensurabilidade de $cf$
segue do Exemplo \ref{exemplo-func-const.mensuravel}.
Se $c>0$, para todo $\alpha\in\R$ temos que
\[ 
	\{ \omega\in \Omega: cf(\omega)>\alpha\}
	=
	\left\{ \omega\in \Omega: f(\omega)>\frac{\alpha}{c} \right\}
	\in \F.
\]
O caso $\alpha<0$ é análogo.

Vamos provar agora que $f^2$ é $\F$-mensurável. 
Vamos dividir o argumento em 
dois casos, $\alpha<0$ e $\alpha \geq 0$.
No primeiro caso $\alpha<0$, temos que 
$\{ \omega\in \Omega: f^2(\omega)>\alpha\} = \Omega$
Se $\alpha\geq 0$ então 
\[ 
	\{ \omega\in \Omega: f^2(\omega)>\alpha\} 
	=
	\{ \omega\in \Omega: f(\omega)>\sqrt{\alpha}\} 
	\cup
	\{ \omega\in \Omega: f(\omega)<-\sqrt{\alpha}\}
	\in 
	\F. 
\]

Passamos agora para a mensurabilidade da soma. 
Por hipótese para qualquer número racional $r$
e qualquer que seja $\alpha\in\R$, temos que 
\[
	S_r
	\equiv 
	\{ \omega\in\Omega: f(\omega)>r \} 
		\cap 
		\{ \omega\in\Omega: g(\omega)>\alpha-r \}
	\in 
	\F.
\]
Já que 
\[
	\{ \omega\in\Omega: (f+g)(\omega)>\alpha \}
	=
	\bigcup_{r\in \Q} S_r
\]
é um conjunto $\F$-mensurável, segue que $f+g$ é mensurável.

Para mostrar que o produto $fg$ é mensurável, basta 
notar que 
\[ fg = \frac{1}{4}[(f+g)^2-(f-g)^2] \]
e usar o resultados provados acima.

Resta mostrar que $|f|$ é mensurável.
Novamente consideramos dois casos, $\alpha<0$ e 
$\alpha\geq 0$. No primeiro caso, $\alpha<0$, temos que 
$\{\omega\in\Omega: |f(\omega)|>\alpha\} = \Omega$.
Por outro lado, se $\alpha\geq 0$ então 
\[ 
	\{\omega\in\Omega: |f(\omega)|>\alpha\} 
	= 
	\{\omega\in\Omega: f(\omega)>\alpha\} 
	\cup
	\{\omega\in\Omega: f(\omega)<-\alpha\} .
\]
como ambos conjuntos do lado direito da igualdade 
acima pertencem a $\F$, segue que 
$\{\omega\in\Omega: |f(\omega)|>\alpha\} \in \F$ o que 
encerra a prova do lema.
\end{proof}


\subsection*{Partes Positiva e Negativa de Funções a Valores Reais}

A cada função $f:\Omega\to\mathbb{R}$ podemos associar duas funções 
{\bf não-negativas} denotadas por $f^+$ e $f^-$ ambas definidas em $\Omega$
por 
	\[
		f^+(\omega)= \sup\{f(\omega),0\} 
		\qquad
		\text{e}
		\qquad
		f^-(\omega)= \sup\{-f(\omega),0\} 
	\]
As funções $f^+$ e $f^-$ são chamadas, respectivamente 
de {\bf parte positiva} e {\bf parte negativa} de $f$.

Note que para qualquer que seja $\omega\in\Omega$, sempre temos
	\[
		f(\omega) = f^+(\omega)-f^{-}(\omega)
		\qquad
		\text{e}
		\qquad
		|f(\omega)| = f^+(\omega)+f^{-}(\omega).
	\]
	
	
\begin{lema}
	Seja $(\Omega,\F)$ um espaço de medida. Uma função 
	$f:\Omega\to\R$, é $\F$-mensurável se, e somente se,
	$f^+$ e $f^-$ são $\F$-mensuráveis.
\end{lema}

\begin{proof}
A prova deste lema é consequência direta das seguintes identidades:
\[
	f^{+} = \frac{1}{2}(|f|+f)
	\qquad
	\text{e}
	\qquad
	f^{-} = \frac{1}{2}(|f|-f).
\]
\end{proof}


Até o momento trabalhamos com o conceito de mensurabilidade em 
$\R$. Como frequentemente iremos trabalhar com sequências de 
funções e estaremos interessados em tomar, supremos, ínfimos,
limites e etc. é tecnicamente conveniente trabalhar com o 
conjunto $\overline{\R}$. Esta é a razão de introduzirmos 
na seção seguinte o conceito de mensurabilidade de uma função 
tomando valores na reta estendida. 






\section{Funções a Valores em $\overline{\R}$ Mensuráveis}

\begin{definicao}[Função Mensurável em $\overline{\R}$]
Seja $(\Omega,\F)$ um espaço de mensurável. Uma função 
$f:\Omega\to\overline{\R}$ é dita $\F$-mensurável, se para todo
$\alpha\in\R$ temos que 
$\{\omega\in\Omega:f(\omega)>\alpha\}\in\F$.
\end{definicao}


A coleção de todas as funções $\F$-mensuráveis 
tomando valores em $\overline{\R}$ é denotada por
$M(\Omega,\F)$. Observe que se $f\in M(\Omega,\F)$
então 
%
\[
\begin{array}{c}
	\displaystyle
	\{ \omega\in\Omega: f(\omega)=+\infty \}
	=
	\bigcap_{n=1}^{\infty} \{x\in\Omega: f(x)>n\}
	\\[0.3cm]
	\text{e}
	\\[0.3cm]
	\displaystyle
	\{ \omega\in\Omega: f(\omega)=-\infty \}
	=
	\left( 
		\bigcup_{n=1}^{\infty} \{x\in\Omega: f(x)>-n\}
	\right)^c
\end{array}
\]
são conjuntos $\F$-mensuráveis.



O próximo lema apresenta uma caracterização 
do conceito de mensurabilidade que acabamos de introduzir.
Ele será muito útil para provar mensurabilidade de funções
tomando valores em $\overline{\R}$. 


\begin{lema}\label{lema-truncamento-func-mensuravel-eh-mensuravel}
	Seja $(\Omega,\F)$ um espaço de medida. 
	Uma função $f:\Omega\to\overline{R}$ é 
	mensurável se, e somente se, os conjuntos 
	\[
	A=\{ \omega\in\Omega: f(\omega)=+\infty \}
	\quad
	\text
	\quad
	B=\{ \omega\in\Omega: f(\omega)=-\infty \}	
	\]
	são $\F$-mensuráveis e a função  
	$f_{1}:\Omega\to\overline{\R}$ dada por 
	\[
	f_1(\omega)=
		\begin{cases}
			f(x),&\text{se}\ x\notin A\cup B;
			\\
			0,&\text{se}\ x\in A\cup B.
		\end{cases}
	\]
	é $\F$-mensurável.
\end{lema}



\begin{proof}
Suponha que a função $f$ seja $\F$ mensurável. 
Como já observamos, se $f\in M(\Omega,\F)$ então $A$ e $B$ 
são $\F$-mensuráveis. Seja $\alpha\in \R$. Se $\alpha\geq 0$
então temos que 
\[
	\{\omega\in\Omega:f_1(\omega)>\alpha\}
	=
	\{\omega\in\Omega:f(\omega)>\alpha\}
	\setminus A.	
\] 
Se $\alpha<0$ então 
\[	
	\{\omega\in\Omega:f_1(\omega)>\alpha\}
	=
	\{\omega\in\Omega:f(\omega)>\alpha\}
	\cup B.	
\]
Portanto podemos concluir que $f_1$ é $\F$-mensurável.


Reciprocamente, suponha que $f_1\in M(\Omega,\F)$ e $A,B\in\F$.
Então para $\alpha\geq 0$ temos
\[
	\{\omega\in\Omega:f(\omega)>\alpha\}
	=
	\{\omega\in\Omega:f_1(\omega)>\alpha\}
	\cup A	
\] 
e se $\alpha<0$ temos que
\[
	\{\omega\in\Omega:f(\omega)>\alpha\}
	=
	\{\omega\in\Omega:f_1(\omega)>\alpha\}
	\setminus B,	
\] 
o que mostra que $f\in M(\Omega,\F)$.
\end{proof}



\begin{corolario}
	Seja $(\Omega,\F)$ um espaço de medida. Se $f\in M(\Omega,\F)$
	então para toda constante $c\in \R$ temos que 
	as funções $cf,\ f^2,\ |f|,\ f^{+}$ e $f^{-}\in M(\Omega,\F)$. 
\end{corolario} 

\begin{proof}
A prova é uma aplicação dos Lemas 
\ref{lema-operacoes-alg-com-func-mensuraveis} 
e
\ref{lema-truncamento-func-mensuravel-eh-mensuravel}.
Para exemplificar vamos mostrar que $f^2\in M(\Omega,\F)$. 
Vamos mostrar primeiro que $(f^2)_1=f_1f_1$.
Sejam $A$ e $B$ os conjuntos definidos no Lema 
\ref{lema-truncamento-func-mensuravel-eh-mensuravel}
com respeito a função $f$.
Se $\omega \notin A\cup B$, então 
$f(\omega)\in\R$ e assim $(f^2)_1(\omega)=f^2(\omega) = f_1(\omega)f_1(\omega)$.
Por outro lado, se $\omega\in A\cup B$ então temos que 
$(f^2)_1(\omega)= 0\cdot 0 = f_1(\omega)f_1(\omega)$.
Já que $(f^2)_1=f_1f_1$ segue do 
Lema \ref{lema-operacoes-alg-com-func-mensuraveis} 
que $(f^2)_1$ é mensurável. Observando que os conjuntos 
$A$ e $B$ para $f^2$ são $\F$-mensuráveis segue do 
Lema \ref{lema-truncamento-func-mensuravel-eh-mensuravel}
que $f^2$ é $\F$-mensurável.
\end{proof}



\begin{observacao}
Se $f$ e $g$ pertencem a $M(\Omega,\F)$, então a função 
soma $(f+g)(\omega)$ não está bem definida pela fórmula 
$(f+g)(\omega)=f(\omega)+g(\omega)$ nos seguintes conjuntos:
\[
\begin{array}{c}
	E_1=\{\omega\in\Omega: f(\omega)=+\infty\ \text{e}\ g(\omega)=-\infty\}
	\\
	\text{e}
	\\
	E_2=\{\omega\in\Omega: f(\omega)=-\infty\ \text{e}\ g(\omega)=+\infty\}.
\end{array}
\] 
Note que $E_1$ e $E_2\in \F$. Assim se definimos $(f+g)$ 
sobre a união destes conjuntos como sendo zero, 
então a função resultante é $\F$-mensurável. 
Após o lema seguinte vamos discutir sobre a mensurabilidade
de $fg$.
\end{observacao}



\begin{lema}\label{inf-sup-func-mensuraval-eh-mensuravel}
	Seja $\{f_n\}$ uma sequência de funções em $M(\Omega,\F)$.
	Defina as seguintes funções:
	\[	
	\begin{array}{cc}
	\displaystyle
	f(\omega) = \inf_{n\to\infty}f_n(\omega), & 	
	\displaystyle
	F(\omega) = \sup_{n\to\infty}f_n(\omega),
	%
	\\[0.5cm]
	\displaystyle
	f^*(\omega) = \liminf_{n\to\infty}f_n(\omega), & 	
	\displaystyle
	F^*(\omega) = \limsup_{n\to\infty}f_n(\omega).	
	\end{array}
	\]
Então $f,F,f^*$ e $F^*$ pertencem a $M(\Omega,\F)$.
\end{lema}


\begin{proof}
Para cada $n\in\N$ fixado, segue da 
hipótese de mensurabilidade de $f_n$ que 
o seguinte conjunto é $\F$-mensurável
\[
\{\omega\in\Omega: f_n(\omega)\geq \alpha\}
=
\bigcap_{k=1}^{\infty}
\left\{
	\omega\in\Omega: f_n(\omega)>\alpha-\frac{1}{k} 
\right\}. 
\]
Como as seguinte igualdade são válidas:
	\[	
	\{\omega\in\Omega: f(\omega)\geq \alpha\} 
	=
	\bigcap_{n=1}^{\infty}
	\left\{
		\omega\in\Omega: f_n(\omega)\geq \alpha 
	\right\}
	\]
e
 	\[
 	\{\omega\in\Omega: F(\omega)> \alpha\} 
	=
	\bigcup_{n=1}^{\infty}
	\left\{
		\omega\in\Omega: f_n(\omega)> \alpha 
	\right\}
	\]
temos que $f$ e $F$ são mensuráveis sempre que $f_n$
é mensurável para todo $n\in\N$.
Já que 
	\[	
	f^*(\omega)
	=
	\sup_{n\geq 1}
	\left\{ \inf_{m\geq n} f_{m}(\omega) \right\}
	\quad
	\text{e}
	\quad
	F^*(\omega)
	=
	\inf_{n\geq 1}
	\left\{ \sup_{m\geq n} f_{m}(\omega) \right\}
	\]
basta aplicar duas vezes consecutivas o resultado que acabamos 
para concluir que ambas $f^*$ e $F^*$ são $\F$-mensuráveis.
\end{proof}




\begin{corolario}\label{cor-lim-mensuravel-eh-mensuravel}
Se $\{f_n\}$ é uma sequência em $M(\Omega,\F)$ tal que
$f_n(\omega)\to f(\omega)$ para todo $\omega\in\Omega$,
então $f\in M(\Omega,\F)$.
\end{corolario} 

\begin{proof}
	Basta observar que 
		\[
			f(\omega) 
			=\lim_{n\to\infty} f_n(\omega)
			=\liminf_{n\to\infty} f_n(\omega).
		\]
\end{proof}


Voltamos a questão da mensurabilidade do produto $fg$,
quando ambas $f$ e $g\in M(\Omega,\F)$. Para isto 
vamos introduzir a noção de truncamento de uma função em 
$M(\Omega,\F)$. Para cada $n\in\N$ considere a função 
$f_n$ que é chamada de um truncamento de $f$, definida por 
	\[
		f_n(\omega)
		=
		\begin{cases}
			f(\omega),&\text{se}\ |f(\omega)|\leq n;
			\\
			n,&\text{se}\ f(\omega)>n;
			\\
			-n,&\text{se}\ f(\omega)<-n.
		\end{cases}
	\]
\begin{exercicio}
	Sejam $f\in M(\Omega,\F)$ e $n\in\N$. 
	Mostre que se $f_n$ é um truncamento de $f$, 
	como definido acima, então $f_n\in M(\Omega,\F)$. 
\end{exercicio}

Se $f_n$ e $g_m$ são truncamentos de $f$ e $g$, respectivamente
segue do Lema \ref{lema-operacoes-alg-com-func-mensuraveis}
que o produto $f_n\cdot g_m$ é mensurável. Já que 
para todo $\omega\in \Omega$ temos 
	\[
		f(\omega)g_m(\omega) 
		=
		\lim_{n\to\infty} f_n(\omega)g_m(\omega)
	\]
segue do Corolário \ref{cor-lim-mensuravel-eh-mensuravel}
que $fg_m \in M(\Omega,\F)$. Observando que
	\[
		(fg)(\omega) 
		=
		f(\omega)g(\omega)
		=
		\lim_{m\to\infty} f(\omega)g_m(\omega)
	\]	
outra aplicação do Corolário \ref{cor-lim-mensuravel-eh-mensuravel}
mostra finalmente que $fg\in M(\Omega,\F)$.




\begin{definicao}[Sequência Monótona de Funções]
 Uma sequência $\{f_n\}$ em $M(\Omega,\F)$ é dita
 monótona não-decrescente, se para todo $\omega\in\Omega$
 fixado, temos que 
 	\[
 		f_n(\omega)\leq f_{n+1}(\omega)
 		\quad
 		\forall n\in\N.
 	\]
Analogamente definimos sequências monótonas não-crescentes.
\end{definicao}




\begin{definicao}[Função Simples]
Seja $(\Omega,\F)$ um espaço mensurável. 
Uma função $f:\Omega\to\overline{R}$ é dita 
simples se assume no máximo uma quantidade
finita de valores, isto é, $\# f(\Omega)<\infty$.

\end{definicao}



O Corolário \ref{cor-lim-mensuravel-eh-mensuravel} afirma que 
limite pontual de funções em $M(\Omega,\F)$ é uma função em 
$M(\Omega,\F)$.
Assumindo apenas que $f$ é não-negativa e mensurável
mostramos no próximo lema um fato ainda mais
forte. Sob tais condições vamos ver que  
$f$ é limite pontual de alguma sequência 
monótona não-decrescente $f_n$ em $M(\Omega,\F)$,
além do mais a demostração fornece uma maneira de construir
tal sequência.

\begin{teorema}\label{teo:aproximacao-monotona-por-func-simples}
	Seja $(\Omega,\F)$ um espaço de medida. 
	Se $f:\Omega\to [0,+\infty]$
	é mensurável, existe uma sequência de funções 
	simples $f_n:\Omega\to[0,\infty]$ tal que
	para todo $\omega\in\Omega$ temos
	$0\leq f_1(\omega)\leq f_2(\omega)\leq \ldots\leq f(\omega)$ 
	e 
	$f_n(\omega)\to f(\omega)$.
\end{teorema}


\begin{proof} 
Fixe $n\in\mathbb{N}$ e considere uma partição do 
intervalo $[0,n)$ em intervalos semi-abertos de
mesmo comprimento igual a $1/2^n$, da seguinte 
forma:
\[
[0,n)
=
\left[ 0,\frac{1}{2^n} \right) 
\cup
\left[ \frac{1}{2^n},\frac{2}{2^n} \right) 
\cup
\ldots
\cup 
\left[ \frac{k}{2^n},\frac{k+1}{2^n} \right) 
\cup
\ldots
\cup
\left[ \frac{n2^n-1}{2^n}, \frac{n2^n}{2^n} \right).
\]
Estes $n2^n$ intervalos 
junto com o intervalo $[n,+\infty]$ formam uma partição 
de $[0,+\infty]$. 

Esta partição induz uma partição em $\Omega$
que é determinada pelas pré-imagens destes conjuntos 
por $f$, isto é, para $k\in\mathbb{N}$ satisfazendo 
$0\leq k\leq n2^{n}-1$, defina
\[
E_{n}^{k}=
f^{-1}\left( \left[\frac{k}{2^n},\frac{k+1}{2^n} \right) \right)
\qquad \text{e} \qquad
E_n^{n2^n}=f^{-1}\big( [n,+\infty]\big).
\]
Com auxilio desta partição, 
definimos uma sequência de funções simples
não-negativas $\{f_n\}$, onde 
\[
f_n =\sum_{k=0}^{n2^n-1} \frac{k}{2^n} 1_{E_{n}^{k}} + n 1_{F_n}.
\]

Segue diretamente da definição dos conjuntos 
$E_{n}^{k}$ que $f_n(\omega)\leq f_{n+1}(\omega)$ 
para todo $\omega\in\Omega$.
Note também que se $f(\omega)<n$ 
então $0\leq f(\omega)-f_n(\omega)<\frac{1}{2^n}$. 
Por outro lado, se 
$f(\omega)=+\infty$ temos que $f_n(\omega)= n$,
logo $f_n(\omega)\to+\infty$.
Portanto para todo $\omega\in\Omega$ temos que 
$f_n(\omega)\to f(\omega)$, quando $n\to\infty$.
\end{proof}

\begin{observacao} 
O leitor mais atento deve ter notado que 
a última estimativa apresentada na demonstração acima 
implica que a convergência das funções simples $f_n$ 
para $f$ é uniforme nos conjuntos 
onde $f$ é limitada, isto é, dado $M>0$ 
seja $\Omega_M=\{\omega\in\Omega; f(\omega)\leq M\}$ então a sequência 
$f_n|_{\Omega_M}\to f|_{\Omega_M}$ uniformemente.
\end{observacao}

 







\section{Funções Mensuráveis a Valores Complexos}

É de grande importância na Teoria da Probabilidade 
o estudo de algumas funções a valores complexos
como, por exemplo, as chamadas funções 
características. 
Por isto, vamos introduzir nesta pequena seção a noção
de mensurabilidade para tais funções. Observe que 
se $f$ é uma função complexa definida em $\Omega$,
isto é, $f:\Omega\to\C$ é uma função, então existem
duas outras funções reais unicamente determinadas
chamadas respectivamente, de partes real e imaginária
denotadas por $u,v:\Omega\to\R$ tais que 
	\[
	f(\omega) =u(\omega)+iv(\omega),
	\]
onde $u(\omega)=\text{Re}(f(\omega))$ e
$v(\omega)=\text{Im}(f(\omega))$.
\begin{definicao}
[Função Complexa Mensurável]
Seja $(\Omega,\F)$ um espaço de medida. 
Uma função complexa $f:\Omega\to\C$ é dita 
mensurável se, e somente se, suas partes 
real e imaginária são funções mensuráveis.
\end{definicao} 

\begin{exercicio}
Mostre que somas, produtos e limites de funções complexas 
mensuráveis é uma função complexa mensurável.
\end{exercicio}








