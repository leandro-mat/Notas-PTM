\chapter[Aula 5]{Funções Mensuráveis e Variáveis Aleatórias}
\chaptermark{}

\section{Funções Mensuráveis}

A partir de agora usamos a notação $\overline{\R}$ 
para denotar a reta estendida, isto é, o conjunto
$\overline{\R} = \R\cup\{-\infty,+\infty\}$. 
Observe que podemos estender naturalmente conceito 
de soma e produto  de números reais
para os seguintes pares de elementos 
de $\overline{\R}$:
\begin{enumerate}
	\item 
	para quaisquer $x,y\in \R$ a soma é a soma usual 
	e o mesmo para o produto. 
	
	\item para todo $x\in \overline{\R}$ temos 
	$0\cdot x=0$.
	
	\item Para todo $x\in \overline{\R}$ diferente de 
	$-\infty$ definimos $x+(+\infty)=+\infty$.

	\item Para todo $x\in \overline{\R}$ diferente de 
	$+\infty$ definimos $x+(-\infty)=-\infty$.
\end{enumerate}

Não vamos nos preocupar, neste momento, em munir $\overline{\R}$ 
de uma topologia, mas vamos definir a $\sigma$-álgebra
de Borel de $\overline{\R}$,
como sendo a coleção formada pela reunião da
coleção $\mathscr{B}(\R)$ 
e de todos conjuntos da forma 
$B\cup\{-\infty\}$, $B\cup\{+\infty\}$ e $B\cup\{-\infty,+\infty\}$,
onde $B$ varia sobre todos os elementos de $\mathscr{B}(\R)$.
Esta coleção que acabamos de definir
é de fato uma $\sigma$-álgebra e será chamada de $\sigma$-álgebra
de borel de $\overline{\R}$ e denotada por $\mathscr{B}(\overline{\R})$.

\begin{exercicio}
	Mostre que a coleção $\mathscr{B}(\overline{\R})$ 
	é uma $\sigma$-álgebra.
\end{exercicio}

\begin{exercicio}
	Podemos ver $\overline{\R}$ como um conjunto totalmente ordenado se 
	consideramos a relação de ordem ``$<$'' obtida pela extensão natural 
	da relação de ordem em $\R$.
	Seja $\tau$ a topologia da ordem definida por ``$<$'' 
	em $\overline{\R}$. Mostre que a $\sigma$-álgebra gerada pelos 
	abertos de $\tau$ coincide com a coleção $\mathscr{B}(\overline{\R})$
	definida acima. 
	
\end{exercicio}










\subsection*{Funções a Valores Reais Mensuráveis}

Nesta subseção vamos considerar funções que
saem de um espaço mensurável $(\Omega,\F)$ e tomam 
valores em $\mathbb{R}$. O caso mais geral, onde
as funções assumem valores em $\overline{R}$ será
tratado na subseção seguinte.


\begin{definicao} 
Seja $(\Omega,\F)$ um espaço mensurável. 
Uma função $f:\Omega\to\R$ é dita $\F$-mensurável
se para todo número real $\alpha$ temos que
\[
	\{\omega \in\Omega : f(\omega)>\alpha \} \in \F.
\]
\end{definicao}

O próximo lema fornece três maneira alternativas de
definir funções mensuráveis. 

\begin{lema}
As seguintes afirmações são equivalentes para uma função 
$f:\Omega\to\R$.
\begin{enumerate}
	\item 
	Para todo $\alpha\in\R$ o conjunto 
	$A_{\alpha}\equiv\{\omega\in\Omega : f(\omega)>\alpha \} \in \F$.

	\item 
	Para todo $\alpha\in\R$ o conjunto 
	$B_{\alpha}\equiv\{\omega\in\Omega : f(\omega)\leq \alpha \} \in \F$.

	\item 
	Para todo $\alpha\in\R$ o conjunto 
	$C_{\alpha}\equiv\{\omega\in\Omega : f(\omega)\geq\alpha \} \in \F$.

	\item 
	Para todo $\alpha\in\R$ o conjunto 
	$D_{\alpha}\equiv\{\omega\in\Omega : f(\omega)<\alpha \} \in \F$.
	
\end{enumerate}
\end{lema}


\begin{proof}
	Já que $A_{\alpha}$ e $B_{\alpha}$ são complementares um do outro
	temos imediatamente que 1 e 2 são equivalentes. Pelo mesmo 
	motivo segue que 3 e 4 são equivalentes. Vamos assumir agora que 
	1 seja válido. Então temos que $A_{\alpha-1/n}\in \F$ para todo
	$n\in\N$. Já que 
	\[ 
		C_{\alpha} = \bigcap_{n=1}^{\infty} A_{\alpha} 
	\]
	temos que 1 implica 3. Por outro lado, a igualdade
	\[
		A_{\alpha} = \bigcup_{n=1}^{\infty} C_{\alpha+1/n}
	\]
	mostra que 3 implica 1.
\end{proof}



\begin{exemplo}\label{exemplo-func-const.mensuravel}
	Toda função constante $f:\Omega\to\R$ é mensurável.
	De fato, se $f(\omega)=c$ para todo $\omega\in\Omega$ 
	então temos que se $\alpha\geq c$ então 
	$\{\omega\in\Omega: f(\omega)>\alpha\} =\emptyset$.
	Por outro lado, se $\alpha<c$ então 
	$\{\omega\in\Omega: f(\omega)>\alpha\} =\Omega$.
\end{exemplo}





\begin{exemplo}
	Seja $(\Omega,\F)$ um espaço mensurável. 
	Se $E\in\F$ então a função $1_E:\Omega\to\R$, a função indicadora de $E$,
	é mensurável.
	Para ver que a afirmação é verdadeira basta notar que 
	$\{\omega\in\Omega: 1_{E}(\omega)>\alpha \}$ é igual $\Omega, E$
	ou $\emptyset$.
\end{exemplo}



\begin{exemplo}
	Considere o espaço mensurável $(\R,\mathscr{B}(\R))$. 
	Então qualquer função $f:\R\to\R$ contínua é 
	$\mathscr{B}(\R)$-mensurável.
	Basta notar que para todo $\alpha\in \R$ temos
	por continuidade que $\{x\in\R: f(x)>\alpha\}$ é 
	um aberto da reta e portanto pertence a $\sigma$-álgebra
	de borel de $\R$.
\end{exemplo}




\begin{exemplo}
	Considere o espaço mensurável $(\R,\mathscr{B}(\R))$ e 
	seja $f:\R\to\R$ uma função monótona não-decrescente.
	Para qualquer $\alpha\in\R$, 
	temos que $\{x\in\R: f(x)<\alpha\}$ ou é uma semi-reta
	da forma $(-\infty,a)$ ou $(-\infty,a]$ ou $\R$ ou $\emptyset$.
	E portanto todo função monótona não-crescente é borel 
	mensurável, isto é, $\mathscr{B}(\R)$-mensurável.
\end{exemplo}


Vamos mostrar na sequência que algumas operações algébricas
entre funções mensuráveis fornece também um função mensurável.

\begin{lema}
	Sejam $(\Omega,\F)$ um espaço de medida e 
	$f,g:\Omega\to\R$ funções $\F$-mensuráveis 
	e $c$ uma constante real. Então as funções 
	\[
		cf,\ f^2,\ f+g,\ fg,\ |f|
	\]
 	são também funções $\F$-mensuráveis.
\end{lema} 


\begin{proof}
Vamos mostrar primeiro que $cf$ é mensurável. 
Se $c=0$ então $cf\equiv 0$ e a mensurabilidade de $cf$
segue do Exemplo \ref{exemplo-func-const.mensuravel}.
Se $c>0$, para todo $\alpha\in\R$ temos que
\[ 
	\{ \omega\in \Omega: cf(\omega)>\alpha\}
	=
	\left\{ \omega\in \Omega: f(\omega)>\frac{\alpha}{c} \right\}
	\in \F.
\]
O caso $\alpha<0$ é análogo.

Vamos provar agora que $f^2$ é $\F$-mensurável. 
Vamos dividir o argumento em 
dois casos, $\alpha<0$ e $\alpha \geq 0$.
No primeiro caso $\alpha<0$, temos que 
$\{ \omega\in \Omega: f^2(\omega)>\alpha\} = \Omega$
Se $\alpha\geq 0$ então 
\[ 
	\{ \omega\in \Omega: f^2(\omega)>\alpha\} 
	=
	\{ \omega\in \Omega: f(\omega)>\sqrt{\alpha}\} 
	\cup
	\{ \omega\in \Omega: f(\omega)<-\sqrt{\alpha}\}
	\in 
	\F. 
\]

Passamos agora para a mensurabilidade da soma. 
Por hipótese para qualquer número racional $r$
e qualquer que seja $\alpha\in\R$, temos que 
\[
	S_r
	\equiv 
	\{ \omega\in\Omega: f(\omega)>r \} 
		\cap 
		\{ \omega\in\Omega: g(\omega)>\alpha-r \}
	\in 
	\F.
\]
Já que 
\[
	\{ \omega\in\Omega: (f+g)(\omega)>\alpha \}
	=
	\bigcup_{r\in \Q} S_r
\]
é um conjunto $\F$-mensurável, segue que $f+g$ é mensurável.

Para mostrar que o produto $fg$ é mensurável, basta 
notar que 
\[ fg = \frac{1}{4}[(f+g)^2-(f-g)^2] \]
e usar o resultados provados acima.

Resta mostrar que $|f|$ é mensurável.
Novamente consideramos dois casos, $\alpha<0$ e 
$\alpha\geq 0$. No primeiro caso, $\alpha<0$, temos que 
$\{\omega\in\Omega: |f(\omega)|>\alpha\} = \Omega$.
Por outro lado, se $\alpha\geq 0$ então 
\[ 
	\{\omega\in\Omega: |f(\omega)|>\alpha\} 
	= 
	\{\omega\in\Omega: f(\omega)>\alpha\} 
	\cup
	\{\omega\in\Omega: f(\omega)<-\alpha\} .
\]
como ambos conjuntos do lado direito da igualdade 
acima pertencem a $\F$, segue que 
$\{\omega\in\Omega: |f(\omega)|>\alpha\} \in \F$ o que 
encerra a prova do lema.
\end{proof}


\subsection*{Partes Positiva e Negativa de Funções a Valores Reais}

A cada função $f:\Omega\to\mathbb{R}$ podemos associar duas funções 
{\bf não-negativas} denotadas por $f^+$ e $f^-$ ambas definidas em $\Omega$
por 
	\[
		f^+(\omega)= \sup\{f(\omega),0\} 
		\qquad
		\text{e}
		\qquad
		f^-(\omega)= \sup\{-f(\omega),0\} 
	\]
As funções $f^+$ e $f^-$ são chamadas, respectivamente 
de {\bf parte positiva} e {\bf parte negativa} de $f$.

Note que para qualquer que seja $\omega\in\Omega$, sempre temos
	\[
		f(\omega) = f^+(\omega)-f^{-}(\omega)
		\qquad
		\text{e}
		\qquad
		|f(\omega)| = f^+(\omega)+f^{-}(\omega).
	\]
	
	
\begin{lema}
	Seja $(\Omega,\F)$ um espaço de medida. Uma função 
	$f:\Omega\to\R$, é $\F$-mensurável se, e somente se,
	$f^+$ e $f^-$ são $\F$-mensuráveis.
\end{lema}

\begin{proof}
A prova deste lema é consequência direta das seguintes identidades:
\[
	f^{+} = \frac{1}{2}(|f|+f)
	\qquad
	\text{e}
	\qquad
	f^{-} = \frac{1}{2}(|f|-f).
\]
\end{proof}


Até o momento trabalhamos com o conceito de mensurabilidade em 
$\R$. Como frequentemente iremos trabalhar com sequências de 
funções e estaremos interessados em tomar, supremos, ínfimos,
limites e etc. é tecnicamente conveniente trabalhar com o 
conjunto $\overline{\R}$. Esta é a razão de introduzirmos 
na seção seguinte o conceito de mensurabilidade de uma função 
tomando valores na reta estendida. 






\section{Funções a Valores em $\overline{\R}$ Mensuráveis}

Em preparação.