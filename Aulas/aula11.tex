\chapter[Aula 11]{Teorema da Existência de Kolmogorov}
\chaptermark{}



\section{Regularidade de Medidas Definidas 
em $\mathscr{B}(\mathbb{R})$.}

Seja $(\Omega,\tau)$ um espaço topológico e 
$\mathscr{B}(\tau)$ a $\sigma$-álgebra gerada
pela topologia $\tau$, em outras palavras
a $\sigma$-álgebra gerada pelos conjuntos abertos.  
Nesta seção vamos estar
interessados no caso especial em que $\Omega=\mathbb{R}^n$
e $\tau$ é a topologia usual do espaço Euclideano $n$-dimensional,
isto é, a topologia induzida pela norma Euclideana. 
Neste caso teremos $\mathscr{B}(\tau)=\mathscr{B}(\mathbb{R}^n)$.

\begin{definicao}[Medidas de Borel Regular]\label{def-regularidade-medida-borel}
Seja $(\Omega,\tau)$ um espaço topológico. 
Uma medida de Borel em $\Omega$ é uma medida
definida sobre a $\sigma$-álgebra de Borel de $\Omega$, i.e.,  
$\mu:\mathscr{B}(\tau)\to [0,+\infty]$.
Uma medida de Borel $\mu$ em $\Omega$ é dita
{\bf regular} quando satisfaz as seguintes condições
para todo $E\in \mathscr{B}(\tau)$: 
\begin{enumerate}
\item  
$
\mu(E) 
= 
\inf \{\mu(A): E\subseteq A,\ \text{com}\ A\ \text{aberto em}\ \Omega  \}
$;

\item
$
\mu(E) 
= 
\sup \{\mu(K): K\subseteq E,\ \text{com}\ K\ \text{compacto em}\ \Omega  \}
$;

\item 
$\mu(K)<+\infty$ para todo $K$ compacto em $\Omega$.
\end{enumerate}
\end{definicao}


No que segue mostramos que toda medida de Borel finita
definida sobre $\mathbb{R}^n$ é regular. 
Na verdade esta afirmação pode ser provada em
um contexto um pouco mais geral, onde consideramos medidas
definidas sobre espaços métricos completos
com a $\sigma$-álgebra e topologias adequados.  

\begin{teorema}\label{teo-regularidade-med-borel}
Se $\mu$ é uma medida de Borel sobre $\mathscr{B}(\mathbb{R}^n)$
tal que $\mu(\mathbb{R}^n)<+\infty$ então $\mu$ é regular.
\end{teorema}



\begin{proof}
Considere a seguinte coleção 
\[
\mathscr{C}:=
\{
E\subset \mathbb{R}^n: \ 
\text{as condições 1-3 da Definição 
\ref{def-regularidade-medida-borel} são satisfeitas }
\}.
\]
Vamos mostrar que $\mathscr{C}$ é uma $\sigma$-álgebra
e também que $\mathscr{C}$ contém a coleção de todos os 
abertos de $\mathbb{R}^n$ o que nos permite concluir que
$\mathscr{B}(\mathbb{R}^n)\subseteq \mathscr{C}$.


Começamos provando que $\mathscr{C}$ é uma $\sigma$-álgebra.
Claramente $\emptyset\in\mathscr{C}$. 
Seja $\{E_n\}_{n\in\mathbb{N}}$ uma coleção de subconjuntos de
$\mathscr{C}$. Vamos mostrar que $E=\cup_{n\geq 1} E_n\in \mathscr{C}$.
Fixe $\varepsilon>0$. Usando que $\mu$ é uma medida finita 
e a definição de regularidade, 
para cada $n\in\mathbb{N}$, podemos encontrar um aberto 
$A_n$, com $E_n\subseteq A_n$ e um compacto $K_n$, com
$K_n\subseteq E_n$ tais que
\[
\mu(A_n-K_n) < \frac{\varepsilon}{2^n}.
\] 

A sequência $F_n = \cup_{m=1}^n E_m$ é uma sequência
crescente tal que $\cup_{n\geq 1}F_n =E$, portanto
segue da continuidade da medida $\mu$ que
\[
\lim_{n\to\infty} \mu\left(F_n \right)
=
\lim_{n\to\infty} \mu\left(\bigcup_{m=1}^n E_m \right)
=
\mu(E).
\]
Já que $\mu(E)<+\infty$ existe $N\in\mathbb{N}$ tal que 
\[ 
\mu(E)-\mu(F_N)<\varepsilon.
\]
Seja $K=K_1\cup K_2\cup\ldots\cup K_N$. Pela definição de 
$\{K_n\}_{n\in\mathbb{N}}$ temos que $K$ é um conjunto compacto de $\mathbb{R}^n$
e também que $K\subseteq E$. Usando propriedades das operações 
de conjuntos, a $\sigma$-aditividade de $\mu$ e a definição de $A_n$ e $K_n$
temos que 
\[
\mu\left(F_N - K \right)
\leq
\mu\left(\bigcup_{m=1}^N (E_m - K_m)  \right)
\leq
\mu\left(\bigcup_{m=1}^N (A_m - K_m)  \right)
\leq
\sum_{m=1}^N \frac{\varepsilon}{2^m}
<
\varepsilon.
\]
Usando as duas estimativas em destaque acima podemos 
verificar que   
\begin{align*}
\mu(E)-\mu(K)
=
\mu(E)-\mu(F_N)+\mu(F_N)-\mu(K)
<
2\varepsilon.
\end{align*}
Já que $K\subset E$, a desigualdade acima que mostra
a condição 2 da Definição \ref{def-regularidade-medida-borel}
é satisfeita para $E$. Como $\mu$ é uma medida finita sobre
$\mathbb{R}^n$ segue que $E$ satisfaz 
a condição 3 da Definição \ref{def-regularidade-medida-borel}.



Vamos verificar que a condição 2 também satisfeita. 
Para isto defina $A=\cup _{n\geq 1}A_n$. O conjunto $A$
é aberto, $A$ contém $E$ e além do mais pela definição de $A_n$
e $K_n$ temos a seguinte desigualdade 
\[
\mu(A)-\mu(E)
\leq 
\sum_{n=1}^{\infty} \mu(A_n-E_n)
\leq 
\sum_{n=1}^{\infty} \mu(A_n-K_n)
<
\varepsilon,
\]
o que implica que a condição 2 é satisfeita. Portanto 
podemos concluir que $E \in \mathscr{C}$.

Vamos verificar agora que $\mathscr{C}$ é fechada para 
complementação. Sejam $E\in\mathscr{C}$ e $\varepsilon>0$.
Então existe um compacto $K$ e um aberto $U$ tais que 
$K\subseteq E\subseteq A$ com $\mu(A-K)<\varepsilon$. Tomando 
complementares temos $A^c\subseteq E^c\subseteq K^c$.
Note que $K^c-A^c=A-K$ e portanto 
$\mu(K^c-A^c)=\mu(A-K)<\varepsilon$.
Claramente $K^c$ é um aberto de $\mathbb{R}^n$ e portanto
podemos concluir da estimativa acima que 
$\mu(K^c)-\mu(E^c)=\mu(K^c-E^c)<\mu(K^c-A^c)<\varepsilon$
e assim que a condição 1 da 
Definição \ref{def-regularidade-medida-borel}
é satisfeita para $E^c$.

Por outro lado não é possível afirmar que 
o conjunto $A^c$ seja compacto, embora 
seja fechado. Para corrigir este problema 
observamos que $\mathbb{R}^n$ pode ser escrito 
como união de uma sequencia crescente de bolas fechadas, 
$\{\overline{B(0,n)}\}_{n\in\mathbb{N}}$ que são compactos. 
Portanto temos da continuidade da medida $\mu$ que 
$\mu( \overline{B(0,n)} )\to \mu(\mathbb{R}^n)$, 
quando $n\to\infty$. Logo dado $\varepsilon >0$
existe $N\in\mathbb{N}$ tal que 
$\mu(\mathbb{R}^n)-\mu( \overline{B(0,N)} ) <\varepsilon$.
Definimos $\widetilde{K} =A^c\cap \overline{B(0,N)}$.
Já que  $\widetilde{K}$ é dado como interseção de um 
fechado com um compacto segue que $\widetilde{K}$ é 
compacto. Além do mais temos que 
$\widetilde{K}\subseteq A^c$.
Já que o conjunto $A^c-\widetilde{K}$ 
está contido no complementar de $\overline{B(0,N)}$
temos que $\mu(A^c-\widetilde{K})<\varepsilon$.
No parágrafo anterior já havíamos observado que 
$A^c\subseteq E^c\subseteq K^c$ e 
$\mu(K^c-A^c)=\mu(A-K)<\varepsilon$.
Portanto podemos concluir que $\mu(E^c-A^c)<\varepsilon$ e 
\[ 
\mu(E^c)-\mu(\widetilde{K})
=
\mu(E^c-\widetilde{K})
=
\mu(E^c-A^c)
+
\mu(A^c-\widetilde{K})
<
2\varepsilon.
\]
Como a desigualdade acima implica na condição 2 
da Definição \ref{def-regularidade-medida-borel}
para $E^c$ segue que $\mathscr{C}$ é fechada 
para complementação e isto completa a prova de 
que $\mathscr{C}$ é uma $\sigma$-álgebra.

Vamos mostrar que todo aberto $E\subset \mathbb{R}^n$ 
pertence a coleção $\mathscr{C}$. Claramente 
a condição 3 da Definição \ref{def-regularidade-medida-borel}
é satisfeita pois, $\mu$ é finita. A condição 1
da Definição \ref{def-regularidade-medida-borel} 
também é satisfeita pois, podemos 
tomar $A=E$. Resta mostrar que a condição 2 
da Definição \ref{def-regularidade-medida-borel} 
é satisfeita. Já que $E$ é aberto, para cada $n\in\mathbb{N}$ 
temos que o conjunto 
\[
F_n = 
\left\{
x\in E: d(E^c,x)\geq \frac{1}{n}
\right\},
\ \text{onde}\ 
d(E^c,x) = \inf\{\|y-x\|: y\in E^c\}
\]
é um um fechado de $\mathbb{R}^n$. 
Note que $F_n\subset F_{n+1}$ e 
também que $\cup_{n\geq 1} F_n =E$. Pela continuidade
de $\mu$ podemos afirmar que $\mu(F_n) \uparrow \mu(E)$. 
Se fosse possível garantir que $F_n$ é compacto 
para todo $n\in\mathbb{N}$ então teríamos provado 
a condição 2. 
Isto é de fato verdadeiro quando $E$ é um conjunto limitado.
Para o caso geral, tomamos $K_n = F_n\cap \overline{B(0,n)}$.
Note que $K_n$ é compacto para todo $n\in\mathbb{N}$ e também 
que $K_n\subset K_{n+1}$ e  $\cup_{n\geq 1} K_n = E$. 
Assim segue da continuidade de $\mu$ que 
$\mu(K_n)\uparrow \mu(E)$ e portanto 
a condição 2 da Definição \ref{def-regularidade-medida-borel}
se verifica para todo $E$ aberto. 
Isto conclui a prova de que todo aberto de $\mathbb{R}^n$
pertence a $\mathscr{C}$. 


Como $\mathscr{C}$ é uma 
$\sigma$-álgebra que contém os abertos, segue que 
$\mathscr{C}$ contém $\mathscr{B}(\mathbb{R}^n)$ e isto 
completa a prova que toda medida de Borel finita em $\mathbb{R}^n$
é regular.   
\end{proof}












\section[Teorema de Kolmogorov para Produtos Cartesianos Enumeráveis]
{Teorema de Kolmogorov para Produtos\\ Cartesianos Enumeráveis}

Na Aula \ref{Lei-Zero-Um-Kolmogorov-Enumeravel}
apresentamos a prova do Teorema da Existência 
de Kolmogorov em uma de suas versões mais simples.
Isto foi feito na seção chamada 
{\it A Medida Produto em $\{0,1,\ldots,n\}^{\mathbb{N}}$}.
Nesta seção vamos mostrar como estender a construção 
apresentada na seção mencionada acima, para o espaço
$\mathbb{R}^{\mathbb{N}}$. O produto cartesiano infinito 
$\mathbb{R}^{\mathbb{N}}$ pode ser pensado como o espaço
de todas as sequências de números reais $(x_1,x_2,\ldots)$
ou equivalentemente como o espaço 
$\{f:\mathbb{N}\to\mathbb{R}: f\ \text{é uma função} \}$.
É muito comum também denotar este espaço por 
$
\mathbb{R}\times\mathbb{R}
\times\ldots\times\mathbb{R}\times\ldots
\equiv 
\mathbb{R}^{\mathbb{N}}
$.

Para cada $n\in\mathbb{N}$ e $E\in\mathscr{B}(\mathbb{R}^n)$
fixados considere o seguinte subconjunto 
$
E\times\mathbb{R}\times\mathbb{R}\times\ldots 
\subseteq 
\mathbb{R}^{\mathbb{N}}.
$
Os conjuntos desta forma serão chamados de
conjuntos cilíndricos ou cilindros finito dimensionais.
A coleção de todos os conjuntos 
desta forma é chamada de coleção dos conjuntos cilíndricos
de $\mathbb{R}^{\mathbb{N}}$. A $\sigma$-álgebra gerada
pelo conjuntos cilíndricos será chamada de $\sigma$-álgebra
produto de $\mathbb{R}^{\mathbb{N}}$ e será denotada por 
$\mathscr{B}\big(\mathbb{R}^{\mathbb{N}}\big)$.

As famílias de medidas de probabilidade que aparecem 
no Teorema de Kolmogorov são tais que para cada $n\geq 1$
temos uma medida de Borel em $\mathbb{R}^n$ satisfazendo 
a seguinte condição de consistência 
\[
\mathbb{P}_{n+1}(E\times\mathbb{R}) = \mathbb{P}_n(E),
\ \forall \ E\in\mathscr{B}(\mathbb{R}^n).
\]
Nossa tarefa é mostrar que podemos 
construir uma medida $\mathbb{P}$ definida sobre 
$\mathscr{B}(\mathbb{R}^{\mathbb{N}})$ tal que 
para qualquer Boreliano $E\subset\mathbb{R}^n$ 
temos 
\[
\mathbb{P}(E\times\mathbb{R}\times\mathbb{R}\times\ldots)
=
\mathbb{P}_n(E).
\] 

Vamos usar a notação $\mathcal{F}_n$ para indicar 
a coleção de todos os conjuntos cilíndricos da 
forma $E\times\mathbb{R}\times\mathbb{R}\times\ldots$
com $E\in\mathscr{B}(\mathbb{R}^n)$. É fácil ver que
$\mathcal{F}_n$ é uma álgebra de conjuntos para 
cada $n\in\mathbb{N}$. Também podemos verificar que 
$\mathcal{F}_1\subseteq \mathcal{F}_2\subseteq \ldots$ 

Uma vez que a coleção $\cup_{n\geq 1}\mathcal{F}_n$ 
é uma união crescente de $\sigma$-álgebras podemos 
concluir que esta união tem estrutura de {\bf álgebra},
mas certamente {\bf não} é uma $\sigma$-álgebra.

Qualquer $\sigma$-álgebra que contém a álgebra
$\cup_{n\geq 1}\mathcal{F}_n$ contém todos os 
cilindros da forma 
$\mathbb{R}\times\mathbb{R}\times\ldots\times 
E\times\mathbb{R}\times\mathbb{R}\times\ldots$
para qualquer $E\in\mathscr{B}(\mathbb{R})$ 
e por esta razão chamamos 
$
\mathscr{B}(\mathbb{R}^{\mathbb{N}})
=
\sigma\big(\cup_{n\geq 1}\mathcal{F}_n \big)
$ 
da $\sigma$-álgebra
produto no produto cartesiano infinito $\mathbb{R}^{\mathbb{N}}$.

Antes de passarmos ao enunciado e a prova do Teorema de Kolmogorov
propriamente dito, vamos apresentar, 
na sequência, um resultado 
(Teorema \ref{teo-intersecao-cilindros-teo-kolmogorov} ) 
de natureza técnica que 
será crucial na prova do 
Teorema de Existência de Kolmogorov no caso de 
produtos infinitos enumeráveis. 

\subsubsection{Interseção de Cilindros Encaixantes 
em $\mathbb{R}^{\mathbb{N}}$.}

Um dos ingredientes chaves na prova do Teorema 
de Kolmogorov é um fato técnico que será usado
durante a prova do teorema de Kolmogorov 
para garantir que a interseção de uma certa sequência 
decrescente de cilindros de $\mathbb{R}^{\mathbb{N}}$
é não-vazia. 
Enunciamos este resultado de maneira precisa abaixo.

\begin{teorema}\label{teo-intersecao-cilindros-teo-kolmogorov}
Para cada $n\in\mathbb{N}$ seja $C_n$ um conjunto 
compacto não-vazio de $\mathbb{R}^n$.
Suponha que estes conjuntos satisfaçam as seguintes
condições: para cada $n\in\mathbb{N}$, 
$(x_1,\ldots,x_n,x_{n+1})\in C_{n+1}$
implica $(x_1,\ldots,x_n)\in C_{n}$.
Então existe uma sequência 
$(x_1,x_2,\ldots)\in\mathbb{R}^{\mathbb{N}}$ tal 
que $(x_1,\ldots,x_n)\in C_n$ para todo $n\in\mathbb{N}$.
\end{teorema}

\begin{proof}
Fixados dois números naturais $m$ e $n$ satisfazendo 
$1\leq m\leq n$ defina a projeção
$\pi^{n}_{m}:\mathbb{R}^n\to\mathbb{R}^m$ por 
$\pi^{n}_{m}(x_1,\ldots,x_m,\ldots,x_n)=(x_1,\ldots,x_m)$.
Já que $\pi^{n}_{m}$ é uma aplicação contínua e 
$C_n$ é um compacto não-vazio (por hipótese) podemos concluir que 
$\pi^{n}_{m}(C_n)$ é um subconjunto compacto não-vazio 
de $\mathbb{R}^m$.

Afirmamos que para cada $m\in\mathbb{N}$ fixado, 
a coleção de subconjuntos compactos 
de $\mathbb{R}^m$ dada por 
$\{\pi^{k}_{m}(C_k): k\geq m\}$, determina uma 
sequência decrescente de compactos de $\mathbb{R}^m$,
isto é,
\begin{equation}\label{continencia-pikm-Cm}
\ldots \subseteq 
\pi^{m+2}_{m}(C_{m+2})\subseteq
\pi^{m+1}_{m}(C_{m+1})\subseteq \pi_{m}^m(C_m).
\end{equation}
Vamos provar a afirmação. Fixe naturais $m,n$ tais que 
$m\leq n$ e tome um ponto qualquer 
$x =(x_1,\ldots,x_{n+1})\in C_{n+1}$. Por hipótese 
podemos afirmar que $x'=(x_1,\ldots,x_n)\in C_n$. 
Logo 
$
\pi^{n+1}_{m}(x)
=(x_1,\ldots,x_m)
=\pi^{n}_{m}(x')
\subseteq \pi^{n}_{m}(C_n)
$.
Portanto todo ponto de $\pi^{n+1}_{m}(C_{n+1})$ pertence 
a $\pi^{n}_{m}(C_n)$ e isto prova a afirmação.

Pelo Teorema de Cantor 
a sequência de subconjuntos compactos encaixantes 
da reta dada por 
\begin{equation}
\label{continencia2-pikm-Cm}
\ldots \subseteq 
\pi_{1}^{3}(C_3)
\subseteq
\pi^{2}_{1}(C_2)
\subseteq 
\pi^{1}_{1}(C_1) 
=
C_1
\end{equation}
tem interseção não vazia.
Seja $x_1$ um ponto arbitrário nesta interseção.
Por construção $x_1\in \pi_{1}^{n}(C_n)$ 
para todo $n\geq 1$. Pela definição da projeção
$\pi_{1}^{n}$ sabemos que     
existe pelo menos um ponto 
$(y_2,\ldots,y_{n})\in \mathbb{R}^{n-1}$ 
tal que $(x_1,y_2,\ldots,y_{n})\in C_n$.
Esta observação mostra que o conjunto 
$C_n(x_1)
\equiv 
\{ 
(y_2,\ldots,y_{n})\in\mathbb{R}^{n-1}
: (x_1,y_2,\ldots,y_{n})\in C_n
\}$ 
é não vazio.
Note que 
$
C_{n}(x_1) 
= 
(\pi_{n}^{1})^{-1}(x_1)\cap C_n.
$
Já que o conjunto unitário $\{x_1\}$ é fechado
segue da continuidade de $\pi_{n}^{1}$ 
que o conjunto 
$C_{n}(x_1)=(\pi_{n}^{1})^{-1}(x_1)\cap C_n$ é a interseção
de um fechado com um compacto e portanto $C_{n}(x_1)$
é compacto.
Além do mais $C_2(x_1), C_3(x_1),\ldots$
satisfaz a mesma condição que satisfaz a 
sequência original $C_1,C_2,\ldots$, isto é, 
para todo $(y_2,\ldots,y_{n+1})\in C_{n+1}(x_1)$
temos que $(y_2,\ldots,y_n)\in C_{n}(x_1)$.
De fato, se $(y_2,\ldots,y_{n+1})\in C_{n+1}(x_1)$
segue da definição de $C_{n+1}(x_1)$ que 
$(x_1,y_2,\ldots,y_{n+1})\in C_{n+2}$. 
De \eqref{continencia-pikm-Cm} segue que 
$(x_1,y_2,\ldots,y_n)\in C_n$ 
logo $(y_2,\ldots,y_{n})\in C_{n}(x_1)$.
Já que $\{C_n(x_1)\}_{n\geq 2}$ possui a propriedade
de $\{C_n\}_{n\geq 1}$ usada para provar 
a continência \eqref{continencia2-pikm-Cm}, 
então podemos garantir que \eqref{continencia2-pikm-Cm}
é também verdadeira para a sequência $\{C_n(x_1)\}_{n\geq 2}$
no lugar de $\{C_n\}_{n\geq 1}$. 
Segue desta sequência de continências,
da compacidade dos elementos da sequência 
$\{C_n(x_1)\}_{n\geq 2}$ e 
do Teorema de Cantor que existe pelo menos 
um ponto $x_2\in\mathbb{R}$ tal que $(x_1,x_2)\in \pi^{n}_2(C_n)$
para todo $n\geq 2$. Procedendo uma indução formal é fácil 
ver que podemos definir uma sequência $x_1,x_2,x_3\ldots$ 
tal que para todo $1\leq m\leq n$ temos 
$(x_1,x_2,\ldots,x_m)\in \pi^{n}_{m}(C_n)$.
Em particular, $(x_1,x_2,\ldots,x_n)\in \pi^{n}_{n}(C_n)=C_n$
para todo $n\geq 1$.
\end{proof}



\subsubsection{O Teorema de Existência de Kolmogorov - Caso Enumerável}

\begin{teorema}
Seja $\{\mathbb{P}_n\}_{n\in\mathbb{N}}$ 
uma família de medidas de probabilidade, 
tal que para cada $n\in\mathbb{N}$ a
medida de probabilidade $\mathbb{P}_n$ é uma medida definida 
sobre $\mathscr{B}(\mathbb{R}^{n})$. 
Suponha que a família $\{\mathbb{P}_n\}_{n\in\mathbb{N}}$  
satisfaz as 
\textbf{condições de consistência de Kolmogorov}, i.e.,
\[
\mathbb{P}_{n+k}(E\times\mathbb{R}^k)
=
\mathbb{P}_n(E)
\quad
\begin{array}{c}
\text{para todo}\ n,k\in\mathbb{N}\ \text{e}\ 
\text{todo boreliano}\ E\subseteq \mathbb{R}^n.
\end{array}
\]
Então existe uma única medida de probabilidade 
$\mathbb{P}$ definida sobre 
$\mathscr{B}\big(\mathbb{R}^{\mathbb{N}}\big)$
tal que para qualquer que seja $n\geq 1$ e 
$E\in \mathscr{B}(\mathbb{R}^{n})$ temos 
a seguinte igualdade 
\[
\mathbb{P}(E\times\mathbb{R}\times\mathbb{R}\times\ldots)
=
\mathbb{P}_n(E).
\]
\end{teorema}




\begin{proof}
Seja $\mathcal{F}_n$ 
a coleção de todos os conjuntos cilíndricos da 
forma $E\times\mathbb{R}\times\mathbb{R}\times\ldots$
com $E\in\mathscr{B}(\mathbb{R}^n)$
e considere a função de conjuntos 
$\mathbb{P}':\cup_{n\geq 1}\mathcal{F}_n\to [0,1]$
tal que para cada $E\in\mathscr{B}(\mathbb{R}^n)$ 
\[
\mathbb{P}'(E\times\mathbb{R}\times\mathbb{R}\times\ldots)
=
\mathbb{P}_n(E).
\] 
A hipótese de consistência da família $\{\mathbb{P}_n\}_{n\in\mathbb{N}}$
garante que $\mathbb{P}'$ está bem definida. 
A ideia da prova é primeiro mostrar que $\mathbb{P}'$ é uma medida na álgebra 
$\cup_{n\geq 1}\mathcal{F}_n$. Em seguida, aplicar o 
Teorema de Extensão de Carathéodory para estender $\mathbb{P}'$
a toda $\sigma$-álgebra produto, de forma que esta 
extensão coincida com $\mathbb{P}'$ 
nos conjuntos cilíndricos.

Primeiro observamos que $\mathbb{P}'(\emptyset)=0$ 
e que $\mathbb{P}'(\mathbb{R}^{\mathbb{N}})=1$.
Observamos que $\mathbb{P}'$ é finitamente aditiva.
De fato, para quaisquer $E,F\in \cup_{n\geq 1}\mathcal{F}_n$ 
existe $k\in\mathbb{N}$ tal que $E,F\in \mathcal{F}_{k}$
e portanto $E$ e $F$ são da forma 
\[
E= E'\times\mathbb{R}\times\mathbb{R}\times\ldots
F= F'\times\mathbb{R}\times\mathbb{R}\times\ldots
\]
para alguns $E',F'\in\mathscr{B}(\mathbb{R}^k)$.
Supondo que $E$ e $F$ são disjuntos, podemos concluir que
$E'$ e $F'$ são disjuntos e portanto 
\[
\mathbb{P}'(E\cup F)
=
\mathbb{P}_{k}(E'\cup F')
=
\mathbb{P}_{k}(E')+\mathbb{P}_{k}(F')
=
\mathbb{P}'(E)+\mathbb{P}'(F).
\]

Já que $\mathbb{P}'$ é finitamente aditiva, para provar 
que esta medida é $\sigma$-aditiva na álgebra 
$\cup_{n\geq 1}\mathcal{F}_n$ é suficiente mostrar que 
se $E_1,E_2,\ldots $ forma uma sequência decrescente de 
conjuntos cilíndricos com 
$\lim_{n\to\infty}\mathbb{P}'(E_n)>0$ então 
$\cap_{n\geq 1}E_n \neq \emptyset$.


Já que estamos assumindo que 
$E_1,E_2,\ldots $ a existência de 
real  $\alpha= \lim_{n\to\infty}\mathbb{P}'(E_n)$
está garantida pela monotonicidade de $\mathbb{P}'$.

A menos de repetição e renumeração de alguns conjuntos 
cilíndricos na coleção $E_1,E_2,\ldots $ 
podemos sem perda de generalidade assumir que 
$E_n\in\mathcal{F}_n$, isto é, 
$E_n=E_n'\times\mathbb{R}\times\mathbb{R}\times\ldots$,
para algum Boreliano $E_n'$. 
A condição $E_{n+1}\subseteq E_{n}$ implica que se
$(y_1,\ldots, y_n,y_{n+1})\in E'_{n+1}$ então 
$(y_1,\ldots, y_n)\in E'_{n}$. 
Por monotonicidade temos que 
\[
\mathbb{P}'(E_n) = \mathbb{P}_{n}(E'_n)\geq \alpha.
\]
Já que $\mathbb{P}_n$ são medidas de probabilidade em 
$\mathscr{B}(\mathbb{R}^n)$ segue do 
Teorema \ref{teo-regularidade-med-borel} que existe
um conjunto compacto $K'_n\subset E'_n$, tal que 
$\mathbb{P}_{n}(E'_n-K_n)<\alpha 2^{-n-1}$.
Defina $K_n = K'_n\times \mathbb{R}\times\mathbb{R}\times\ldots$
e seja $C_n = K_1\cap K_2\cap\ldots\cap K_n$. Para todo $n\geq 1$ 
temos que $C_n\in \cup_{n\geq 1}\mathcal{F}_n$,  
\[
\ldots\subseteq C_3\subseteq C_2\subseteq C_1
\]
e usando a monotonicidade da sequência $\{E_n\}_{n\in\mathbb{N}}$ 
temos que 
\[
\mathbb{P}'(E_n-C_n)
=
\mathbb{P}'\left( \bigcap_{k=1}^n E_k- \bigcap_{k=1}^n K_k \right)
\leq
\sum_{k=1}^{n} \mathbb{P}_{k}(E'_k-K'_k)
<
\frac{\alpha}{2}.
\]
\end{proof}