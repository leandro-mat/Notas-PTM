\chapter[Aula 11]{Teorema da Existência de Kolmogorov}
\chaptermark{}

\section[Teorema de Kolmogorov para Produtos Cartesianos Enumeráveis]
{Teorema de Kolmogorov para Produtos\\ Cartesianos Enumeráveis}

Na Aula \ref{Lei-Zero-Um-Kolmogorov-Enumeravel}
apresentamos a prova do Teorema da Existência 
de Kolmogorov em uma de suas versões mais simples.
Isto foi feito na seção chamada 
{\it A Medida Produto em $\{0,1,\ldots,n\}^{\mathbb{N}}$}.
Nesta seção vamos mostrar como estender a construção 
apresentada na seção mencionada acima, para o espaço
$\mathbb{R}^{\mathbb{N}}$. O produto cartesiano infinito 
$\mathbb{R}^{\mathbb{N}}$ pode ser pensado como o espaço
de todas as sequências de números reais $(x_1,x_2,\ldots)$
ou equivalentemente como o espaço 
$\{f:\mathbb{N}\to\mathbb{R}: f\ \text{é uma função} \}$.
É muito comum também denotar este espaço por 
$
\mathbb{R}\times\mathbb{R}
\times\ldots\times\mathbb{R}\times\ldots
\equiv 
\mathbb{R}^{\mathbb{N}}
$.

Para cada $n\in\mathbb{N}$ e $E\in\mathscr{B}(\mathbb{R}^n)$
fixados consideramos o seguinte subconjunto 
$
E\times\mathbb{R}\times\mathbb{R}\times\ldots 
\subseteq 
\mathbb{R}^{\mathbb{N}}.
$
A coleção de todos os conjuntos 
desta forma é chamada de coleção dos conjuntos cilíndricos
de $\mathbb{R}^{\mathbb{N}}$. A $\sigma$-álgebra gerada
pelo conjuntos cilíndricos será chamada de $\sigma$-álgebra
produto de $\mathbb{R}^{\mathbb{N}}$ e será denotada por 
$\mathscr{B}\big(\mathbb{R}^{\mathbb{N}}\big)$.


\begin{teorema}
Seja $\{\mathbb{P}_n\}_{n\in\mathbb{N}}$ 
uma família de medidas de probabilidade, 
tal que para cada $n\in\mathbb{N}$ a
medida de probabilidade $\mathbb{P}_n$ é uma medida definida 
sobre $\mathscr{B}(\mathbb{R}^{n})$. 
Suponha que a família $\{\mathbb{P}_n\}_{n\in\mathbb{N}}$  
satisfaz as 
\textbf{condições de consistência de Kolmogorov}, i.e.,
\[
\mathbb{P}_{n+k}(E\times\mathbb{R}^k)
=
\mathbb{P}_n(E)
\quad
\begin{array}{c}
\text{para todo}\ n,k\in\mathbb{N}\ \text{e}\ 
\text{todo boreliano}\ E\subseteq \mathbb{R}^n.
\end{array}
\]
Então existe uma única medida de probabilidade 
$\mathbb{P}$ definida sobre 
$\mathscr{B}\big(\mathbb{R}^{\mathbb{N}}\big)$
tal que para qualquer que seja $n\geq 1$ e 
$E\in \mathscr{B}(\mathbb{R}^{n})$ temos 
a seguinte igualdade 
\[
\mathbb{P}(E\times\mathbb{R}\times\mathbb{R}\times\ldots)
=
\mathbb{P}_n(E).
\]
\end{teorema}
