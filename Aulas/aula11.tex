\chapter[Aula 11]{Teorema da Existência de Kolmogorov}
\chaptermark{}

\section[Teorema de Kolmogorov para Produtos Cartesianos Enumeráveis]
{Teorema de Kolmogorov para Produtos\\ Cartesianos Enumeráveis}

Na Aula \ref{Lei-Zero-Um-Kolmogorov-Enumeravel}
apresentamos a prova do Teorema da Existência 
de Kolmogorov em uma de suas versões mais simples.
Isto foi feito na seção chamada 
{\it A Medida Produto em $\{0,1,\ldots,n\}^{\mathbb{N}}$}.
Nesta seção vamos mostrar como estender a construção 
apresentada na seção mencionada acima, para o espaço
$\mathbb{R}^{\mathbb{N}}$. O produto cartesiano infinito 
$\mathbb{R}^{\mathbb{N}}$ pode ser pensado como o espaço
de todas as sequências de números reais $(x_1,x_2,\ldots)$
ou equivalentemente como o espaço 
$\{f:\mathbb{N}\to\mathbb{R}: f\ \text{é uma função} \}$.
É muito comum também denotar este espaço por 
$
\mathbb{R}\times\mathbb{R}
\times\ldots\times\mathbb{R}\times\ldots
\equiv 
\mathbb{R}^{\mathbb{N}}
$.

Para cada $n\in\mathbb{N}$ e $E\in\mathscr{B}(\mathbb{R}^n)$
fixados consideramos o seguinte subconjunto 
$
E\times\mathbb{R}\times\mathbb{R}\times\ldots 
\subseteq 
\mathbb{R}^{\mathbb{N}}.
$
A coleção de todos os conjuntos 
desta forma é chamada de coleção dos conjuntos cilíndricos
de $\mathbb{R}^{\mathbb{N}}$. A $\sigma$-álgebra gerada
pelo conjuntos cilíndricos será chamada de $\sigma$-álgebra
produto de $\mathbb{R}^{\mathbb{N}}$ e será denotada por 
$\mathscr{B}\big(\mathbb{R}^{\mathbb{N}}\big)$.


\subsubsection{Interseção de Cilindros Encaixantes 
em $\mathbb{R}^{\mathbb{N}}$.}

Um dos ingredientes chaves na prova do Teorema 
de Kolmogorov é um fato bastante técnico que 
garante que a interseção de uma sequência 
decrescente de cilindros de $\mathbb{R}^{\mathbb{N}}$
é não-vazia. Enunciamos este resultado de maneira precisa
abaixo.

\begin{teorema}
Para cada $n\in\mathbb{N}$ seja $C_n$ um conjunto 
compacto não-vazio de $\mathbb{R}^n$.
Suponha que estes conjuntos satisfaçam as seguintes
condições: para cada $n\in\mathbb{N}$, 
$(x_1,\ldots,x_n,x_{n+1})\in C_{n+1}$
implica $(x_1,\ldots,x_n)\in C_{n}$.
Então existe uma sequência 
$(x_1,x_2,\ldots)\in\mathbb{R}^{\mathbb{N}}$ tal 
que $(x_1,\ldots,x_n)\in C_n$ para todo $n\in\mathbb{N}$.
\end{teorema}

\begin{proof}
Fixados dois números naturais $m$ e $n$ satisfazendo 
$1\leq m\leq n$ defina a projeção
$\pi^{n}_{m}:\mathbb{R}^n\to\mathbb{R}^m$ por 
$\pi^{n}_{m}(x_1,\ldots,x_m,\ldots,x_n)=(x_1,\ldots,x_m)$.
Já que $\pi^{n}_{m}$ é uma aplicação contínua e 
$C_n$ é um compacto não-vazio (por hipótese) podemos concluir que 
$\pi^{n}_{m}(C_n)$ é um subconjunto compacto não-vazio 
de $\mathbb{R}^m$.

Afirmamos que para cada $m\in\mathbb{N}$ fixado, 
a coleção de subconjuntos compactos 
de $\mathbb{R}^m$ dada por 
$\{\pi^{k}_{m}(C_k): k\geq m\}$, determina uma 
sequência decrescente de compactos de $\mathbb{R}^m$,
isto é,
\begin{equation}\label{continencia-pikm-Cm}
\ldots \subseteq 
\pi^{m+2}_{m}(C_{m+2})\subseteq
\pi^{m+1}_{m}(C_{m+1})\subseteq \pi_{m}^m(C_m).
\end{equation}
Vamos provar a afirmação. Fixe naturais $m,n$ tais que 
$m\leq n$ e tome um ponto qualquer 
$x =(x_1,\ldots,x_{n+1})\in C_{n+1}$. Por hipótese 
podemos afirmar que $x'=(x_1,\ldots,x_n)\in C_n$. 
Logo 
$
\pi^{n+1}_{m}(x)
=(x_1,\ldots,x_m)
=\pi^{n}_{m}(x')
\subseteq \pi^{n}_{m}(C_n)
$.
Portanto todo ponto de $\pi^{n+1}_{m}(C_{n+1})$ pertence 
a $\pi^{n}_{m}(C_n)$ e isto prova a afirmação.

Pelo Teorema de Cantor 
a sequência de subconjuntos compactos encaixantes 
da reta dada por 
\begin{equation}
\label{continencia2-pikm-Cm}
\ldots \subseteq 
\pi_{1}^{3}(C_3)
\subseteq
\pi^{2}_{1}(C_2)
\subseteq 
\pi^{1}_{1}(C_1) 
=
C_1
\end{equation}
tem interseção não vazia.
Seja $x_1$ um ponto arbitrário nesta interseção.
Por construção $x_1\in \pi_{1}^{n}(C_n)$ 
para todo $n\geq 1$. Pela definição da projeção
$\pi_{1}^{n}$ sabemos que     
existe pelo menos um ponto 
$(y_2,\ldots,y_{n})\in \mathbb{R}^{n-1}$ 
tal que $(x_1,y_2,\ldots,y_{n})\in C_n$.
Esta observação mostra que o conjunto 
$C_n(x_1)
\equiv 
\{ 
(y_2,\ldots,y_{n})\in\mathbb{R}^{n-1}
: (x_1,y_2,\ldots,y_{n})\in C_n
\}$ 
é não vazio.
Note que 
$
C_{n}(x_1) 
= 
(\pi_{n}^{1})^{-1}(x_1)\cap C_n.
$
Já que o conjunto unitário $\{x_1\}$ é fechado
segue da continuidade de $\pi_{n}^{1}$ 
que o conjunto 
$C_{n}(x_1)=(\pi_{n}^{1})^{-1}(x_1)\cap C_n$ é a interseção
de um fechado com um compacto e portanto $C_{n}(x_1)$
é compacto.
Além do mais $C_2(x_1), C_3(x_1),\ldots$
satisfaz a mesma condição que satisfaz a 
sequência original $C_1,C_2,\ldots$, isto é, 
para todo $(y_2,\ldots,y_{n+1})\in C_{n+1}(x_1)$
temos que $(y_2,\ldots,y_n)\in C_{n}(x_1)$.
De fato, se $(y_2,\ldots,y_{n+1})\in C_{n+1}(x_1)$
segue da definição de $C_{n+1}(x_1)$ que 
$(x_1,y_2,\ldots,y_{n+1})\in C_{n+2}$. 
De \eqref{continencia-pikm-Cm} segue que 
$(x_1,y_2,\ldots,y_n)\in C_n$ 
logo $(y_2,\ldots,y_{n})\in C_{n}(x_1)$.
Já que $\{C_n(x_1)\}_{n\geq 2}$ possui a propriedade
de $\{C_n\}_{n\geq 1}$ usada para provar 
a continência \eqref{continencia2-pikm-Cm}, 
então podemos garantir que \eqref{continencia2-pikm-Cm}
é também verdadeira para a sequência $\{C_n(x_1)\}_{n\geq 2}$
no lugar de $\{C_n\}_{n\geq 1}$. 
Segue desta sequência de continências,
da compacidade dos elementos da sequência 
$\{C_n(x_1)\}_{n\geq 2}$ e 
do Teorema de Cantor que existe pelo menos 
um ponto $x_2\in\mathbb{R}$ tal que $(x_1,x_2)\in \pi^{n}_2(C_n)$
para todo $n\geq 2$. Procedendo uma indução formal é fácil 
ver que podemos definir uma sequência $x_1,x_2,x_3\ldots$ 
tal que para todo $1\leq m\leq n$ temos 
$(x_1,x_2,\ldots,x_m)\in \pi^{n}_{m}(C_n)$.
Em particular, $(x_1,x_2,\ldots,x_n)\in \pi^{n}_{n}(C_n)=C_n$
para todo $n\geq 1$.
\end{proof}

\subsubsection{Regularidade das Medidas Definidas 
sobre $\mathscr{B}(\mathbb{R})$.}


\subsubsection{O Teorema de Existência de Kolmogorov - Caso Enumerável}

\begin{teorema}
Seja $\{\mathbb{P}_n\}_{n\in\mathbb{N}}$ 
uma família de medidas de probabilidade, 
tal que para cada $n\in\mathbb{N}$ a
medida de probabilidade $\mathbb{P}_n$ é uma medida definida 
sobre $\mathscr{B}(\mathbb{R}^{n})$. 
Suponha que a família $\{\mathbb{P}_n\}_{n\in\mathbb{N}}$  
satisfaz as 
\textbf{condições de consistência de Kolmogorov}, i.e.,
\[
\mathbb{P}_{n+k}(E\times\mathbb{R}^k)
=
\mathbb{P}_n(E)
\quad
\begin{array}{c}
\text{para todo}\ n,k\in\mathbb{N}\ \text{e}\ 
\text{todo boreliano}\ E\subseteq \mathbb{R}^n.
\end{array}
\]
Então existe uma única medida de probabilidade 
$\mathbb{P}$ definida sobre 
$\mathscr{B}\big(\mathbb{R}^{\mathbb{N}}\big)$
tal que para qualquer que seja $n\geq 1$ e 
$E\in \mathscr{B}(\mathbb{R}^{n})$ temos 
a seguinte igualdade 
\[
\mathbb{P}(E\times\mathbb{R}\times\mathbb{R}\times\ldots)
=
\mathbb{P}_n(E).
\]
\end{teorema}
