\chapter[Aula 6]{Variáveis Aleatórias e Independência}
\chaptermark{}

\section{Variáveis Aleatórias}

\begin{definicao}[Variável Aleatória]\label{def-var-aleatoria}
Seja $(\Omega,\F)$ um espaço de medida e $\Lambda\in\F$.
Uma função $X:\Lambda\to\overline{\R}$ tal que para todo 
$B\in\mathscr{B}(\overline{\R})$ temos 
	\[
		\{\omega \in\Lambda: X(\omega)\in B\} 
		\in \Lambda\cap\F,
	\]
onde $\Lambda\cap \F$ denota a coleção de todos os 
subconjuntos de $\Omega$ da forma $\Lambda\cap F$
com $F\in\F$.
\end{definicao}

\begin{observacao}
Esta definição nesta generalidade é necessária por razões
lógicas em algumas aplicações, mas para a discussão das
propriedades básicas de variáveis aleatórias, podemos 
supor que $\Lambda =\Omega$.
\end{observacao}


\begin{exercicio}
Suponha que $\Lambda=\Omega$ na Definição \ref{def-var-aleatoria}.
Mostre que uma variável aleatória é uma função $\F$-mensurável
tomando valores em $\overline{\R}$ no sentido da seção anterior.  
\end{exercicio}

Seja $(\Omega,\F,\P)$ um espaço de probabilidade. 
Se $X:\Omega\to\overline{\R}$ é uma variável aleatória então vamos 
usar a notação
	\[
		\P(X\in B) \equiv \P(\{\omega\in\Omega: X(\omega)\in B\}).
	\]
Vamos usar a abreviação v.a. para nos referir a uma variável aleatória
e ao invés de escrever $X:\Omega\to \overline{\R}$ é uma 
v.a., vamos escrever simplesmente $X$ é uma v.a..
Quando $X(\Omega)\subset \R$ vamos dizer que $X$ é uma 
v.a. real.





\begin{proposicao}
	Se $X$ uma v.a. real em $(\Omega,\F,\P)$ então 
	$\mu:\F\to [0,1]$ dada por 
		\[
			\mu(B)\equiv \P(X^{-1}(B))=\P(X\in B)
		\]
	é uma medida de probabilidade em 
	$(\R,\mathscr{B}(\R))$.
\end{proposicao}

\begin{proof}
Claramente $\mu(B)\geq 0$ para todo $B\in \mathscr{B}(\R)$. 
Se $\{A_n\}$ é uma sequência de conjuntos 
mutuamente disjunta em $\mathscr{B}(\R)$ então 
$\{X^{-1}(A_n)\}$ é uma sequência mutuamente disjunta em $\Omega$, 
portanto 
\begin{align*}
	\mu \left( \bigcup_{n=1}^{\infty} A_n   \right)
	=
	\P \left( X^{-1}\left(\bigcup_{n=1}^{\infty} A_n \right) \right)
	=
	\P \left( \bigcup_{n=1}^{\infty} X^{-1}(A_n) \right)
	=&
	\sum_{n=1}^{\infty}\P(X^{-1}(A_n))
	\\
	=&
	\sum_{n=1}^{\infty}\mu(A_n)
\end{align*}
Já que $X^{-1}(\R)=\Omega$, temos que $\mu(\R)=1$ e
isto encerra a prova de que $\mu$ é uma medida de probabilidade.
\end{proof}

A medida de probabilidade $\mu$ induzida pela v.a. real $X$, 
definida na proposição acima, é frequentemente denotada por 
	\[
		\mu = \P\circ X^{-1}.
	\]
Neste caso, a função distribuição $F$ associada 
a medida $\mu$ é chamada 
{\bf função distribuição de $X$}, 
\index{Função!Distribuição de $X$}
mais especificamente 
	\[
		F(x) = \mu((-\infty,x]) = \P(X\leq x).
	\] 
Quando estivermos lidando com mais de uma v.a. usamos 
a notação $F_X$ para indicar que estamos falando da 
função distribuição de $X$.


\begin{teorema}
	Seja $(\Omega,\F)$ um espaço mensurável. 
	Se $X$ é uma v.a. real e $f:\R\to\R$ é uma 
	função mensurável (com respeito a $\sigma$-álgebra de Borel), 
	então $f(X)$ é uma v.a. real.
\end{teorema}

\begin{proof}
 Segue das propriedades elementares de composição de função que 
 $(f(X))^{-1}(A) = (f\circ X)^{-1}(A) = X^{-1}(f^{-1}(A))$.
 Logo 
 \[
 	(f\circ X)^{-1}(\mathscr{B}(\R))
 	=
 	X^{-1}(f^{-1}(\mathscr{B}(\R)))
 	=
 	X^{-1}(\mathscr{B}(\R))
 	\subset 
 	\F.
 \]
O que completa a demostração.
\end{proof}









\section{Independência}

Independência é uma propriedade básica de eventos e variáveis 
aleatórias em vários modelos de probabilidade. A definição deste
conceito é motivada pelo raciocínio intuitivo de que a ocorrência 
ou não de um evento não afeta nossa estimativa da probabilidade 
que um evento independente ocorra ou não. 
Apesar deste conceito ter um  apelo 
intuitivo é importante entender que independência em Teoria 
da Probabilidade é um conceito técnico com uma definição 
precisa e que deve ser verificada, cada modelo específico
que estiver sendo estudado.
 
Certamente existem exemplos de eventos dependentes que nossa 
intuição nos diz que eles devem ser dependentes e exemplos que 
nossa intuição diz que não devem ser independentes, mas que 
satisfazem a definição. Assim devemos recorrer sempre a definição 
para termos certeza sobre a independência de determinados eventos.  


\begin{center}
	{\red Em andamento.}
\end{center}


