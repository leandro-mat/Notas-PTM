\chapter[Aula 6]{Variáveis Aleatórias e Independência}
\chaptermark{}

\section{Variáveis Aleatórias}

\begin{definicao}[Variável Aleatória]\label{def-var-aleatoria}
Seja $(\Omega,\F)$ um espaço de medida e $\Lambda\in\F$.
Uma função $X:\Lambda\to\overline{\R}$ tal que para todo 
$B\in\mathscr{B}(\overline{\R})$ temos 
	\[
		\{\omega \in\Lambda: X(\omega)\in B\} 
		\in \Lambda\cap\F,
	\]
onde $\Lambda\cap \F$ denota a coleção de todos os 
subconjuntos de $\Omega$ da forma $\Lambda\cap F$
com $F\in\F$.
\end{definicao}

\begin{observacao}
Esta definição nesta generalidade é necessária por razões
lógicas em algumas aplicações, mas para a discussão das
propriedades básicas de variáveis aleatórias, podemos 
supor que $\Lambda =\Omega$.
\end{observacao}


\begin{exercicio}
Suponha que $\Lambda=\Omega$ na Definição \ref{def-var-aleatoria}.
Mostre que uma variável aleatória é uma função $\F$-mensurável
tomando valores em $\overline{\R}$ no sentido da seção anterior.  
\end{exercicio}